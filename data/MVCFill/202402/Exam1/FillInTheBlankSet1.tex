%SECTIONS 12.1 TO 12.2 (3D, vectors)



\ifnum \Version=1
\part If $P$ is the point $(2, 8, 4)$, then the distance between $P$ and the $xy$-plane is $\framebox{\strut\hspace{1cm}}$ and the distance between $P$ and the $y$-axis is $\framebox{\strut\hspace{1cm}}$. 

\ifnum \Solutions=1 {\color{DarkBlue} \textit{Answer:} the point is 4 units above the $xy$-plane, so the first distance is 4. Looking down the $y$-axis, the point is 2 units to the left of the $y$-axis and 8 units above it, so using a right-angle triangle and the Pythgorean theorem the point is $\sqrt{2^2 + 4^2} = \sqrt{20}$ units away for the $y$-axis. 
} 
\else
  
\fi
\fi


\ifnum \Version=2
\part The point on the sphere $(x-2)^2+(y-4)^2+(z-7)^2=4$ nearest to the $xy-$plane is $P=(a,b,c)$, where $a=\framebox{\strut\hspace{.8cm}}, b=\framebox{\strut\hspace{.8cm}}, c=\framebox{\strut\hspace{.8cm}}$. The radius of the sphere is $r = \framebox{\strut\hspace{.8cm}}$. 

\ifnum \Solutions=1 {\color{DarkBlue} \textit{Answer:} $a=2$, $b=4$, $c=5$, $r = 2$. \\[12pt] \textit{Solutions:} The sphere has center $(2,4,7)$ and radius $r = 2$. Because the radius is 2 and the center has $z$--coordinate $z=7$, the sphere lies above the $xy$-plane. The point on the bottom of the sphere closest to the $xy$-plane will be 2 units directly below the center. The coordinate is $(2,4,5)$, so $a=2$, $b=4$, $c=5$. 

} 
\else
  
\fi
\fi






\ifnum \Version=3
\part An equation of the plane that is perpendicular to the $x$-axis and passes through the point $P(3,4,5)$ is $\framebox{\strut\hspace{4cm}}$. 

\ifnum \Solutions=1 {\color{DarkBlue} \textit{Answer:} The plane has equation 
\begin{align}
    \vec n \cdot (\vec x - \vec x_0) & = 0
\end{align}
Where $\vec n$ is a vector normal to the plane, $\vec x_0$ is any point in the plane, and $\vec x $ is the variable vector. We can use: 
\begin{align}
    \begin{pmatrix} 1\\0 \\0 \end{pmatrix} \cdot \left( \begin{pmatrix} x\\y\\z\end{pmatrix}  - \begin{pmatrix}3\\4\\5 \end{pmatrix} \right) & = 0 \\
    x-3 &=0 \\
    x&=3
\end{align}
} 
\else
  
\fi
\fi









\ifnum \Version=4
\part The distance between the plane $x+2y+2z=2$ and the point $S(8,2,1)$ is \framebox{\strut\hspace{1cm}}. The distance between $S$ and the $yz$-plane is $\framebox{\strut\hspace{1cm}}$. 

\ifnum \Solutions=1 {\color{DarkBlue} \textit{Solutions:} A normal to the plane is $\mathbf n = \langle 1,2,2\rangle$, and $|\mathbf{n}| = \sqrt{1^2+2^2+2^2} = \sqrt{9}=3$. A point on the plane can be found by setting $y=z=0$ and solving for $x$. Doing so gives us the point $P(2,0,0)$. Then $\mathbf{PS} = \langle6,2,1\rangle$. 
\begin{align}
    d 
    = \left| \mathbf{PS} \cdot \frac{\mathbf n}{|\mathbf{n}|} \right| 
    = \frac{1}{3}\langle6,2,1\rangle \cdot \langle 1,2,2\rangle = \frac{12}{3} = 4
\end{align}
The point $S$ is 8 units away from the $yz$-plane because the point has coordinates $(8,2,1)$. So the distance between $S$ and the $yz$-plane is 8. 
}

\else

\fi
\fi

% OOOPS!! SHOULDN'T BE HERE
\ifnum \Version=5

\part The cosine of the angle between the vectors $\langle 4,0,3\rangle$  and $\langle 2,1,2\rangle$ is \framebox{\strut\hspace{1cm}}. 

\ifnum \Solutions=1 {\color{DarkBlue} \textit{Solutions:} $\displaystyle \cos\theta = \frac{\langle 4,0,3\rangle \cdot \langle 2,1,2\rangle}{\sqrt{4^2+3^2} \sqrt{2^2+2^2+1}} = \frac{8+0+6}{\sqrt{25}\cdot \sqrt9}= \frac{14}{15}$. 
}
\else
  
\fi
\fi


% OOOPS!! SHOULDN'T BE HERE
\ifnum \Version=6

\part The projection of the point $P(2,3,4)$ onto the $yz$-plane is the point $Q(c_1,c_2,c_3)$ where $c_1 = \framebox{\strut\hspace{1cm}}$, $c_2 = \framebox{\strut\hspace{1cm}}$, $c_3 = \framebox{\strut\hspace{1cm}}$.

\ifnum \Solutions=1 {\color{DarkBlue} \textit{Solutions:} the closest point on the $yz$-plane to the point $P$ will have the same $y$ and $z$ coordinates as $P$, and will have $x$-coordinate of zero. The point is $Q(0,3,4)$, so $c_1 = 0$, $c_2=3$, $c_3=4$. No need for projection formulas. 

}
\else
  
\fi
\fi







\ifnum \Version=7
\part The point on the sphere $(x-2)^2+(y-4)^2+(z-6)^2=4$ nearest to the $yz-$plane is $P=(a,b,c)$, where $a=\framebox{\strut\hspace{1cm}}, b=\framebox{\strut\hspace{1cm}}, c=\framebox{\strut\hspace{1cm}}$.

\ifnum \Solutions=1 {\color{DarkBlue} \textit{Answer:} $a=0$, $b=4$, $c=6$. \\[12pt] \textit{Solutions:} The sphere has center $(2,4,6)$ and radius $2$. Because the radius is 2 and the center has $x$--coordinate $x=2$, the sphere lies to the right of the $yz$-plane. The point on the the sphere closest to the $xy$-plane will be 2 units directly from the center. The coordinate is $(0,4,5)$, so $a=0$, $b=4$, $c=6$. 



} 
\else
  
\fi
\fi


\ifnum \Version=8 %
\part The cosine of the angle between the vectors $\langle 2,3,-6\rangle$  and $\langle 2,2,1\rangle$ is $\cos \theta = \framebox{\strut\hspace{1cm}}$. 

\ifnum \Solutions=1 {\color{DarkBlue} \textit{Answer:} $4/21$ \\[12pt] \textit{Solutions:} $\displaystyle \cos\theta = \frac{\langle 2,3,-6\rangle \cdot \langle 2,2,1\rangle}{\sqrt{2^2+3^2+6^2} \sqrt{2^2+2^2+1}} = \frac{4}{7\cdot 3}= \frac{4}{21}$. 
}
\else
  
\fi
\fi



\ifnum \Version=9

\part The midpoint of the line segment that joins points $P(1,5,2)$ and $Q(9,3,2)$ is $S(a,b,c)$, where $a=\framebox{\strut\hspace{1cm}}$, $b=\framebox{\strut\hspace{1cm}}$, $c=\framebox{\strut\hspace{1cm}}$.

\ifnum \Solutions=1 {\color{DarkBlue} \textit{Solutions:} The midpoint is found by taking the average of the corresponding coordinates of the two points. The midpoint will be the point $(5,4,2)$. The midpoint of a line was covered in Section 12.2. 
}
\else
  
\fi
\fi 


\ifnum \Version=10
\part The equation $x^2 + y^2 - 6y + z^2 + 4z = 12$ represents a sphere whose radius is $r = \framebox{\strut\hspace{1cm}}$. The center of the sphere is at the point $P(a,b,c)$ where $a = \framebox{\strut\hspace{1cm}}$, $b = \framebox{\strut\hspace{1cm}}$, c= $\framebox{\strut\hspace{1cm}}$. 

\ifnum \Solutions=1 {\color{DarkBlue} \textit{Answer:} Completing the square and rearranging like terms: 
\begin{align}
    x^2 + y^2 - 6y + z^2 + 4z &= 12 \\
    x^2 + (y^2 - 6y +9 - 9) + (z^2 + 4z + 4 - 4 )&= 12 \\
    x^2 + (y -3)^2 + (z + 2)^2 - 9 - 4 &= 12 \\
    x^2 + (y -3)^2 + (z + 2)^2 &= 25 
\end{align}
The sphere has radius 5 and is centered at the point $(0,3,-2)$. 

} 
\else
  
\fi
\fi