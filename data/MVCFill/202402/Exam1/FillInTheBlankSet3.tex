% SECTIONS 12.5
\ifnum \Version=1
\part The cosine of the angle between the planes $2x-2y-z=1$ and $x+2y+2z=2$ is \framebox{\strut\hspace{1cm}}.

\ifnum \Solutions=1 {\color{DarkBlue} \textit{Answer:} $-4/9$ \\[12pt] \textit{Solutions:} 

$$\cos \theta = \frac{\langle 2,-2,-1\rangle \cdot \langle 1,2,2\rangle}{\sqrt{2^2+2^2+1^2}\sqrt{1^2+2^2+2^2}}= \frac{-4}{9}$$

} 
\else
  
\fi
\fi

\ifnum \Version=2
\part The distance between the plane $x+4y+8z=1$ and the point $S(4,0,0)$ is \framebox{\strut\hspace{1cm}}.

\ifnum \Solutions=1 {\color{DarkBlue} \textit{Answer:} $1/3$ \\[12pt] \textit{Solutions:} A normal to the plane is $\mathbf n = \langle 1,4,8\rangle$, and $|\mathbf{n}| = \sqrt{1^2+4^2+8^2} = \sqrt{81}=9$. A point on the plane can be found by setting $y=z=0$ and solving for $x$. Doing so gives us the point $P(1,0,0)$. Then $\mathbf{PS} = \langle3,0,0\rangle$. The distance is 
\begin{align}
    d = \left| \mathbf{PS} \cdot \frac{\mathbf n}{|\mathbf{n}|} \right| = \frac{1}{9}\langle3,0,0\rangle \cdot \langle 1,4,8\rangle = \frac13.
\end{align}
} 
\else
  
\fi
\fi


\ifnum \Version=3

\part The plane that passes through $P(5,0,2)$ and contains the line $x=1+t$, $y=t$, $z=2+t$ is \framebox{\strut\hspace{4cm}}. 

\ifnum \Solutions=1 {\color{DarkBlue} \textit{Solutions:} Plane is parallel to direction vector of given line, $\mathbf  v = \langle 1,1,1\rangle$. Line contains $Q(1,0,2)$, so plane also parallel to the vector $\mathbf{PQ} = \langle 1,0,2 \rangle - \langle 5,0,2 \rangle = \langle -4,0,0 \rangle$. It would also be ok to use any scalar multiple of this. \\[12pt]Unit normal to plane is $$\mathbf n = \mathbf  v \times \mathbf {PQ} = \begin{vmatrix} i & j & k \\ 1&1&1 \\ -4&0&0\end{vmatrix} = \langle 0, -4, 4\rangle$$ So using point $P$ and $\mathbf n$, the plane has equation \begin{align}
    (0)(x-5)+(-4)(y-0) + (4)(z-2) = 0 
\end{align}which could be simplified to  $$y-z=2$$ It isn't necessary to simplify the equation further. But we could also express the answer as 
\begin{align}
    y - z + 2 = 0
\end{align}
}
\else
  
\fi
\fi


\ifnum \Version=4

\part The plane that passes through $P(5,0,2)$ and contains the line $x=1+4t$, $y=t$, $z=2+3t$ is \framebox{\strut\hspace{4cm}}. 

\ifnum \Solutions=1 {\color{DarkBlue} \textit{Solutions:} Plane is parallel to direction vector of given line, $\mathbf  v = \langle 4,1,3\rangle$. Line contains $Q(1,0,2)$, so plane also parallel to the vector $\mathbf{PQ} = \langle 1,0,2 \rangle - \langle 5,0,2 \rangle = \langle -4,0,0 \rangle$. It would also be ok to use any scalar multiple of this. \\[12pt]Unit normal to plane is $$\mathbf n = \mathbf  v \times \mathbf {PQ} = \begin{vmatrix} i & j & k \\ 4&1&3 \\ -4&0&0\end{vmatrix} = \langle 0, -12, 4\rangle$$ So using point $P$ and $\mathbf n$, the plane has equation \begin{align}
    (0)(x-5)+(-12)(y-0) + (4)(z-2) = 0 
\end{align}which could be simplified to  $$3y-z=-2$$ It isn't necessary to simplify the equation. 
}
\else
  
\fi
\fi

\ifnum \Version=5

\part The plane that passes through $P(5,0,2)$ and contains the line $x=1+4t$, $y=t$, $z=2+2t$ is \framebox{\strut\hspace{4cm}}. 

\ifnum \Solutions=1 {\color{DarkBlue} \textit{Solutions:} Plane is parallel to direction vector of given line, $\mathbf  v = \langle 4,1,2\rangle$. Line contains $Q(1,0,2)$, so plane also parallel to the vector $\mathbf{PQ} = \langle 1,0,2 \rangle - \langle 5,0,2 \rangle = \langle -4,0,0 \rangle$. It would also be ok to use any scalar multiple of this. \\[12pt]Unit normal to plane is $$\mathbf n = \mathbf  v \times \mathbf {PQ} = \begin{vmatrix} i & j & k \\ 4&1&2 \\ -4&0&0\end{vmatrix} = \langle 0, -8, 4\rangle$$ So using point $P$ and $\mathbf n$, the plane has equation \begin{align}
    (0)(x-5)+(-8)(y-0) + (4)(z-2) = 0 
\end{align}which could be simplified to  $$2y-z=-2$$ It isn't necessary to simplify the equation. 
}
\else
  
\fi
\fi




\ifnum \Version=6
\part The distance between the plane $x+4y+8z=1$ and the point $S(28,0,0)$ is \framebox{\strut\hspace{1cm}}.

\ifnum \Solutions=1 {\color{DarkBlue} \textit{Answer:} $3$ \\[12pt] \textit{Solutions:} A normal to the plane is $\mathbf n = \langle 1,4,8\rangle$, and $|\mathbf{n}| = \sqrt{1^2+4^2+8^2} = \sqrt{81}=9$. A point on the plane can be found by setting $y=z=0$ and solving for $x$. Doing so gives us the point $P(1,0,0)$. Then $\mathbf{PS} = \langle27,0,0\rangle$. The distance is 
\begin{align}
    d 
    = \left| \mathbf{PS} \cdot \frac{\mathbf n}{|\mathbf{n}|} \right| 
    = \frac{1}{9}\langle 27,0,0\rangle \cdot \langle 1,4,8\rangle = 3.
\end{align}
} 
\else
  
\fi
\fi






\ifnum \Version=7
\part The distance between the plane $x+2y+2z=2$ and the point $S(8,2,1)$ is \framebox{\strut\hspace{1cm}}. The distance between $S$ and the $yz$-plane is $\framebox{\strut\hspace{1cm}}$. 

\ifnum \Solutions=1 {\color{DarkBlue} \textit{Solutions:} A normal to the plane is $\mathbf n = \langle 1,2,2\rangle$, and $|\mathbf{n}| = \sqrt{1^2+2^2+2^2} = \sqrt{9}=3$. A point on the plane can be found by setting $y=z=0$ and solving for $x$. Doing so gives us the point $P(2,0,0)$. Then $\mathbf{PS} = \langle6,2,1\rangle$. 
\begin{align}
    d 
    = \left| \mathbf{PS} \cdot \frac{\mathbf n}{|\mathbf{n}|} \right| 
    = \frac{1}{3}\langle6,2,1\rangle \cdot \langle 1,2,2\rangle = \frac{12}{3} = 4
\end{align}
The point $S$ is 8 units away from the $yz$-plane because the point has coordinates $(8,2,1)$. So the distance between $S$ and the $yz$-plane is 8. 
}

\else

\fi
\fi




\ifnum \Version=8

\part The plane that passes through $P(0,-2,2)$ and contains the line $x=1+t$, $y=2t$, $z=2-t$ is \framebox{\strut\hspace{4cm}}. The distance between $P$ and the $xz$-plane is $\framebox{\strut\hspace{1cm}}$. 

\ifnum \Solutions=1 {\color{DarkBlue} \textit{Solutions:} The plane is parallel to direction vector of given line, $\mathbf  v = \langle 1,2,-1 \rangle$. Line contains $Q(1,0,2)$, so plane also parallel to the vector $\mathbf{PQ} = \langle 1,0,2 \rangle - \langle 0,-2,2 \rangle = \langle 1,2,0 \rangle$. It would also be ok to use any scalar multiple of this. \\[12pt]A normal to plane is 

$$\mathbf n 
= \mathbf  v \times \mathbf {PQ} 
= \begin{vmatrix} i & j & k \\ 1&2&-1 \\ 1&2&0\end{vmatrix} 
= \langle 2, -1, 0\rangle$$ 

So using point $P(0,-2,2)$ and $\mathbf n$, the plane has equation \begin{align}
    2(x-0) - (y+2) + (0)(z-2) = 0 
\end{align}
which could be simplified to $2x-y = 2$, but it isn't necessary to simplify the equation. The point $S$ is 2 units away from the $yz$-plane because the point has coordinates $(0,-2,2)$. So the distance between $S$ and the $yz$-plane is 2. 
}
\else
  
\fi
\fi

\ifnum \Version=9
\part The cosine of the angle, $\theta$, between the planes $3x+4z=1$ and $4x+3y=2$ is $\cos \theta 
 =\framebox{\strut\hspace{1cm}}$. 

\ifnum \Solutions=1 {\color{DarkBlue} \textit{Answer:} $-4/9$ \\[12pt] \textit{Solutions:} 

$$\cos \theta = \frac{\langle 3,0,4\rangle \cdot \langle 4,3,0\rangle}{\sqrt{3^2+4^2}\sqrt{3^2+4^2}}= \frac{24}{25}$$

} 
\else
  
\fi
\fi

\ifnum \Version=10

\part The plane that passes through $P(1,3,2)$ and contains the line $x=1+t$, $y=2+3t$, $z=2+2t$ is \framebox{\strut\hspace{4cm}}. The distance between the point $P$ and the $x$-axis is \framebox{\strut\hspace{1cm}}. 

\ifnum \Solutions=1 {\color{DarkBlue} \textit{Solutions:} The plane is parallel to direction vector of given line, $\mathbf  v = \langle 1,3,2\rangle$. Line contains $Q(1,2,2)$, so the plane is also parallel to the vector $\mathbf{PQ} = \langle 1,2,2 \rangle - \langle 1,3,2 \rangle = \langle 0,-1,0 \rangle$. It would also be ok to use any scalar multiple of this. \\[12pt]
A normal to the plane is $$\mathbf n = \mathbf  v \times \mathbf {PQ} = \begin{vmatrix} i & j & k \\ 1&3&2 \\ 0&-1&0\end{vmatrix} = \langle 2,0,-1\rangle = 2\mathbf i-\mathbf k$$ So using point $P(1,3,2)$ and $\mathbf n$, the plane has equation \begin{align}
    (2)(x-1)+(0)(y-3) + (-1)(z-2) = 0 
\end{align}which could be simplified to other forms, such as $$2x-z=0$$ But it isn't necessary to simplify the equation. The distance between the point and the $x$-axis is $\sqrt{3^2+2^2} = \sqrt{13}$.
}
\else
  
\fi
\fi