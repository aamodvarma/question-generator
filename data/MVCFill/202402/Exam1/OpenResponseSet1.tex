\ifnum \Version=1
\question[6] Suppose $R$ is the point $R(0,1,2)$, $L$ is the line $\mathbf r(t) = \langle t,t,t\rangle $. Please show your work. 
\begin{parts}
    \part Construct the equation of the plane that passes through $R$ and contains the line $L$. 
    
\ifnum \Solutions=1 {\color{DarkBlue} \textit{Solutions:} Two points on the line are $O(0,0,0)$ and $Q(1,1,1)$. So two vectors parallel to the plane are $\mathbf {OR} = \langle 0,1,2\rangle$ and $\mathbf{ OQ} = \langle 1,1,1\rangle$. A normal vector to the plane is $$\mathbf {OR} \times \mathbf {OQ} = \begin{vmatrix} \mathbf i&\mathbf j&\mathbf k\\0&1&2\\1&1&1 \end{vmatrix} = \langle -1, 2, -1\rangle$$The plane has equation 
    
    $$-1(x-0) +(2)(y - 1) + (-1)(z-2) = 0$$ 
    
    \textit{ Note: The above is sufficient but we could also rearrange this equation to obtain, for example, $-x+2y-z = 0$. Note also that \textbf{it is not sufficient} to leave the answer as $-1(x-0) +(2)(y - 1) + (-1)(z-2)$, because this is not an equation. The question asked for an equation, not an expression. We need to have an equals sign somewhere in our answer. 
}
    } 
   \else
      \vspace{9cm}
   \fi
   \part Calculate the distance between the point $R$ and the line $L$.
   
    \ifnum \Solutions=1 {\color{DarkBlue} \textit{Solutions:} 
    A point on the line that corresponds to where $t=0$ is $O(0,0,0)$. The line has direction vector $\mathbf v = \langle 1,1,1\rangle$. The distance $d$ can be obtained with the formula 
    \begin{align}
        d &= \frac{|\mathbf{OR}\times \mathbf v|}{|\mathbf v|}
    \end{align}
    And 
    \begin{align}
        \mathbf{OR} &= \langle 0,1,2 \rangle \\
        \mathbf{OR} \times \mathbf v &= \begin{vmatrix} \mathbf i&\mathbf j&\mathbf k \\ 0&1&2 \\ 1&1&1\end{vmatrix} = \langle -1,2,-1\rangle\\
        |\mathbf{OR}\times \mathbf v| &= \sqrt{1+4+1} = \sqrt{6}\\
        |\mathbf v| &= \sqrt{1+1+1} = \sqrt{3}\\
        d &= \frac{|\mathbf{OR}\times \mathbf v|}{|\mathbf v|} = \sqrt{ 2 }
    \end{align}
    } 
    \else
    \vspace{4cm}
    \fi       
\end{parts}

\fi    







\ifnum \Version=2

\question[6] Consider the point $S(4,1,1)$ and the line $L$ whose parametric equations are $x=2+t$, $y=1-3t$, $z=t$. Please show your work for the following. \\

\begin{parts}
    \part Calculate the distance between $S$ and $L$. 

    \ifnum \Solutions=1 {\color{DarkBlue} \textit{Solutions:} 
    A point on the line that corresponds to where $t=0$ is $P(2,1,0)$. The line has direction vector $v = \langle 1,-3,1\rangle$. The distance $d$ can be obtained with the formula 
    \begin{align}
        d &= \frac{|\mathbf{PS}\times v|}{|v|}
    \end{align}
    And 
    \begin{align}
        \mathbf{PS} &= \langle4,1,1\rangle - \langle 2,1,0\rangle = \langle 2,0,1 \rangle \\
        \mathbf{PS} \times v &= \begin{vmatrix} \mathbf i&\mathbf j&\mathbf k \\ 2&0&1 \\ 1&-3&1\end{vmatrix} = \langle 3, -1,-6\rangle\\
        |\mathbf{PS}\times v| &= \sqrt{9+1+36} = \sqrt{46}\\
        |v| &= \sqrt{1+9+1} = \sqrt{11}\\
        d &= \frac{|\mathbf{PS}\times v|}{|v|} = \sqrt{ \frac{46}{11} }
    \end{align}
    } 
    \else
    \vspace{4cm}
    \fi        

    \part Determine the point $P$ where $L$ intersects the plane $x+y+4z=1$. Your answer should be expressed as a point in $\mathbb R^3$. 
    
    \ifnum \Solutions=1 {\color{DarkBlue}
    \textbf{Solutions:} The line will intersect the plane when it satisfies $x+y+4z=1$. Substituting the parametric equations for the line into the plane equation we obtain
    \begin{align}
        (2+t) + (1-3t) + 4(t) & = 1\\
        3 +2t & = 1 \\
        t&= -1
    \end{align}
    With $t=-1$, the parametric equations give us 
    \begin{align}
        x&= 2 - 1 = 1 \\
        y&= 1 -3(-1) = 4 \\
        z&= -1 
    \end{align}
    The point is $P(1,4,-1)$. 
    } 
    \else 
    \newpage
    \fi    
\end{parts}

\fi    






\ifnum \Version=3

\question[6] A particle is located at the point $P(1, 2, 1)$ at time $t = 0$. It travels in a straight line to the point $Q(5, 6, 8)$, has speed 18 at $t=0$, and acceleration $\mathbf a = 4\mathbf i + 4\mathbf{j} + 7\mathbf{k} $. Determine the velocity of the particle, $\mathbf  v(t)$.  Please show your work.

\ifnum \Solutions=1 {\color{DarkBlue} \textit{Answer:} $\langle 4t+8,4t+8,7t+14 \rangle$ \\[12pt] \textit{Solutions:} The velocity $\mathbf v$ is given by the integral of the acceleration, so
    \begin{align}
        \mathbf v = \int \mathbf a(t) \, dt = \langle 4t, 4t, 7t\rangle + \mathbf c_1
    \end{align}
    The speed at $t=0$ is 18 and the particle is moving in the direction of $\langle 4,4,7\rangle$, so 
    \begin{align}
        \mathbf v(0) = \frac{18}{\sqrt{16+16+49}} \langle 4,4,7 \rangle = \frac{18}{\sqrt{81}} \langle 4,4,7 \rangle = \mathbf c_1 \\
        \langle 8,8,14\rangle = \mathbf c_1
    \end{align}
    Thus
    \begin{align}
        \mathbf v = \langle 4t,4t,7t\rangle + \langle 8,8,14\rangle = \langle 4t+8,4t+8,7t+14 \rangle
    \end{align}
}
\else
  
\fi
\fi

\ifnum \Version=4
        
\question[6] Consider the two planes $P_1$ and $P_2$. 
\begin{align}
    P_1: & \quad 2x+3y+6z=2  \\
    P_2: & \quad 4x+6y+12z=6
\end{align}
Please show your work for the following.
\begin{parts}
    \part Calculate the distance between planes $P_1$ and $P_2$.
    
    \ifnum \Solutions=1 {\color{DarkBlue} \textit{Solutions:} The planes are parallel because their normal vectors are parallel. A point on the first plane is $P(1,0,0)$, a point on the other plane is $S(0,1,0)$, then $\mathbf {PS} = \langle -1,1,0 \rangle $, and a normal vector to both planes is $\mathbf n = \langle 2,3,6\rangle$. The distance between that point and the other plane is the length of the projection of vector $\mathbf {PS}$ onto $\mathbf n$. 
    \begin{align} 
    d = \left| \text{proj}_{\mathbf n} \mathbf {PS} \right| 
    &= \left | \frac{ \mathbf {PS} \cdot \mathbf n}{\mathbf n \cdot \mathbf n} \mathbf n \right| \\
    &= \left| \frac{\langle -1,1,0\rangle \cdot \langle 2,3,6\rangle }{2^2+3^3+6^2}  \mathbf n  \right| \\
    &= \left| \frac{1}{49} \mathbf n  \right | \\
    &= \frac{1}{\sqrt{49}} \sqrt{4+9+36} \\
    &= \frac{1}{\sqrt{49}} \sqrt{49}\\
    &= 1
    \end{align}
    The distance between the two planes is $1$.\\[12pt]
    \textbf{Additional Solution Notes}\\
    Note that there are other formulas for the distance between a point and a plane that can be used. One such formula, which is in the textbook, is 
    \begin{align}
        d =  \frac{\left| \mathbf{PS} \cdot \mathbf n \right|}{|\mathbf n|}
    \end{align}
    Yet another formula for the distance between parallel planes $ax+by+cz+d_1=0$ and $ax+by+cz+d_2=0$ is
    \begin{align}
        d =  \frac{\left| d_1 - d_2 \right|}{\sqrt{a^2+b^2+c^2}}
    \end{align}
    Also note that a good way to check your work is to consider other points for P and Q and see if you get the same result for the distance. 
    }
    \else
        \vspace{5cm}
    
    \fi    
    \part Determine the parametric equations for the line of intersection between plane $P_1$ and the plane $P_3$, which is given by $2x+4y+8z=4$.

    \ifnum \Solutions=1 {\color{DarkBlue} 
    \textbf{Solutions:} the set of points that are common to $P_1$ and $P_3$ satisfy the linear system whose augmented matrix is
    \begin{align}
        \begin{pmatrix} 2&3&6&2\\2&4&8&4\end{pmatrix} 
        \sim \begin{pmatrix} 2&3&6&2\\0&1&2&2\end{pmatrix} 
        \sim \begin{pmatrix} 2&0&0&-4\\0&1&2&2\end{pmatrix} 
        \sim \begin{pmatrix} 1&0&0&-2\\0&1&2&2\end{pmatrix} 
    \end{align}
    So if the variable vector $\mathbf x = \langle x,y,z \rangle$ represents the line of intersection, then:
    \begin{align}
        \mathbf x &= \begin{pmatrix}x\\y\\z \end{pmatrix} = \begin{pmatrix} -2 \\2-2z \\z  \end{pmatrix}
    \end{align}
    Letting $z=t$, we have the parametric equations
    \begin{align}
        x&=-2 \\
        y&= 2-2t \\
        z&= t
    \end{align}
    } 
    \else 
    \newpage
    \fi

    
\end{parts}

\fi    


\ifnum \Version=5

\question[6] A particle is located at the point $P(1, 2, 5)$ at time $t = 0$. It travels in a straight line to the point $Q(2, 4, 7)$, has speed 12 at $t=0$, and acceleration $\mathbf a = \mathbf i + 2\mathbf{j} + 2\mathbf{k} $. Determine the velocity of the particle, $\mathbf  v(t)$.  Please show your work.

\ifnum \Solutions=1 {\color{DarkBlue} \textit{Answer:} $\langle t+4, 2t+8, 2t+8 \rangle$ \\[12pt] \textit{Solutions:} The velocity $\mathbf v$ is given by the integral of the acceleration, so
    \begin{align}
        \mathbf v = \int \mathbf a(t) \, dt = \langle t, 2t, 2t\rangle + \mathbf c_1
    \end{align}
    The speed at $t=0$ is 12 and the particle is moving in the direction of $\langle 1,2,2\rangle$, so 
    \begin{align}
        \mathbf v(0) = \frac{12}{\sqrt{1+4+4}} \langle 1,2,2 \rangle = \mathbf c_1 \\
        \langle 4,8,8\rangle = \mathbf c_1
    \end{align}
    Thus
    \begin{align}
        \mathbf v = \langle t, 2t, 2t\rangle + \langle 4,8,8\rangle = \langle t+4, 2t+8, 2t+8 \rangle
    \end{align}
}
\else
  
\fi
\fi



\ifnum \Version=6

\question[6] Consider the point $S(4,1,1)$ and the $L_1$ whose parametric equations are $x=2+t$, $y=1-t$, $z=t$. Please show your work for the following. \\
\begin{parts}
    \part Calculate the distance between the point and line $L_1$. 
    \ifnum \Solutions=1 {\color{DarkBlue} \textit{Solutions:} 
A point on the line that corresponds to where $t=0$ is $P(2,1,0)$. The line has direction vector $v = \langle 1,-1,1\rangle$. The distance $d$ can be obtained with the formula 
\begin{align}
    d &= \frac{|\mathbf{PS}\times v|}{|v|}
\end{align}
And 
\begin{align}
    \mathbf{PS} &= \langle4,1,1\rangle - \langle 2,1,0\rangle = \langle 2,0,1 \rangle \\
    \mathbf{PS} \times v &= \begin{vmatrix} \mathbf i&\mathbf j&\mathbf k \\ 2&0&1 \\ 1&-1&1\end{vmatrix} = \langle 1, -1, -2\rangle\\
    |\mathbf{PS}\times v| &= \sqrt{1+1+4} = \sqrt{6}\\
    |v| &= \sqrt{1+1+1} = \sqrt{3}\\
    d &= \frac{|\mathbf{PS}\times v|}{|v|} = \sqrt{ \frac{6}{3} }=\sqrt2
\end{align}
} 
\else
\vspace{9cm}
\fi    
\part Construct the component equation of the plane that contains $L_1$ and the line $L_2$ whose parametric equations are $x=2+2t$, $y=1+4t$, $z=3t$. 
\ifnum \Solutions=1 {\color{DarkBlue} \\[12pt] 
\textbf{Solutions:} Direction vectors for the lines are: 
\begin{align}
    L_1: \quad \mathbf v  &= \langle 1,-1,1\rangle \\
    L_2: \quad \mathbf u &= \langle 2,4,3 \rangle
\end{align}
The plane contains the direction vectors of both lines, and a vector perpendicular to them is
\begin{align}
    \mathbf n &= \mathbf v \times \mathbf u = \begin{vmatrix} \mathbf i&\mathbf j&\mathbf k \\ 1&-1&1 \\ 2&4&3\end{vmatrix} = \langle -7 , -1, 6 \rangle
\end{align}
A point in both lines is $P(2,1,0)$. The plane has component equation 
\begin{align}
    -7(x - 2) - (y - 1) + 6z = 0
\end{align}
It is not necessary to simplify further. 
} 
\else 
\fi
\end{parts}

\fi    




\ifnum \Version=7

\question[6] Consider the point $S(4,3,1)$ and the $L_1$ whose parametric equations are $x=2+t$, $y=1-2t$, $z=2t$. Please show your work for the following. \\
\begin{parts}
    \part Calculate the distance between the point and line $L_1$. 
    \ifnum \Solutions=1 {\color{DarkBlue} \textit{Solutions:} 
    A point on the line that corresponds to where $t=0$ is $P(2,1,0)$. The line has direction vector $\mathbf v = \langle 1,-2,2\rangle$. The distance $d$ can be obtained with the formula 
    \begin{align}
        d &= \frac{|\mathbf{PS}\times \mathbf v|}{|\mathbf v|}
    \end{align}
    And 
    \begin{align}
        \mathbf{PS} &= \langle 4,3,1 \rangle - \langle 2,1,0 \rangle 
        = \langle 2,2,1 \rangle \\
        \mathbf{PS} \times \mathbf v 
        &= \begin{vmatrix} \mathbf i&\mathbf j&\mathbf k \\ 2&2&1 \\ 1&-2&2\end{vmatrix} 
        = \langle 6, -3, -6\rangle\\
        |\mathbf{PS}\times \mathbf v| &= \sqrt{36+36+9} = \sqrt{81} = 9 \\
        |\mathbf v| &= \sqrt{1+4+4} = \sqrt{9} = 3\\
        d &= \frac{|\mathbf{PS}\times \mathbf v|}{|\mathbf v|} =  \frac{9}{3} = 3
    \end{align}
    } 
    \else
    \vspace{9cm}
\fi    
\part Construct the component equation of the plane that contains $L_1$ and the line $L_2$. The line $L_2$ has the parametric equations $x=2+2t$, $y=1$, $z=-3t$. 
\ifnum \Solutions=1 {\color{DarkBlue} \\[12pt] 
\textbf{Solutions:} Direction vectors for the lines are: 
\begin{align}
    L_1: \quad \mathbf v  &= \langle 1,-2,2\rangle \\
    L_2: \quad \mathbf u &= \langle 2,0,3 \rangle
\end{align}
The plane contains the direction vectors of both lines, and a vector perpendicular to them is
\begin{align}
    \mathbf n &= \mathbf v \times \mathbf u = \begin{vmatrix} \mathbf i & \mathbf j & \mathbf k \\ 1&-2&2 \\ 2&0&-3\end{vmatrix} = \langle 6 - 0 , -(-3-4), 0-4 \rangle = \langle 6, 7, 4 \rangle 
\end{align} 
A point in both lines is $P(2,1,0)$. The plane has component equation 
\begin{align}
    6(x - 2) + 7(y - 1) + 4z = 0
\end{align}
A few notes about the solution to part (b). 
\begin{itemize}
    \item It is not necessary to simplify further. But we could, for example, write the equation of the plane as
$$6x+7y+4z = 19$$
    \item Note that we can use other points instead of $P$, but the point that is used must be in the plane. 
    \item Likewise we can use any two vectors that are parallel to the plane and are not parallel to each other. So instead of using $\mathbf u$ and $\mathbf v$, we can use other vectors. But regardless of which vectors we use, their cross product would need to be some multiple of $\mathbf n = \langle 6,7,4\rangle$. 
\end{itemize}

} 
\else 
\fi
\end{parts}

\fi    






\ifnum \Version=8
\question[6] Suppose $R$ is the point $R(0,1,2)$, $L$ is the line $\mathbf r(t) = \langle 3t,2,4t\rangle $. Please show your work. 
\begin{parts}
    \part Construct the equation of the plane that passes through $R$ and contains the line $L$. 
    
\ifnum \Solutions=1 {\color{DarkBlue} \textit{Solutions:} Two points on the line are $P(0,2,0)$ and $Q(3,2,4)$. So two vectors parallel to the plane are $\mathbf{PR}$ and $\mathbf{QR}$, where 
\begin{align*}
    \mathbf{PR} &= R - P = \langle0,1,2\rangle - \langle 0,2,0\rangle = \langle 0,-1,2\rangle \\
    \mathbf{QR} &= R - Q = \langle0,1,2\rangle - \langle 3,2,4\rangle = \langle -3,-1,-2\rangle 
\end{align*}

A normal vector to the plane is $$\mathbf {PR} \times \mathbf {QR} = \begin{vmatrix} \mathbf i&\mathbf j&\mathbf k\\0&-1&2\\-3&-1&-2 \end{vmatrix} = \langle 4,-6,-3\rangle$$The plane has equation 
    
    $$4(x-0) +(-6)(y - 1) + (-3)(z-2) = 0$$ 
    
    \textit{ Note: The above is sufficient but we could also rearrange this equation to obtain, for example, $$4x-6y-3z+12=0$$
}
    } 
   \else
      \vspace{9cm}
   \fi
   \part Calculate the distance between the point $R$ and the line $L$.
   
    \ifnum \Solutions=1 {\color{DarkBlue} \textit{Solutions:} 
    A point on the line that corresponds to where $t=0$ is $P(0,2,0)$. The line has direction vector 
    $$\mathbf v = \langle 3,0,4\rangle$$
    The distance $d$ can be obtained with the formula 
    \begin{align}
        d &= \frac{|\mathbf{PR}\times \mathbf v|}{|\mathbf v|}
    \end{align}
    And 
    \begin{align}
        \mathbf{PR} &= R - P = \langle 0,1,2 \rangle  - \langle 0,2,0\rangle= \langle 0,-1,2 \rangle \\
        \mathbf{PR} \times \mathbf v 
        &= \begin{vmatrix} \mathbf i&\mathbf j&\mathbf k \\ 0&-1&2 \\ 3&0&4\end{vmatrix} 
        = \langle -4,6,3\rangle\\
        |\mathbf{PR}\times \mathbf v| &= \sqrt{4^2+6^2+3^2} = \sqrt{16+36+9} = \sqrt{61}\\
        |\mathbf v| &= \sqrt{3^2+4^2} = \sqrt{25} = 5\\
        d &= \frac{|\mathbf{OR}\times \mathbf v|}{|\mathbf v|} = \sqrt{ 61 }/5
    \end{align}
    Unsimplified answers are acceptable. It is ok to leave the answer as $\sqrt{61/25}$.
    } 
    \else
    \vspace{4cm}
    \fi       
\end{parts}

\fi    




\ifnum \Version=9

\question[6] Consider the point $S(6,5,2)$ and the $L_1$ whose parametric equations are $x=2+t$, $y=1-2t$, $z=2t$. Please show your work for the following. \\
\begin{parts}
    \part Calculate the distance between the point and line $L_1$. 
    \ifnum \Solutions=1 {\color{DarkBlue} \textit{Solutions:} 
    A point on the line that corresponds to where $t=0$ is $P(2,1,0)$. The line has direction vector $\mathbf v = \langle 1,-2,2\rangle$. The distance $d$ can be obtained with the formula 
    \begin{align}
        d &= \frac{|\mathbf{PS}\times \mathbf v|}{|\mathbf v|}
    \end{align}
    And 
    \begin{align}
        \mathbf{PS} &= \langle 6,5,2 \rangle - \langle 2,1,0 \rangle 
        = \langle 4,4,2 \rangle \\
        \mathbf{PS} \times \mathbf v 
        &= \begin{vmatrix} \mathbf i&\mathbf j&\mathbf k \\ 4&4&2 \\ 1&-2&2\end{vmatrix} 
        = \langle 12, -6, -12 \rangle\\
        |\mathbf{PS}\times \mathbf v| &= \sqrt{144+36+144} = \sqrt{324} = 18 \\
        |\mathbf v| &= \sqrt{1+4+4} = \sqrt{9} = 3\\
        d &= \frac{|\mathbf{PS}\times \mathbf v|}{|\mathbf v|} =  \frac{18}{3} = 6
    \end{align}
    For full credit the cross product and the lengths should be calculated, but unsimplified answers are ok. For example it is ok to leave the answer as $$\frac{\sqrt{324}}{\sqrt{9}}$$
    } 
    \else
    \vspace{9cm}
\fi    
\part Construct the component equation of the plane that contains $L_1$ and the line $L_2$. The line $L_2$ has the parametric equations $x=2+2t$, $y=1$, $z=-3t$. 
\ifnum \Solutions=1 {\color{DarkBlue} \\[12pt] 
\textbf{Solutions:} Direction vectors for the lines are: 
\begin{align}
    L_1: \quad \mathbf v  &= \langle 1,-2,2\rangle \\
    L_2: \quad \mathbf u &= \langle 2,0,-3 \rangle
\end{align}
The plane contains the direction vectors of both lines, and a vector perpendicular to them is
\begin{align}
    \mathbf n &= \mathbf v \times \mathbf u = \begin{vmatrix} \mathbf i & \mathbf j & \mathbf k \\ 1&-2&2 \\ 2&0&-3\end{vmatrix} = \langle 6 , 7, 4 \rangle 
\end{align} 
A point in both lines is $P(2,1,0)$. The plane has component equation 
\begin{align}
    6(x - 2) + 7(y - 1) + 4z = 0
\end{align}
It is not necessary to simplify further. 
} 
\else 
\fi
\end{parts}

\fi    





\ifnum \Version=10
\question[6] Suppose $R$ is the point $R(0,1,2)$, $L$ is the line $\mathbf r(t) = \langle 3t,2,4t\rangle $. Please show your work. 
\begin{parts}
    \part Construct the equation of the plane that passes through $R$ and contains the line $L$. 
    
\ifnum \Solutions=1 {\color{DarkBlue} \textit{Solutions:} Two points on the line are $P(0,2,0)$ and $Q(3,2,4)$. So two vectors parallel to the plane are $\mathbf{PR}$ and $\mathbf{QR}$, where 
\begin{align*}
    \mathbf{PR} &= R - P = \langle0,1,2\rangle - \langle 0,2,0\rangle = \langle 0,-1,2\rangle \\
    \mathbf{QR} &= R - Q = \langle0,1,2\rangle - \langle 3,2,4\rangle = \langle -3,-1,-2\rangle 
\end{align*}

A normal vector to the plane is $$\mathbf {PR} \times \mathbf {QR} = \begin{vmatrix} \mathbf i&\mathbf j&\mathbf k\\0&-1&2\\-3&-1&-2 \end{vmatrix} = \langle 4,-6,-3\rangle$$The plane has equation 
    
    $$4(x-0) +(-6)(y - 1) + (-3)(z-2) = 0$$ 
    
    \textit{ Note: The above is sufficient but we could also rearrange this equation to obtain, for example, $$4x-6y-3z+12=0$$
}
    } 
   \else
      \vspace{9cm}
   \fi
   \part Calculate the distance between the point $R$ and the line $L$.
   
    \ifnum \Solutions=1 {\color{DarkBlue} \textit{Solutions:} 
    The line has direction vector 
    $$\mathbf v = \langle 3,0,4\rangle$$
    The distance $d$ can be obtained with the formula 
    \begin{align}
        d &= \frac{|\mathbf{PR}\times \mathbf v|}{|\mathbf v|}
    \end{align}
    And 
    \begin{align}
        \mathbf{PR} &= \langle 0,-1,2 \rangle \\
        \mathbf{PR} \times \mathbf v 
        &= \begin{vmatrix} \mathbf i&\mathbf j&\mathbf k \\ 0&-1&2 \\ 3&0&4\end{vmatrix} 
        = \langle -4,6,3\rangle\\
        |\mathbf{PR}\times \mathbf v| &= \sqrt{16+36+9} = \sqrt{61}\\
        |\mathbf v| &= \sqrt{9+16} = \sqrt{25}\\
        d &= \frac{|\mathbf{OR}\times \mathbf v|}{|\mathbf v|} = \sqrt{ 61 }/\sqrt{25} = \sqrt{61}/5
    \end{align}
        Unsimplified answers are acceptable. It is ok to leave the answer in the form $\sqrt{61/25}$.
    } 
    \else
    \vspace{4cm}
    \fi       
\end{parts}

\fi    
