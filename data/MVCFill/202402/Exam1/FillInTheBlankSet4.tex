%12.6 

\ifnum \Version=1 
\part Identify the surface $x^2-y+2z^2 = 4$ by type (cone, paraboloid, etc.). \framebox{\strut\hspace{3cm}}.

\ifnum \Solutions=1 {\color{DarkBlue} \textit{Answer:} elliptical paraboloid \\[12pt] \textit{Solutions:} The curve can be expressed as $y=x^2+2z^2 -4$. In the plane $x=0$ the curve is a parabola $y=2z^2-4$. In the plane $z=0$ the curve is a parabola $y=x^2-4$. Both parabolas open over the $y$-axis. It is more accurate to refer to this shape as an elliptical paraboloid, but we'll be nice and give full credit for writing paraboloid. We do this because many people refer to $y = x^2$ and $y=2x^2$ as parabolas. So, for us, it isn't necessary to indicate that the surface is an \textbf{elliptical} paraboloid. It is sufficient to refer to this surface as a paraboloid. But be careful: a hyperbolic paraboloid is not a paraboloid! Do not refer to hyperbolic paraboloid as a paraboloid. 
} 
\else
  
\fi
\fi


\ifnum \Version=2
\part Identify the surface $-x^2+y^2+4z^2 = 0$ by type (cone, paraboloid, etc.). \framebox{\strut\hspace{3cm}}.

\ifnum \Solutions=1 {\color{DarkBlue} \textit{Answer:} elliptical cone \\[12pt] \textit{Solutions:} The curve can be expressed as $x^2=y^2+4z^2$. In the plane $x=k$ for constant $k$, the curve is an ellipse with equation $y^2+4z^2=k$. The surface intersects the plane $z=0$ along two straight lines $x=\pm y$, and likewise the plane $y=0$ along the lines $z = \pm 4z$. \textit{Note that it isn't necessary to indicate that the surface is an \textbf{elliptical} cone. It is sufficient to refer to this surface as a cone. There is no such thing as a hyperbolic cone, in this course.}
} 
\else
  
\fi
\fi



\ifnum \Version=3
\part Identify the surface $x^2+2z^2 = 16+4y-y^2$ by type (cone, ellipsoid, etc.). \framebox{\strut\hspace{4.0cm}}.

\ifnum \Solutions=1 {\color{DarkBlue} \textit{Answer:} ellipsoid\\[12pt] \textit{Solutions:} Rearrange and complete the square. 
\begin{align*}
    x^2+2z^2 &= 16+4y-y^2\\
   16&= y^2-4y+ x^2+2z^2 \\
   16&= y^2-4y+4-4 + x^2+2z^2 \\
   16&= (y-2)^2 -4 + x^2+2z^2 \\
   20&= (y-2)^2 + x^2+2z^2 
\end{align*}
Not a sphere because coefficient in front of squared terms not all the same. Surface is an ellipsoid. We can refer to spheres as ellipsoids (because a sphere is a special case of an ellipsoid), but we can't refer to an ellipsoid as a sphere unless the ellipsoid is a shpere. 
} 
\else
  
\fi
\fi






\ifnum \Version=4
\part Identify the surface $y+z^2 = 1-x^2$ by type (cone, ellipsoid, etc.). \framebox{\strut\hspace{4.6cm}}.

\ifnum \Solutions=1 {\color{DarkBlue} \textit{Answer:} paraboloid. The vertex of the paraboloid is at the point $(0,1,0)$. It is ok to call this an elliptical paraboloid, because a paraboloid is a type of elliptical paraboloid. On the other hand, it is not correct to call this a hyperbolic paraboloid, because that is a very different surface! 
} 
\else
  
\fi
\fi


\ifnum \Version=5
\part Identify the surface $x-y^2-2y=4z^2 $ by type (cone, ellipsoid, etc.). \framebox{\strut\hspace{4cm}}.

\ifnum \Solutions=1 {\color{DarkBlue} \textit{Answer:} paraboloid \\[12pt] \textit{Solutions:} In the plane $x=k$ for constant $k$, the curves are ellipses.But in the plane $y=0$, we have a parabola $x = 4z^2$. So we can call this surface a paraboloid or elliptical paraboloid. It is more accurate to refer to this shape as an elliptical paraboloid, but we'll be nice and give full credit for writing paraboloid. We do this because many people refer to $y = x^2$ and $y=2x^2$ as parabolas. So, for us, it isn't necessary to indicate that the surface is an \textbf{elliptical} paraboloid. It is sufficient to refer to this surface as a paraboloid. But be careful: a hyperbolic paraboloid is not a paraboloid! Do not refer to hyperbolic paraboloid as a paraboloid.
} 
\else
  
\fi
\fi


\ifnum \Version=6
\part Identify the surface $x-y^2+z^2 = 0$ by type (cone, ellipsoid, etc.). \framebox{\strut\hspace{4.6cm}}.

\ifnum \Solutions=1 {\color{DarkBlue} \textit{Answer:} saddle, or hyperbolic paraboloid. It would not be correct to refer to this shape as a paraboloid. A paraboloid is a very different surface. 
} 
\else
  
\fi
\fi

\ifnum \Version=7
\part Identify the surface $x^2-y^2+z = 0$ by type (cone, ellipsoid, etc.). \framebox{\strut\hspace{4.6cm}}.

\ifnum \Solutions=1 {\color{DarkBlue} \textit{Answer:} saddle, or hyperbolic paraboloid. It would not be correct to refer to this shape as a paraboloid. A paraboloid is a very different surface. 
} 
\else
\fi
\fi

\ifnum \Version=8
\part Identify the surface $-x^2-y^2+z = 0$ by type (cone, ellipsoid, etc.). \framebox{\strut\hspace{4.6cm}}.

\ifnum \Solutions=1 {\color{DarkBlue} \textit{Answer:} elliptical paraboloid. 
} 
\else
  
\fi
\fi


\ifnum \Version=9
\part Identify the surface $-x^2+y-2z^2 = 0$ by type (cone, ellipsoid, etc.). \framebox{\strut\hspace{4.6cm}}.

\ifnum \Solutions=1 {\color{DarkBlue} \textit{Answer:} paraboloid, or elliptical paraboloid. 
} 
\else
  
\fi
\fi


\ifnum \Version=10
\part Identify the surface $x-5y^2-2z^2 = 0$ by type (cone, ellipsoid, etc.). \framebox{\strut\hspace{4.6cm}}.

\ifnum \Solutions=1 {\color{DarkBlue} \textit{Answer:} paraboloid, or elliptical paraboloid. 
} 
\else
  
\fi
\fi