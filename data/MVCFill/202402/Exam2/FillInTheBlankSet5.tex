% QUESTIONS FOR THOMAS SECTION 14.6 
\ifnum \Version=1
% THOMAS 14.6
% FROM 2022
\part If $z=z(x,y) = x^2y$, then the differential $dz = \framebox{\strut\hspace{3cm}}$.
\ifnum \Solutions=1 {\color{DarkBlue} \\ \textit{Answer:} $dz=2xy \, dx + x^2 \, dy$ \\[12pt] \textit{Solutions:} Applying the formula for the differential of $z=f(x,y)$ we obtain $dz = \frac{\partial f}{\partial x}dx + \frac{\partial f}{\partial y}dy =  2xy \, dx + x^2 \, dy$.
} 
\else
  
\fi     
\fi


\ifnum \Version=2
% THOMAS 14.6
% FROM 2022
% 
\part The equation of the tangent plane to $z=x^2+4y^2$ at the point $P(2,-1,8)$ is \framebox{\strut\hspace{3cm}}. 

\ifnum \Solutions=1 {\color{DarkBlue} \vspace{6pt} \textit{Answer:} $4(x-2) - 8(y+1) + z-8 =0$ \\[12pt] \textit{Solutions:} Use the formula for the equation to a tangent plane, at a point $P(x_0,y_0)$, to a surface of the form $z =f(x,y)$.
\begin{align*}
    f_x(x_0,y_0) (x-x_0) + f_y(x_0,y_0) (y-y_0) - (z-z_0) &= 0 \\ 
    2x\big|_P(x-2) + 8y\big|_P(y+1) - z+8&=0\\
    2\cdot2(x-2) + 8\cdot 1(y+1) - z+8&=0\\
    4(x-2) + 8(y+1) - z+8&=0
\end{align*}
The work shown above is sufficient. But if you like you can simplify further. 
\begin{align*}
    4(x-2) - 8(y+1) - z + 8 &=0\\
    4x - 8y -8 -8 + 8 & = z\\
    4x - 8y -8 & = z
\end{align*}
} 
\else
\fi
\fi

\ifnum \Version=3
% THOMAS 14.6
% FROM 2022
% 
\part The linearization $L(x,y)$ of $z=4x^2+2y^3$ at $P(2,1)$ is \framebox{\strut\hspace{4cm}}.

\ifnum \Solutions=1 {\color{DarkBlue} \textit{Solutions:} $L(x,y) = f(2,1) + f_x(x-2) + f_y(y-1) = 18+16(x-2)+6(y-1)$. 

} 
\else
  
\fi
\fi

\ifnum \Version=4
\part The unit vector $\mathbf u = \langle a, b \rangle$ points in direction in which $f(x,y)=3y-2x^2$ decreases most rapidly at $P(1,1)$, where $a = \framebox{\strut\hspace{0.8cm}}$, $b = \framebox{\strut\hspace{0.8cm}}$.

\ifnum \Solutions=1 {\color{DarkBlue}  \textit{Solutions:} $- (\nabla f  )\big|_P = - (\langle -4x,3\rangle  ) \big|_P= \langle 4,-3\rangle$. But $\mathbf u$ is a unit vector so $$\mathbf u = \langle 4/5, -3/5\rangle$$ So $a=4/5$ and $b=-3/5$. 

} 
\else
  
\fi
\fi


\ifnum \Version=5

\part The equation of the tangent plane to the surface $z= 5 - x^2 -y^2$ at $P(2,1,0)$ is $ \framebox{\strut\hspace{3.5cm}}.$

\ifnum \Solutions=1 {\color{DarkBlue} \textit{Answer:} $z=10-4x-2y$. \\[12pt] \textit{Solutions:} Set $z =f(x,y) = 4 -x^2 - y^2$. Then
\begin{align*}
    f_x &= -2x , \ f_x(2,1) = -4\\
    f_y &= -2y , \ f_y(2,1) = -2
\end{align*}
    Use the formula for the equation to a tangent plane, at a point $P(x_0,y_0)$, to a surface of the form $z =f(x,y)$.
\begin{align*}
    (z-z_0) &= f_x(x_0,y_0) (x-x_0) + f_y(x_0,y_0) (y-y_0) \\ 
    z- 0 &=f_x(2,1)(x-2) + f_y(2,1)(y-1)\\
    z &= -4(x-2) -2(y-1) 
\end{align*}
The above is sufficient but we can simplify to $z = -4x-2y +10$.
} 
\else
  
\fi

\fi





\ifnum \Version=6 % OK BUT A LOT OF ALGEBRA? 

\part The equation of the tangent plane to $x^2-xy-y^2+z=4$ at $P(2,1,3)$ is $ \framebox{\strut\hspace{3.5cm}}.$

\ifnum \Solutions=1 {\color{DarkBlue} \textit{Solutions:} Set $z =f(x,y) = 4 -x^2+xy +y^2$. Then
\begin{align*}
    f_x &= -2x+y , \ f_x(2,1) = -3\\
    f_y &= x+2y , \ f_y(2,1) = 4
\end{align*}
    Use the formula for the equation to a tangent plane, at a point $P(x_0,y_0)$, to a surface of the form $z =f(x,y)$.
\begin{align*}
    f_x(x_0,y_0) (x-x_0) + f_y(x_0,y_0) (y-y_0) - (z-z_0) &= 0 \\ 
    f_x(2,1)(x-2) + f_y(2,1)(y-1)-(z-3) &=0 \\
    -3(x-2) + 4(y-1) - z + 3 &=0\\
    -3x + 4y -z +6-4+3 &=0 \\
    3x-4y+z&=5
\end{align*}
} 
\else
  
\fi

\fi




% QUESTIONS FOR THOMAS SECTION 14.6 
\ifnum \Version=7
% THOMAS 14.6
% FROM 2022
\part If $z(x,y) = 2xy + x$, then the differential $dz = \framebox{\strut\hspace{3cm}}$. And the equation of the tangent plane to $z$ at the point $P(1,0,1)$ is \framebox{\strut\hspace{3cm}}. 

\ifnum \Solutions=1 {\color{DarkBlue} \textit{Solutions:} there are two parts. \begin{itemize}
    \item Applying the formula for the differential of $z=f(x,y)$ we obtain $$dz = \frac{\partial f}{\partial x}dx + \frac{\partial f}{\partial y}dy =  (2y + 1) \, dx + 2x \, dy$$ 
    \item Use the formula for the equation to a tangent plane, at a point $P(x_0,y_0)$, to a surface of the form $z =f(x,y)$.
\begin{align*}
    f_x(x_0,y_0) (x-x_0) + f_y(x_0,y_0) (y-y_0) - (z-z_0) &= 0 \\ 
    (2y+1)\big|_P(x-1) + 2x\big|_P(y-0) - (z-1) &=0\\
    (x-1) + 2y - z+1&=0\\
    x+2y-z&=0
\end{align*}
Simplification not necessary. There are many other equivalent ways of expressing the plane. 
\end{itemize}
} 
\else
  
\fi     
\fi


\ifnum \Version=8
% THOMAS 14.6
% FROM 2022
% 
\part Suppose $z=2x+4y^2+xy$. Then the differential $dz = \framebox{\strut\hspace{3cm}}$. The equation of the tangent plane to $z$ at the point $P(1,0,2)$ is $\framebox{\strut\hspace{3cm}}$. 

\ifnum \Solutions=1 {\color{DarkBlue} \textit{Solutions:} there are two parts. \begin{itemize}
    \item Applying the formula for the differential of $z=f(x,y)$ we obtain $$dz = \frac{\partial f}{\partial x}dx + \frac{\partial f}{\partial y}dy =  (2 + y) \, dx + (8 y + x)\, dy$$ 
    \item Use the formula for the equation to a tangent plane, at a point $P(x_0,y_0)$, to a surface of the form $z =f(x,y)$.
\begin{align*}
    f_x(x_0,y_0) (x-x_0) + f_y(x_0,y_0) (y-y_0) - (z-z_0) &= 0 \\ 
    (2+y)\big|_P(x-1) + (8y+x)\big|_P(y-0) - (z-2) &=0\\
    2(x-1) + y - z+2&=0\\
    2x+y-z&=0
\end{align*}
Simplification not necessary. There are many other equivalent ways of expressing the plane. 
\end{itemize}
} 
\else
  
\fi     
\fi


\ifnum \Version=9
% THOMAS 14.6
% FROM 2022
% 
\part Suppose $z(x,y)=2x^3 + 2y^2$. Then the differential $dz = \framebox{\strut\hspace{3cm}}$. The linearization of $z(x,y)$ at $P(1,2)$ is $L(x,y) = \framebox{\strut\hspace{4cm}}$. 

\ifnum \Solutions=1 {\color{DarkBlue} \textit{Solutions:} We need the partial derivatives. 
    \begin{align}
        \DZDX & = 6x^2 \\
        \DZDY &= 4y
    \end{align}
    We can use these derivatives for both parts. 
    \begin{itemize}
    \item 
    Applying the formula for the differential of $z=f(x,y)$ we obtain 
    $$dz 
    = \frac{\partial f}{\partial x}dx + \frac{\partial f}{\partial y}dy 
    =  6x^2 \, dx + 4y\, dy
    $$ 
    \item If we let $f=z$, the linearization is 
    \begin{align}
        L(x,y) 
        &= f(1,2) + f_x(1,2)(x-1) + f_y(1,2)(y-1) \\
        &= 9+6(x-1)+8(y-2) \\
        &= 6x + 9y -14
    \end{align}
    
Simplification not necessary. There are many equivalent ways of expressing the linearization. 
\end{itemize}
} 
\else
  
\fi     
\fi




\ifnum \Version=10
% THOMAS 14.6
% FROM 2022
% 
\part Suppose $z(x,y)= x^3 + 2y^2$. Then the differential $dz = \framebox{\strut\hspace{3cm}}$. The linearization of $z(x,y)$ at $P(2,1)$ is $L(x,y) = \framebox{\strut\hspace{4cm}}$. 

\ifnum \Solutions=1 {\color{DarkBlue} \textit{Solutions:} We need the partial derivatives. 
    \begin{align}
        \DZDX & = 3x^2 \\
        \DZDY &= 4y
    \end{align}
    We can use these derivatives for both parts. 
    \begin{itemize}
    \item 
    Applying the formula for the differential of $z=f(x,y)$ we obtain 
    $$dz 
    = \frac{\partial f}{\partial x}dx + \frac{\partial f}{\partial y}dy 
    =  6x^2 \, dx + 4y\, dy
    $$ 
    \item If we let $f=z$, the linearization is 
    \begin{align}
        L(x,y) 
        &= f(2,1) + f_x(2,1)(x-1) + f_y(2,1)(y-1) \\
        &= 10+12(x-2)+4(y-1) \\
        &= 12x + 4y -16
    \end{align}
    
Simplification not necessary. There are many equivalent ways of expressing the linearization. 
\end{itemize}
} 
\else
  
\fi     
\fi




\ifnum \Version=11
% THOMAS 14.6
% FROM 2022
% 
\part Suppose $z=2x+4y^2+xy$. Then the differential $dz = \framebox{\strut\hspace{3cm}}$. The equation of the tangent plane to $z$ at the point $P(1,0,2)$ is $\framebox{\strut\hspace{3cm}}$. 

\ifnum \Solutions=1 {\color{DarkBlue} \textit{Solutions:} there are two parts. \begin{itemize}
    \item Applying the formula for the differential of $z=f(x,y)$ we obtain $$dz = \frac{\partial f}{\partial x}dx + \frac{\partial f}{\partial y}dy =  (2 + y) \, dx + (8 y + x)\, dy$$ 
    \item Use the formula for the equation to a tangent plane, at a point $P(x_0,y_0)$, to a surface of the form $z =f(x,y)$.
\begin{align*}
    f_x(x_0,y_0) (x-x_0) + f_y(x_0,y_0) (y-y_0) - (z-z_0) &= 0 \\ 
    (2+y)\big|_P(x-1) + (8y+x)\big|_P(y-0) - (z-2) &=0\\
    2(x-1) + y - z+2&=0\\
    2x+y-z&=0
\end{align*}
Simplification not necessary. There are many other equivalent ways of expressing the plane. 
\end{itemize}
} 
\else
  
\fi     
\fi