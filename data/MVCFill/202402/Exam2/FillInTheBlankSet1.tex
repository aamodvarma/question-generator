% SECTION 14.1


\ifnum \Version=1
% THOMAS 14.1
% Studio worksheet

\part The equation of the level curve of $f(x,y)=x-y^2+5$ that passes through the point $P(3,2)$ is $\framebox{\strut\hspace{4cm}}.$

\ifnum \Solutions=1 {\color{DarkBlue} 

\textit{Answer:} At point $P$, $$f(3,2) = 4$$
The level curve has equation $$4 = x-y^2+5$$ Further simplification not necessary. It is also ok to express this equation as $$x-y^2+1=0$$
}
\else

\fi

\fi




\ifnum \Version=2
    % THOMAS 14.1
    % VERBATUM 2022
    
    \part Suppose $z(x,y) = \ln(1-x^2-y^2)$. The domain of $z(x,y)$ is \framebox{\strut\hspace{3cm}} and the range of $z(x,y)$ is  \framebox{\strut\hspace{3cm}}. 
    
    \ifnum \Solutions=1 
    
    {\color{DarkBlue} 
        
    \textbf{Domain}: the argument of our logarithmic function is $$1-x^2-y^2$$ The argument of any logarithmic function must be greater than zero for the logarithm to be defined. So the domain is $1-x^2-y^2>0$. \\[4pt] Note that we could also express the domain in other ways. The statement $x^2+y^2<1$ is also sufficient. 
    
    \textbf{Range}: For any $x > 0$, the logarithmic function $f(x) = \ln x$ can take on any value. But the domain of $z=z(x,y) = 1-x^2-y^2$ is restricted to the interior of the unit circle because we found that $x^2+y^2<1$. Note that at the origin, $z = \ln1 = 0$. Away from the origin $z<0$, and as we approach the edge of the unit circle $z \to - \infty$. The range is the interval $(-\infty, 0]$. \\[4pt] Again, note that we could also express the range in other ways. The statement $-\infty < z \le 0$ is also sufficient. 
    }
    \else
    
    \fi
\fi









\ifnum \Version=4
% THOMAS 14.1
% VERBATUM 2022

\part Suppose $f(x,y) = \sqrt{x-y}$. The domain of $f(x,y)$ is \framebox{\strut\hspace{3cm}} and the range of $f(x,y)$ is \framebox{\strut\hspace{3cm}}. 

\ifnum \Solutions=1 {\color{DarkBlue} 

\textit{Answer:} Domain is $y \le x$, and range is the interval $[0,\infty)$. \\[12pt] \textit{Solutions:} 

\textbf{Domain}: the argument of a square root function must be greater or equal to zero, so the domain is $x-y \ge 0$. But we could also express the domain in other ways. The statement $x \ge y$ is also sufficient, or we could use set builder notation, with $\{ (x,y) \in \mathbb R^2 \, | \, x \ge y \}$. 

\textbf{Range}: The square root function can take on any non-negative value. The domain of $z$ is restricted to the region on and below the line $y=x$. The range is the interval $[0, \infty)$. \textit{Note: we could also express the range in other ways. The statement $0 \le f(x,y) < \infty$ is also sufficient. }

} 
\else
  
\fi
\fi



\ifnum \Version=5
% THOMAS 14.1
% VERBATUM 2022

\part Suppose $f(x,y) = \ln(1-x^2-y^2)$. The domain of $f(x,y)$ is \framebox{\strut\hspace{3cm}} and the range of $f(x,y)$ is \framebox{\strut\hspace{3cm}}. 

\ifnum \Solutions=1 {\color{DarkBlue} 

\textit{Solutions:} 

\textbf{Domain}: the argument of a logarithm function must be greater than zero, so the domain is $1-x^2-y^2 > 0$. But we could also express the domain in other ways. The statement $x^2 + y^2 < 1$ is also sufficient, or we could use set builder notation, with $\{ (x,y) \in \mathbb R^2 \, | \, x^2 + y^2 < 1 \}$. 

\textbf{Range}: The domain of $f(x,y)$ is restricted to the region inside the unit circle. At the origin the function $f$ is equal to zero. Away from the origin $f$ is negative, and $f$ tends to negative infinity as we approach the edge of the circle. The range is the set $(\infty,0]$. And we could also express the range in other ways. The statement $-\infty < f(x,y) \le 0$ is also sufficient. 

} 
\else
  
\fi
\fi


\ifnum \Version=6
% THOMAS 14.1
% VERBATUM 2022

\part The domain of $f(x,y) = \ln(9-x^2-y^2)$ is $ \framebox{\strut\hspace{4cm}}.$

\ifnum \Solutions=1 {\color{DarkBlue} \textit{Answer:} We need $9-x^2-y^2 > 0$, or $x^2+y^2 < 9$. 
}
\else
  
\fi



\fi





\ifnum \Version=7
% THOMAS 14.1
% VERBATUM 2022

\part If $f(x,y) = \sqrt{4-x^2-y^2}$, then the domain of $f(x,y)$ is \framebox{\strut\hspace{3cm}}, the range of $f(x,y)$ is \framebox{\strut\hspace{3cm}}. 

\ifnum \Solutions=1 {\color{DarkBlue} 

\textit{Solutions:} 

\textbf{Domain}: the argument of a square root function must be greater or equal to zero, so the domain is $x^2+y^2\le4$. But we could also express the domain in other ways. The statement $4-x^2-y^2 \ge 0$ is also sufficient, or we could use set builder notation, with $\{ (x,y) \in \mathbb R^2 \, | \,x^2+y^2\le4 \}$. 

\textbf{Range}: The square root function can take on any non-negative value. The domain of $z$ is restricted to the interior of the unit circle. At $(0,0)$, $f = \sqrt4 = 2$. As we move away from the origin, $f$ tends to zero. The range is the interval $[0, 2]$. But we could also express the range in other ways. The statement $0 \le f(x,y) \le 2$ is also sufficient. 

} 
\else
  
\fi
\fi



\ifnum \Version=8
% THOMAS 14.1

\part If $\displaystyle f(x,y) = \sqrt{x+y-1}$, then the domain of $f(x,y)$ is \framebox{\strut\hspace{3cm}} and the range of $f(x,y)$ is \framebox{\strut\hspace{3cm}}.

\ifnum \Solutions=1 {\color{DarkBlue} 

\textit{Solutions:} 

\textbf{Domain}: the argument of a square root function must be greater or equal to zero, so the domain is $x+y-1 \ge 0$. But we could also express the domain in other ways. The statement $y \ge 1 - x$ is also sufficient, or we could use set builder notation, with $\{ (x,y) \in \mathbb R^2 \, | \,y \ge 1 - x \}$. 

\textbf{Range}: The square root function can take on any non-negative value. The domain of $f$ is restricted to the region on and above the line $y=1-x$. On the line we have $f=0$. And away from the line the function increases, and become as large as we like. The range is the interval $[0, \infty)$. We could also express the range in other ways. The statement $0 \le f(x,y) < \infty $ is also sufficient. 

} 
\else
  
\fi
\fi


\ifnum \Version=9
% THOMAS 14.1

\part If $\displaystyle f(x,y) = \sqrt{x^2+y-1}$, then the domain of $f(x,y)$ is \framebox{\strut\hspace{3cm}} and the equation of the level curve that passes through the point $(0,5)$ is \framebox{\strut\hspace{3cm}}. 

\ifnum \Solutions=1 {\color{DarkBlue} 

\textit{Solutions:} 

\textbf{Domain}: the argument of a square root function must be greater or equal to zero, so the domain is $x^2+y-1 \ge 0$. But we could also express the domain in other ways. The statement $y \ge 1 - x^2$ is also sufficient, or we could use set builder notation, with $\{ (x,y) \in \mathbb R^2 \, | \,y \ge 1 - x^2 \}$. 

\textbf{Level Curve}: $f(0,5) = +2$, so the level curve is $$2 = \sqrt{x^2+y-1}$$ Further simplification not necessary. 

} 
\else
  
\fi
\fi

\ifnum \Version=10
% THOMAS 14.1

\part If $\displaystyle f(x,y) = \frac{1}{\sqrt{x^2+y-1}}$, then the domain of $f(x,y)$ is \framebox{\strut\hspace{3cm}} and the equation of the level curve that passes through the point $(0,2)$ is \framebox{\strut\hspace{3cm}}. 

\ifnum \Solutions=1 {\color{DarkBlue} 

\textit{Solutions:} 

\textbf{Domain}: the argument of a square root function must be greater or equal to zero, and the denominator cannot be zero. So the domain is $x^2+y-1 > 0$. But we could also express the domain in other ways. The statement $y > 1 - x^2$ is also sufficient, or we could use set builder notation, with $\{ (x,y) \in \mathbb R^2 \, | \,y > 1 - x^2 \}$. 

\textbf{Level Curve}: $f(0,2) = +1$, so the level curve is $$1 = \sqrt{x^2+y-1}$$ Further simplification not necessary. 

} 
\else
  
\fi
\fi


\ifnum \Version=11
% THOMAS 14.1

\part If $\displaystyle f(x,y) = \sqrt{x+y-4}$, then the domain of $f(x,y)$ is \framebox{\strut\hspace{3cm}} and the range of $f(x,y)$ is \framebox{\strut\hspace{3cm}}.

\ifnum \Solutions=1 {\color{DarkBlue} 

\textit{Solutions:} 

\textbf{Domain}: the argument of a square root function must be greater or equal to zero, so the domain is $x+y-4 \ge 0$. But we could also express the domain in other ways. The statement $y \ge 4 - x$ is also sufficient, or we could use set builder notation, with $\{ (x,y) \in \mathbb R^2 \, | \,y \ge 4 - x \}$. 

\textbf{Range}: The square root function can take on any non-negative value. The domain of $f$ is restricted to the region on and above the line $y=4-x$. On the line we have $f=0$. And away from the line the function increases, and become as large as we like. The range is the interval $[0, \infty)$. We could also express the range in other ways. The statement $0 \le f(x,y) < \infty $ is also sufficient. 

} 
\else
  
\fi
\fi