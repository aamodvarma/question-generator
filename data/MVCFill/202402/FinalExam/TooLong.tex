% TOO LONG FOR FINAL EXAM
\ifnum \Version=9

\part Identify the maximum and minimum values of the function $f(x,y) = x^2 + y^2 -x -y + 1$ in the disk $x^2+y^2 \le 1$. 
\ifnum \Solutions=1 {\color{DarkBlue}  \textit{Solutions:} The partials we need are: 
\begin{align*}
    f_x = 2x-1 &=0 \ \quad \Rightarrow \quad x= 1/2 \\
    f_y = 2y-1 &=0 \ \quad \Rightarrow \quad y= 1/2 
\end{align*}
There is a critical point at $(1/2,1/2)$. Along the boundary using Lagrange Multipliers, set $g = x^2 + y^2 -1 = 0$, assuming the origin is not a critical point,
\begin{align*}
    \nabla f = \lambda \nabla g \quad \Rightarrow \quad 
    \begin{pmatrix} 2x-1 \\ 2y-1 \end{pmatrix} = \lambda \begin{pmatrix} 2x\\2y \end{pmatrix} \quad \Rightarrow \quad 
    \begin{pmatrix} 2x(1-\lambda) \\ 2y(1- \lambda) \end{pmatrix} &= \begin{pmatrix} 1 \\ 1 \end{pmatrix} 
\end{align*}
    So $x=y=\frac{1}{2(1-\lambda)}$. Substitute into constraint: 
    \begin{align*}
        1=(\frac{1}{2(1-\lambda)})^2 + (\frac{1}{2(1-\lambda)})^2 &= \frac{1}{2(1-\lambda)^2}\ \quad \Rightarrow \quad (1-\lambda)^2 = \frac{1}{2} \ \quad \Rightarrow \quad
        \lambda = 1 \pm \frac1{\sqrt2}
    \end{align*}
    Substitute into expressions for $x$ and $y$, and evaluate $f(x,y)$ at critical points.
    \begin{align*}
        x=y&=\frac{1}{2(1-\lambda)} = \frac{1}{2(1- ( 1 \pm \frac1{\sqrt2})} =  \pm \frac{\sqrt2}{2}\\
        f(1/2,1/2) &= \frac14+\frac14-\frac12-\frac12+1 = 1/2 \ (min)\\
        f(\frac{\sqrt2}{2},\frac{\sqrt2}{2}) &= \frac24 + \frac24 - \frac{\sqrt2}{2}-\frac{\sqrt2}{2} + 1 = 2-\sqrt2 \\
        f(-\frac{\sqrt2}{2},-\frac{\sqrt2}{2}) &= \frac24 + \frac24 + \frac{\sqrt2}{2}+\frac{\sqrt2}{2} + 1 = 2+\sqrt2 \ (max)
    \end{align*}
    Can also solve using a parameterization of boundary. 
    } 
   \else
      \vspace{8cm}
   \fi
    
\fi



% TOO LONG FOR FINAL EXAM THIS YEAR?
\ifnum \Version=10

\question[4] Identify the maximum and minimum values of the function $f(x,y) = x^2 + y^2 -x -y + 1$ in the disk $x^2+y^2 \le 1$. 
\ifnum \Solutions=1 {\color{DarkBlue}  \textit{Solutions:} The partials we need are: 
\begin{align*}
    f_x = 2x-1 &=0 \ \quad \Rightarrow \quad x= 1/2 \\
    f_y = 2y-1 &=0 \ \quad \Rightarrow \quad y= 1/2 
\end{align*}
There is a critical point at $(1/2,1/2)$. Along the boundary using Lagrange Multipliers, set $g = x^2 + y^2 -1 = 0$, assuming the origin is not a critical point,
\begin{align*}
    \nabla f = \lambda \nabla g \quad \Rightarrow \quad 
    \begin{pmatrix} 2x-1 \\ 2y-1 \end{pmatrix} = \lambda \begin{pmatrix} 2x\\2y \end{pmatrix} \quad \Rightarrow \quad 
    \begin{pmatrix} 2x(1-\lambda) \\ 2y(1- \lambda) \end{pmatrix} &= \begin{pmatrix} 1 \\ 1 \end{pmatrix} 
\end{align*}
    So $x=y=\frac{1}{2(1-\lambda)}$. Substitute into constraint: 
    \begin{align*}
        1=(\frac{1}{2(1-\lambda)})^2 + (\frac{1}{2(1-\lambda)})^2 &= \frac{1}{2(1-\lambda)^2}\ \quad \Rightarrow \quad (1-\lambda)^2 = \frac{1}{2} \ \quad \Rightarrow \quad
        \lambda = 1 \pm \frac1{\sqrt2}
    \end{align*}
    Substitute into expressions for $x$ and $y$, and evaluate $f(x,y)$ at critical points.
    \begin{align*}
        x=y&=\frac{1}{2(1-\lambda)} = \frac{1}{2(1- ( 1 \pm \frac1{\sqrt2})} =  \pm \frac{\sqrt2}{2}\\
        f(1/2,1/2) &= \frac14+\frac14-\frac12-\frac12+1 = 1/2 \ (min)\\
        f(\frac{\sqrt2}{2},\frac{\sqrt2}{2}) &= \frac24 + \frac24 - \frac{\sqrt2}{2}-\frac{\sqrt2}{2} + 1 = 2-\sqrt2 \\
        f(-\frac{\sqrt2}{2},-\frac{\sqrt2}{2}) &= \frac24 + \frac24 + \frac{\sqrt2}{2}+\frac{\sqrt2}{2} + 1 = 2+\sqrt2 \ (max)
    \end{align*}
    Can also solve using a parameterization of boundary. 
    } 
   \else
      \vspace{8cm}
   \fi
    
\fi




% NOT SURE WHERE THIS MIGHT GO??
\ifnum \Version=9
\question[6] Determine the curvature of $\mathbf r(t) = t \mathbf i + t^2 \mathbf j + (t^2+1) \mathbf k$ at the point $P(1,1,2)$.

\ifnum \Solutions=1 {\color{DarkBlue} \textit{Solutions:} we consider two different methods. Either method is sufficient but the first method requires less work. 

    \subsection*{METHOD 1} 
    Use the formula
    $$\kappa = \frac{|\mathbf v \times \mathbf a|}{|\mathbf v|^3}$$
    Then $\mathbf v = \mathbf i +2t\mathbf j + 2t\mathbf k$, and $\mathbf a = 2\mathbf j+2\mathbf k$. Then $$\mathbf v \times \mathbf a = \begin{vmatrix} i&j&k\\ 1&2t&2t \\0&2&2 \end{vmatrix} = \langle 0,-2,2 \rangle $$ And $$\kappa = |\mathbf v \times \mathbf a| / |\mathbf v|^3 = \sqrt{0^2+2^2 +2^2} / \sqrt{1+8t^2}$$ At $P(1,1,2)$, $t=1$ and we get $$\kappa = \frac{\sqrt{8}}{\left(\sqrt{1+8}\right) ^3} = \frac{2\sqrt2}{27}$$
    \subsection*{METHOD 2} 
    Use the formula    
    $$\kappa = \frac{1}{|\mathbf v|}\left|\frac{d\mathbf T}{dt}\right|$$
    First compute $\mathbf v(t)$.
    \begin{align*}
        \mathbf v &= \mathbf r'(t) = \mathbf i +2t\mathbf j + 2t\mathbf k
    \end{align*}
    Therefore $|\mathbf v| = \sqrt{1+4t^2+4t^2 } = \sqrt{1+8t^2}$. At $t=1$, $|\mathbf v| = 3$. Unit tangent vector: 
    \begin{align*}
        \mathbf T &= \frac{\mathbf v(t)}{|\mathbf v(t)|} = \frac{1}{\sqrt{1+8t^2}}(\mathbf i +2t\mathbf j + 2t\mathbf k)\\
        \frac{d\mathbf T}{dt} &= \frac{d}{dt} \left( \left\langle \frac{1}{\sqrt{1+8t^2}}, \frac{2t}{\sqrt{1+8t^2}}, \frac{2t}{\sqrt{1+8t^2}}\right\rangle\right) \\
        &= \left\langle 
        \frac{-8t}{(1+8t^2)^{3/2}}, 
        \frac{2}{\sqrt{1+8t^2}} + \frac{2t}{(1+8t^2)^{3/2}}\left( \frac{-16t}2 \right), 
        \frac{2}{\sqrt{1+8t^2}} + \frac{2t}{(1+8t^2)^{3/2}}\left( \frac{-16t}2 \right)
        \right\rangle 
    \end{align*}    
    At $t=1$ this becomes
    \begin{align*}
        \frac{d\mathbf T}{dt}(1) 
        &= \left\langle 
        \frac{-8}{9^{3/2}}, 
        \frac23 + \frac{2}{9^{3/2}}\left(\frac{-16}{2}\right) , 
        \frac23 + \frac{2}{9^{3/2}}\left(\frac{-16}{2}\right) 
        \right\rangle \\
        &= \left\langle \frac{-8}{27}, \frac23 - \frac{16}{27}, \frac23 - \frac{16}{27}\right\rangle\\
        &= \frac{1}{27}\left\langle-8, 2, 2\right\rangle\\
        \left|\frac{d\mathbf T}{dt}(1) \right| &= \frac{1}{27}\sqrt{64+4+4} \\
        &= \frac{6\sqrt2}{27}
    \end{align*}        
    Therefore
    $$\kappa = \frac{1}{|\mathbf v|}\left|\frac{d\mathbf T}{dt}\right| = \frac{1}{3}\frac{6\sqrt2}{27} = \frac{2\sqrt2}{27}$$
    } 
   \else
      \vspace{6cm}
   \fi
\fi    

