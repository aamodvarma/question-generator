% SOMETHING FROM 16.6 TO 16.8

% READY
\ifnum \Version=1
    \part The integral of $G = x + y + z$ over the portion of the plane $2x + 2y + z = 2$ in the first octant is $\int_a^b \int_c^d h(x,y) \, dy \, dx$, where $a = \framebox{\strut\hspace{0.65cm}}, b = \framebox{\strut\hspace{0.65cm}}, c = \framebox{\strut\hspace{1cm}}, d = \framebox{\strut\hspace{1cm}}, h = \framebox{\strut\hspace{2.2cm}}$. 
    
    \ifnum \Solutions=1 {\color{DarkBlue}  \textit{Solutions:} This is a surface integral and we can use: 
    \begin{align}
        \iint_S G(x,y,z) \, d\sigma 
    = \iint_S G(x,y,z) \, \frac{|\nabla F|}{|\nabla F \cdot \mathbf p| } \, dA    
    \end{align}
    We can set $F = 2x+2y+z=2$, then
    \begin{align*}
        \nabla F & = \langle2,2,1\rangle \\
        |\nabla F | &= 3 \\
        \mathbf p &= \mathbf k \\
        \frac{|\nabla F|}{|\nabla F \cdot \mathbf p|} & = \frac{3}{1} = 3 \\
        G(x,y,z(x,y)) &= x + y  + (2 - 2x - 2y) = 2 - x - y
    \end{align*}
    The plane $2x + 2y + z = 2$ cuts the $xy-$plane on the line $y = 1 - x$. Thus
    \begin{align}
        \int_a^b \int_c^d h(x,y) \, dy \, dx &= \int_0^1 \int_0^{1-x} (2-x-y) \, 3 \, dy \, dx
    \end{align}
    Thus
    \begin{align}
        a &=0 , \quad 
        b =1, \quad 
        c =0, \quad 
        d = 1-x, \quad 
        h = 6 - 3x - 3y
    \end{align}
    Another way to obtain the surface area element $d\sigma$ is to use $\mathbf r(x,y) = x\mathbf i + y \mathbf j + (2-2x-2y)\mathbf k$, and compute $d\sigma = |\mathbf r_x \times \mathbf r_y| dy\,dx$. 
    } 
   \else
      
   \fi
\fi 


\ifnum \Version=2
    \part The outward flux of $\mathbf F=\langle z,y,x\rangle$ across the unit sphere is equal to \framebox{\strut\hspace{1.5cm}}. 

    \ifnum \Solutions=1 {\color{DarkBlue}  \textit{Solutions:} using the divergence theorem,
    \begin{align*}
        \nabla \cdot \mathbf F &= \DDX(z) + \DDY(y) + \DDZ(x) = 0 + 1 + 0 = 1\\
        \text{outward flux} &= \iint_S \mathbf F\cdot \mathbf n \, d\sigma = \iiint \nabla \cdot \mathbf F \, dV = \iiint 1 \, dV 
    \end{align*}
    The last integral is just the volume of the unit sphere which is $\frac{4\pi}3$. But if you prefer you could work out the volume using spherical coordinates: 
    \begin{align*}
        \iiint \nabla \cdot \mathbf F \, dV &= \iiint 1 \, dV \\
        &= \int_0^{2\pi} \int_0^{\pi} \int_0^1 \rho^2 \sin\phi \, d\rho d\phi d\theta \\
        &= \int_0^{2\pi} \int_0^{\pi} \frac13 \sin\phi \, d\phi d\theta \\
        &= \frac 13 \int_0^{2\pi} \left. -  \cos\phi \right|_0^{\pi}  d\theta \\
        &= \text{we should finish writing this out one day}
    \end{align*}    
    } 
   \else
      
   \fi
\fi 








\ifnum \Version=3
    \part Suppose $\mathbf F(x,y,z) = \langle 5x, -y, z\rangle$, and $S$ is the surface of the solid bounded by $x^2+y^2=1$, $z=1$ and $z=4$. The flux of $\mathbf F$ across $S$ is \framebox{\strut\hspace{1.5cm}}. 
    
    \ifnum \Solutions=1 {\color{DarkBlue}  \textit{Solutions:} using the divergence theorem,
    \begin{align*}
        \iint_S \mathbf F\cdot \mathbf n \, d\sigma &= \iiint_D \nabla \cdot \mathbf F \, dV \\ &= \iiint_D (5-1 +1) \, dV \\
        &= 5 \iiint_D  \, dV \\
        &= 5\cdot(\text
        {volume of cylinder}) \\
        &= 5\cdot 3\pi \\
        &= 15\pi
    \end{align*}
    } 
   \else
      
   \fi
\fi 




\ifnum \Version=4
    \part If $f(x,y,z) = xyz$, then $\text{curl grad }f = \nabla \times \nabla f = p(x,y,z)\mathbf i + q(x,y,z)\mathbf j + r(x,y,z)\mathbf k$, where $p(x,y,z) = \framebox{\strut\hspace{2cm}}$, and $q(x,y,z) = \framebox{\strut\hspace{2cm}}$. 
    
    \ifnum \Solutions=1 {\color{DarkBlue} The curl of a gradient of any function with second partials is the zero vector. So without computation we should get $p=q=r = 0$. 
    } 
   \else
      
   \fi
\fi 




\ifnum \Version=5
    \part If $\mathbf F(x,y,z) =  xyz\mathbf i -  4x^2yz\mathbf j + z\mathbf k$, then $\text{curl }\mathbf F = f(x,y,z)\mathbf i + g(x,y,z)\mathbf j + h(x,y,z)\mathbf k$, where $f(x,y,z) = \framebox{\strut\hspace{2cm}}$, and $g(x,y,z) = \framebox{\strut\hspace{2cm}}$. 
    
    \ifnum \Solutions=1 {\color{DarkBlue} The curl is the cross product
    \begin{align*}
        \text{curl }\mathbf F &= \begin{vmatrix} \mathbf i & \mathbf j & \mathbf k \\ \DDX & \DDY & \DDZ \\ xyz & -4x^2yz & z \end{vmatrix} 
        = 4x^2y \mathbf i - (0 - xy)\mathbf j + h(x,y,z)\mathbf k
        = 4x^2y \mathbf i + xy\mathbf j + h(x,y,z)\mathbf k 
    \end{align*}
    Thus $f(x,y,z) = 4x^2y$ and $g(x,y,z) = xy$. 
    } 
   \else
      
   \fi
\fi 




\ifnum \Version=6
    \part If $S$ is the surface of the solid bounded by $x^2+y^2=1$, $z=-1$ and $z=x+y+10$, and $\mathbf F(x,y,z) = \langle 5x, x-y, 2-4z\rangle$, then the flux of $\mathbf F$ across $S$ is \framebox{\strut\hspace{1.5cm}}. 
    
    \ifnum \Solutions=1 {\color{DarkBlue}  \textit{Solutions:} using the divergence theorem,
    \begin{align*}
        \iint_S \mathbf F\cdot \mathbf n \, d\sigma = \iiint \nabla \cdot \mathbf F \, dV = \iiint (5-1 -4) \, dV = 0
    \end{align*}
    } 
   \else
      
   \fi
\fi 





\ifnum \Version=7
    \part Suppose $\mathbf F(x,y,z) = \langle 2x, x-y, 2z\rangle$, and $S$ is the surface of the solid bounded by $x^2+y^2=1$, $z=1$ and $z=4$. The flux of $\mathbf F$ across $S$ is \framebox{\strut\hspace{1.5cm}}. 
    
    \ifnum \Solutions=1 {\color{DarkBlue}  \textit{Solutions:} using the divergence theorem,
    \begin{align*}
        \iint_S \mathbf F\cdot \mathbf n \, d\sigma &= \iiint_D \nabla \cdot \mathbf F \, dV \\ &= \iiint_D (2-1 +2) \, dV \\
        &= 3\iiint_D  \, dV \\
        &= 3\cdot(\text
        {volume of cylinder}) \\
        &= 3\cdot 3\pi \\
        &= 9\pi
    \end{align*}
    } 
   \else
      
   \fi
\fi 

\ifnum \Version=8
    \part The integral of $G = x + y + z$ over the portion of the plane $2x + 2y + z = 2$ in the first octant is $\int_a^b \int_c^d h(x,y) \, dy \, dx$, where $a = \framebox{\strut\hspace{0.65cm}}, b = \framebox{\strut\hspace{0.65cm}}, c = \framebox{\strut\hspace{1cm}}, d = \framebox{\strut\hspace{1cm}}, h = \framebox{\strut\hspace{2.2cm}}$. 
    
    \ifnum \Solutions=1 {\color{DarkBlue}  \textit{Solutions:} This is a surface integral and we can use: 
    \begin{align}
        \iint_S G(x,y,z) \, d\sigma 
    = \iint_S G(x,y,z) \, \frac{|\nabla F|}{|\nabla F \cdot \mathbf p| } \, dA    
    \end{align}
    We can set $F = 2x+2y+z=2$, then
    \begin{align*}
        \nabla F & = \langle2,2,1\rangle \\
        |\nabla F | &= 3 \\
        \mathbf p &= \mathbf k \\
        \frac{|\nabla F|}{|\nabla F \cdot \mathbf p|} & = \frac{3}{1} = 3 \\
        G(x,y,z(x,y)) &= x + y  + (2 - 2x - 2y) = 2 - x - y
    \end{align*}
    The plane $2x + 2y + z = 2$ cuts the $xy-$plane on the line $y = 1 - x$. Thus
    \begin{align}
        \int_a^b \int_c^d h(x,y) \, dy \, dx &= \int_0^1 \int_0^{1-x} (2-x-y) \, 3 \, dy \, dx
    \end{align}
    Thus
    \begin{align}
        a &=0 , \quad 
        b =1, \quad 
        c =0, \quad 
        d = 1-x, \quad 
        h = 6 - 3x - 3y
    \end{align}
    Another way to obtain the surface area element $d\sigma$ is to use $\mathbf r(x,y) = x\mathbf i + y \mathbf j + (2-2x-2y)\mathbf k$, and compute $d\sigma = |\mathbf r_x \times \mathbf r_y| dy\,dx$. 
    } 
   \else
      
   \fi
\fi 

\ifnum \Version=9
    \part The area of the portion of the plane $y+2z = 2$ inside the cylinder $x^2+y^2=1$ is $\int_0^b \int_c^d h(r,\theta) \, dr \, d\theta$, where $b = \framebox{\strut\hspace{1cm}}, c = \framebox{\strut\hspace{1.5cm}}, d = \framebox{\strut\hspace{1.5cm}}, h = \framebox{\strut\hspace{2.2cm}}$. 
    
    \ifnum \Solutions=1 {\color{DarkBlue}  \textit{Solutions:} This is a surface integral and we can use: 
    \begin{align}
        \iint_S  G \, d\sigma 
    = \iint_S  \, \frac{|\nabla F|}{|\nabla F \cdot \mathbf p| } \, dA    , \ G = 1
    \end{align}
    We can set $F = y+2z=2$, then
    \begin{align*}
        \nabla F & = \langle0,1,2\rangle \\
        |\nabla F | &= \sqrt5 \\
        \mathbf p &= \mathbf k \\
        \frac{|\nabla F|}{|\nabla F \cdot \mathbf p|} & = \frac{\sqrt 5}{2} 
    \end{align*}
    The plane is inside the cylinder with unit radius, so $0 \le \theta \le 2\pi$, $0 \le r \le 1$. 
    \begin{align}
        \int_a^b \int_c^d h(x,y) \, r \, dr \, d\theta &= \int_0^{2\pi} \int_0^{1} \frac{\sqrt 5}{2} \, r \, dr \, d\theta
    \end{align}
    Thus
    \begin{align}
        b =2\pi, \quad 
        c =0, \quad 
        d = 1, \quad 
        h = \frac{\sqrt 5 r }{2}
    \end{align}
    Another way to obtain the surface area element $d\sigma$ is to use $\mathbf r(x,y) = x\mathbf i + y \mathbf j + (2-2x-2y)\mathbf k$, and compute $d\sigma = |\mathbf r_x \times \mathbf r_y| dy\,dx$. 
    } 
   \else
      
   \fi
\fi 



\ifnum \Version=10
    \part Suppose $\mathbf F(x,y,z) = \langle 5x, 5y, 2z\rangle$, and $S$ is the surface of the solid bounded by $x^2+y^2=4$, $z=2$ and $z=3$. Then the flux of $\mathbf F$ across $S$ is $\int_A^B \int_C^D \int_H^K g(r,\theta,z) \, dz \, dr\, d\theta$, where $B = \framebox{\strut\hspace{2cm}}, D = \framebox{\strut\hspace{2cm}}, K = \framebox{\strut\hspace{2cm}}, g = \framebox{\strut\hspace{2cm}}$. 
    
    \ifnum \Solutions=1 {\color{DarkBlue}  \textit{Solutions:} using the divergence theorem,
    \begin{align*}
        \iint_S \mathbf F\cdot \mathbf n \, d\sigma &= \iiint_D \nabla \cdot \mathbf F \, dV \\ &= \iiint_D (5+5+2) \, dV \\
        &= \int_0^{2\pi} \int_0^2 \int_2^3 12 r \, dz \, dr\, d\theta \\
    \end{align*}
    Thus
    \begin{align}
        B =2\pi, \quad 
        D = 2, \quad 
        K = 3, \quad 
        h = 12r
    \end{align}    
    } 
   \else
      
   \fi
\fi 


\ifnum \Version=11
    \part Suppose $C$ is the curve $x^2 + y^2 =4$ in the $xy-$plane, counterclockwise when viewed from above. Using Stokes' Theorem, the circulation of the field $\mathbf F = x^2\mathbf i + 2x\mathbf j + z^2 \mathbf k$ around the curve $C$ is $\int_A^B\int_C^D h(r,\theta) \, dr \, d\theta$, where $B = \framebox{\strut\hspace{1.5cm}}, D = \framebox{\strut\hspace{1.5cm}}, h = \framebox{\strut\hspace{1.5cm}}$. 

    \ifnum \Solutions=1 {\color{DarkBlue}  \textit{Solutions:} using Stokes' theorem,
    \begin{align*}
        \nabla \times \mathbf F &= \begin{vmatrix} \mathbf i& \mathbf j & \mathbf k \\ \DDX & \DDY & \DDZ \\ x^2 & 2x & z^2 \end{vmatrix} = 2\mathbf k \\
        \mathbf n &= \mathbf k\\
        \textbf{curl}\mathbf F \cdot \mathbf n &= 2 \\
        \oint_C \mathbf F\cdot d\mathbf r &= \iint_S \nabla \times \mathbf F\cdot \mathbf n \, d\sigma = \int_0^{2\pi} \int_0^2 2 r \, dr \, d\theta
    \end{align*}
    Thus
    \begin{align}
        B =2\pi, \quad 
        D = 2, \quad 
        h = 2r
    \end{align}        
    } 
   \else
      
   \fi
\fi 

\ifnum \Version=12
    \part The outward flux of $\mathbf F=\langle y,x,y+3z\rangle$ across the unit sphere is equal to \framebox{\strut\hspace{1.5cm}}. 

    \ifnum \Solutions=1 {\color{DarkBlue}  \textit{Solutions:} using the divergence theorem,
    \begin{align*}
        \nabla \cdot \mathbf F &= \DDX(y) + \DDY(x) + \DDZ(y+3z) = 3 \\
        \text{outward flux} &= \iint_S \mathbf F\cdot \mathbf n \, d\sigma = \iiint \nabla \cdot \mathbf F \, dV = \iiint 3 \, dV 
    \end{align*}
    The last integral is three times the volume of the unit sphere which is $4\pi$. 
    } 
   \else
      
   \fi
\fi 





\ifnum \Version=13
    \part If $S$ is the surface of the solid bounded by $x^2+y^2=1$, $z=2$ and $z=10$, and $\mathbf F(x,y,z) = \langle 5x, x-y, 2-z\rangle$, then the flux of $\mathbf F$ across $S$ is \framebox{\strut\hspace{1.5cm}}. 
    
    \ifnum \Solutions=1 {\color{DarkBlue}  \textit{Solutions:} using the divergence theorem,
    \begin{align*}
        \iint_S \mathbf F\cdot \mathbf n \, d\sigma 
        &= \iiint \nabla \cdot \mathbf F \, dV \\
        &= \iiint (5-1 -1) \, dV \\
        &= 3 \iiint_D \, dV \\
        &= 3\cdot \text{(volume of cylinder)}\\
        &= 3\cdot 8 \cdot (\pi r^2), \quad r = 1 \\
        &= 24\pi
    \end{align*}
    } 
   \else
      
   \fi
\fi 


\ifnum \Version=14
    \part If $f(x,y,z) = x^2yz$, then $\text{curl grad }f = \nabla \times \nabla f = p(x,y,z)\mathbf i + q(x,y,z)\mathbf j + r(x,y,z)\mathbf k$, where $p(x,y,z) = \framebox{\strut\hspace{2cm}}$, and $q(x,y,z) = \framebox{\strut\hspace{2cm}}$. 
    
    \ifnum \Solutions=1 {\color{DarkBlue} The curl of a gradient of any function with second partials is the zero vector. So without computation we should get $p=q=r=0$. 
    } 
   \else
      
   \fi
\fi 