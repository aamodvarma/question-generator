\documentclass[11pt]{exam}

% LOAD PACKAGES
\usepackage{amsmath} % allows for align env and other things
\usepackage{amssymb} % 
\usepackage{mathtools} % allows for single apostrophe
\usepackage{enumitem} % allows for alpha lettering in enumerated lists
\usepackage{lastpage}
\usepackage{array} % for table alignments
\usepackage{graphicx} % if images are needed

\addpoints

% Initials
\newcommand{\Initials}{\textit{\Course, \TestName. Your initials: \underline{\hspace{3cm}}} \vspace{1pt}}
\newcommand{\InitialsLeft}{\noindent \hspace{-18pt}\textit{\Course, \TestName. Your initials: \underline{\hspace{3cm}}} \vspace{1pt}}
\newcommand{\InitialsRight}{\begin{flushright}\textit{\Course, \TestName. Your initials: \underline{\hspace{3cm}}} \vspace{1pt}\end{flushright}}


% INSTRUCTIONS FOR DISTANCE LEARNING COURSES: PROCTORS/FACILITATORS
\newcommand{\InstructionsDistanceProctors}{\begin{itemize}

    \item A proctor/facilitator is required to be present while the student is taking this test.
    
    \item A proctor/facilitator must connect with the on-campus class during this test.   
        
    \item For any connection help: gtonlinesupport@pe.gatech.edu
    
    \item No communication is allowed among students taking the exam.

\end{itemize}}


% INSTRUCTIONS FOR DISTANCE LEARNING COURSES: STUDENTS
\newcommand{\InstructionsDistanceStudents}{\begin{itemize} \setlength\itemsep{.15em}
    
    % \item If there are questions during the exam, students can call/text \InstructorContact, at \InstructorNumber, under supervision of their proctor/facilitator.     
        
    \item {\bf Show your work} and justify your answers for all questions unless stated otherwise.

    \item You will have \Duration continuous minutes (no breaks) to take the exam. 
    
    \item There are \Points total points possible.

    \item Calculators, notes, cell phones, books are not allowed.
    
    \item Please write your answers neatly and show all of your work. 
    \item Use dark and clear writing: your exam will be scanned into a digital system.
    
    \item Exam pages are double sided. Be sure to complete both sides. 
    
    \item Leave a 1 inch border around the edges of exams.
    
    \item Check that every page has the same booklet number.

\end{itemize}}


% INSTRUCTIONS FOR DISTANCE LEARNING COURSES: STUDENTS
\newcommand{\InstructionsCovid}{\begin{itemize} \setlength\itemsep{.15em}
    
    \item If there are questions during the exam, students can email their instructor or message them through Canvas. 
    
    % \InstructorContact, at \InstructorNumber, under supervision of their proctor/facilitator.     
        
    \item {\bf Show your work} and justify your answers for all questions unless stated otherwise.

    \item You will have \Duration continuous minutes (no breaks) to take the exam. 
    
    \item There are \Points total points possible.
    
    \item Your work must be your own. Please do not communicate with anyone other than the instructor during the exam. Otherwise, students can use any resources available to them the answer the questions that are given. 
    
    \item Students can take this exam at home.
    
    \item Please write your answers neatly and show all of your work. 
    
    \item Students should scan their work and submit it through Canvas. There should be an \textbf{assignment} in Canvas for this exam. The process for submitting your work will be similar to what you have used for homework. 
    
    \item Work must be submitted today by 12:30 ET. 
    
    \item Please use dark and clear writing so that the scan is easy to read. 
    
    \item Please write your name or initials at the top of every page and solve the questions in the exam in the order they are given. 
    
    \item There is no need to print the exam at home. As long as you solve problems in the order they are given (just like the written homework sets), you can write out your work on your own paper. But of course, you can print the exam out if you prefer and write your answers on the printed copy. 
    
    \item Please upload your work as a single PDF file. If this is not possible you can email your work to your instructor. 


    
\end{itemize}}

% FANCY HEADERS - MAKE EMPTY
\pagestyle{headandfoot}
\runningfooter{}{}{}


% ADJUST MARGINS FOR DISTANCE LEARNING REQUIREMENTS
\usepackage[tmargin=1.7in,bmargin=1.05in,left=1in,right=1in]{geometry}


% TIKZ DIAGRAMS
\usepackage{color}
\usepackage{tikz}  \usetikzlibrary{arrows} 
\usetikzlibrary{calc} 


% ADJUST FIRST LINE IN PARAGRAPH INDENTATION 
\setlength\parindent{0pt}


% COURSE SPECIFIC INFORMATION
\newcommand{\Course}{Math 2552}
\newcommand{\Semester}{Spring}
\newcommand{\Year}{2020}
\newcommand{\Instructors}{Dr. Greg Mayer}

% WHO TO CONTACT DURING EXAM IF QUESTIONS
\newcommand{\InstructorContact}{Dr. Greg Mayer}


\usepackage{spalign} % Joe Rabinoff's matrix package

\newcommand{\LastPage}{\begin{center}\textit{This page may be used for scratch work. Please indicate clearly if you would like your work on this page to be graded. }\end{center}   }

\newcommand{\Scratch}{\begin{center}\textit{This page may be used for scratch work for the previous question. Your proctor should scan this page whether it is used or not. }\end{center}   }


% DERIVATIVES
\newcommand{\dydt}{{\frac{dy}{dt}}} % 
\newcommand{\dydx}{{\frac{dy}{dx}}} % 
\newcommand{\dydtt}{{\frac{d ^2y}{dt^2}}} % 
\newcommand{\dydxx}{{\frac{d^2y}{dx^2}}} % 
\newcommand{\dydttt}{{\frac{d^3y}{dt^3}}} % 

\newcommand{\ddt}{{\frac{d}{dt}}} % 
\newcommand{\ddx}{{\frac{d}{dx}}} % 
\newcommand{\dudt}{{\frac{du}{dt}}} % 
\newcommand{\dvdx}{{\frac{dv}{dx}}} % 
\newcommand{\dxdt}{{\frac{dx}{dt}}} % 
\newcommand{\dxdtt}{{\frac{d^2x}{dt^2}}} % 
\newcommand{\dzdt}{{\frac{dz}{dt}}} % 



% TEST SPECIFIC INFORMATION
\newcommand{\TestName}{Final Exam}
\newcommand{\TestDate}{Apr 23 at 8:00 am}
\newcommand{\TestTime}{4.5 hrs}
\newcommand{\Duration}{270  }
\newcommand{\Points}{70 }


\begin{document}
    
    
\vspace*{-1cm}

\begin{center}
{\Large \TestName, \Course }
\end{center}

% \begin{center}    
% {\small
% Instructor: \Instructors \\ Administered on \TestDate. Students should have 3 hours to take this exam. 
% }
% \end{center}


% INSTRUCTIONS FOR STUDENTS
\vspace{2pt}
\begin{center}\textbf{{\large Instructions (PLEASE READ)}}\end{center}
\textbf{Formatting and Timing}
{\small \InstructionsFormatAndTiming}
\textbf{Submission}
{\small \InstructionsSubmission}
\textbf{Questions}
{\small \InstructionsQuestions}
\textbf{Integrity}
{\small \InstructionsHonor}



\vspace{1cm}
\begin{center}
    A TABLE OF LAPLACE TRANSFORMS IS ON THE LAST PAGE
\end{center}


\newpage \InitialsRight

\begin{questions}

% ~~~~~~~~~~~~~~~~~~~~~~~~~~~~~~~~~~~~~~~~~~~~~~~~~~~~~~~~~~~~~~~~~~~~~~

    \question[6] % 5.8.9
    Compute the inverse Laplace transform of $F(s)$ using the convolution theorem. You may leave your answer in terms of an integral in the $t$ domain. 
    
    $$F(s) = \frac{1}{(s+3)^4(s^2+4)}$$  % A
    % $$F(s) = \frac{s}{(s+4)^3(s^2+16)}$$  % Q
    
    \vspace{8cm} 
    
    \question[2] % 8.1.11 and 13
    Use one iteration of the Euler method to estimate the solution to the IVP at the point $t = 0.1$. 
    $$y' = \frac t2 + 5\sqrt y, \quad y(0) = 4$$ % 11, A
    % $$y' = \frac{4 - ty}{5 + y^2}, \quad y(0) = 2$$ % 13, Q
    
    
    
\newpage \InitialsLeft

    \question[6] Consider the linear system of first order differential equations $\displaystyle \mathbf x \, ' = A \mathbf x$, where $A$ has eigenvalues and eigenvectors 
    $$\lambda_1 = 0, \ \mathbf v_1 = \begin{pmatrix} 1 \\ 0 \\ 1 \end{pmatrix}, 
    \quad \lambda _2 = -1 , \  \mathbf v_2 = \begin{pmatrix} 2 \\ 0 \\ 3 \end{pmatrix}, 
    \quad \lambda _3 = -1 , \  \mathbf v_3 = \begin{pmatrix} 1 \\ 2 \\ 0 \end{pmatrix} 
    $$
    % FROM WS 3.3 AND 6.2 
    
    \begin{enumerate}[label=\roman*)]
      \item Identify three solutions to the system, $\mathbf x_1(t)$, $\mathbf x_2(t)$, and $\mathbf x_3(t)$. 
      \vspace{3cm} 
      \item Use a determinant to identify values of $t$, if any, where $\mathbf x_1$, $\mathbf x_2$, and $\mathbf x_3$ form a fundamental set of solutions. You may assume that they are solutions to $\mathbf x \, ' = A\mathbf x$. 
      \vspace{3cm}
      \item Given the initial value $\mathbf x(0) = \begin{pmatrix}0\\2\\0\end{pmatrix}$, give a solution to the IVP.  
      
    \end{enumerate}    
    
    \newpage \InitialsLeft
    \question[6] Solve the following IVP using the Laplace Transform. 
        $$y''-y=-k\delta(t-3), \quad y(0)=2,\quad y'(0)=4, \quad k \in \mathbb R$$  % Worksheet 5.7 # 1
      
    
    
    
\newpage \InitialsRight
    \question[10] 
    Consider the system $\displaystyle \frac{dx}{dt} = \frac32 x - x^2 - \frac{xy}{2} , \quad \frac{dy}{dt} = 2y - \frac 32 xy - \frac{y^2}{2}$. % A HW7.2#1
    
    \begin{parts} 
        \part Identify all critical points of the system.
        \vspace{7cm}
        \part For each critical point, use eigenvalues to classify the critical points according to stability (stable, unstable, asymptotically stable) and type (saddle, proper node, etc). 
    \end{parts}    
\newpage \InitialsLeft \\ \LastPage     
    

% \newpage \InitialsRight

%     \question[8] Solve the following IVP using the Laplace Transform. 
%         $$y'' - 2y'  + 2y =\cos t, \quad y(0)=1,\quad y'(0)=0$$  % HW 5.4.8
  
    
    
    
\newpage \InitialsLeft
    \question[2] State the longest interval, if any, in which the given IVP is certain to have a unique, twice-differentiable solution. Do not attempt to solve the differential equation. 
    
    $$\sqrt{ t-4} \, y''+3ty'+4y=2\ln(7 - t),\quad y(3)=0,\ y'(3)=-1$$ % 4.1.1 WS, A
    % $$\ln (4 - t) \, y''+\frac{t}{t^2-25}y'+y=0, \quad y(1)=4,\ y'(1)=1$$ % 4.1.1 WS Q
    
    \vspace{2cm} 

    % \question[8] 
    % Consider the IVP 
    % $$y' - 2 t = t + 2e^t, \quad y(0) = y_0$$ % 2.2.31 A
    % \begin{parts} 
    %     \part Solve the IVP. \vspace{8cm}
    %     \part Determine the value of $y_0$ that separates solutions that grow positively as $t \to \infty$ to those that grow negatively. 
    % \end{parts}
            

    

    
    
% \newpage \InitialsRight

    \question[6] A 2 kg mass stretches a spring 1 meter. The mass is pushed upward, contracting the spring a distance of 0.2 meters, and then set in motion with an upward velocity of 5 m/s. There is no damping. Set up the IVP and solve it. You may use that the acceleration due to gravity is $g = 10$ m/s$^2$. % \vspace{9cm} \part[1] What is the frequency of the motion? It is not necessary to specify units. \vspace{1cm} \part[3] Sketch the phase portrait for this system. Include the trajectory, $\vec r(t)$, in the phase portrait that corresponds to the initial conditions. Clearly label this trajectory and do not forget to label your axes.  \end{parts} 
    % 4.4# 6a or 7, or midterm 2A #3
    
    
    
\newpage \InitialsLeft

    \question[10] % 
    Suppose $\vec x \, ' = A \vec x$, where $A$ is the $2\times 2$ matrix 
    $$A = \spalignmat{1 -1;1 3}$$
    \begin{parts} 
    
        \part Determine the eigenvalues and eigenvectors of $A$. 
        \vspace{5cm}
        \part Express the general solution of $\vec x \, ' = A \vec x$ in terms of real valued functions. 
        \vspace{8cm} 
        \part Sketch the phase portrait of the system. Do not forget to label your axes.      
        \vspace{1cm} 
    \end{parts}    
    

    

    
\newpage \InitialsRight

    \question[10] Consider the autonomous differential equation 
    $$\dydt = y(k - y), \quad t \ge 0, \quad k  > 0$$  % 1.2,9 and 2.5, A 
    \begin{enumerate}[label=(\roman*)]
        \item list the critical points \vspace{1cm}
        \item sketch the phase line and classify the critical points according to their stability \vspace{4cm}
        \item Determine where $y$ is concave up and concave down \vspace{6cm}
        \item sketch several solution curves in the $ty$-plane. 
    \end{enumerate}

\newpage \InitialsLeft

    \question[8] 
    
    Use variation of parameters to identify the general solution to \[\vec{x} \, ' = \left( \begin{array}{rr} 1 & 5 \\ 5 & 1 \end{array} \right) \vec{x}  + \left( \begin{array}{r}  40\\ 0\end{array} \right)  \]   % 4.7.2 A
    
    % Use variation of parameters to identify the general solution to $$t y'' - (1+t)y' + y = t^2e^{2t}, \quad t > 0$$ Solutions to the corresponding homogeneous equation are $y_1(t) = 1+t$, $y_2(t) = e^t$. % 4.7.23 Q
    


\newpage \InitialsRight

    \question[4] 
    Consider the homogeneous system 
    % $$\vec x\,'=A\vec x=\spalignmat{0,1;-25,\alpha}\vec x,\ \alpha \in \mathbb R$$ % 3.4 14, A
    $$\vec x\,'=A\vec x=\spalignmat{0,1;-4,\alpha;}\vec x,\ \alpha \in \mathbb R$$ % 3.4 14, Q
    \begin{parts}
        \part Determine the eigenvalues of $A$ in terms of $\alpha$. 
        \vspace{5cm} 
        
        \part Identify the values of $\alpha$ (if any) where the qualitative nature of phase portrait for system changes. 
        \vspace{4cm} 
        
        % \part For each value of $\alpha$ that you listed in part (b), sketch the phase portrait for a value of $\alpha$ that is slightly less than the value(s) that you identified. For example, if you identified only one $\alpha$ value, you need to only give one sketch. Do not forget to label your axes. 
    \end{parts}
           
        
    
\end{questions}




\newpage \InitialsLeft \\ \LastPage


\newpage \InitialsRight

\newcounter{NumberInTable}
\newcommand{\LTNUM}{\stepcounter{NumberInTable}{\theNumberInTable.}}

\vspace{-0ex}
    \renewcommand{\arraystretch}{1.8}
    \begin{center}
    \subsection*{Elementary Laplace Transforms}
    \vspace{.5cm}
    
    \begin{tabular}{ p{1cm} p{6cm} p{8cm} }
        & $f(t)$ & $\mathcal  L\{f(t)\}  =F(s)$ \\ \hline
        \LTNUM & $1$ & $\displaystyle \frac{1}{s}$  \\ 
        \LTNUM & $e^{at}$	& $\displaystyle \frac{1}{s-a}$ \\ 
        \LTNUM & $t^n$, $n$ positive integer	& $\displaystyle \frac{n!}{s^{n+1}}$  \\ 
        \LTNUM & $t^p$, $p > -1$	& $\displaystyle \frac{\Gamma(p+1)}{s^{p+1}}$  \\ 
        \LTNUM &$\sin at$ 	& $\displaystyle \dfrac{a}{s^2+a^2}$ \\ 
        \LTNUM &$\cos at$ 	& $\displaystyle \dfrac{s}{s^2+a^2}$ \\ 
        \LTNUM &$\sinh at$	& $\displaystyle \dfrac{a}{s^2-a^2}$ \\ 
        \LTNUM &$\cosh at$	& $\displaystyle \dfrac{s}{s^2-a^2}$ \\ 
        \LTNUM &$e^{at}\sin bt$	& $\displaystyle \dfrac{b}{(s-a)^2+b^2}$   \\ 
        \LTNUM &$e^{at}\cos bt$	& $\displaystyle \dfrac{s-a}{(s-a)^2+b^2}$  \\ 
        \LTNUM &$t^ne^{at}$, $n$ positive integer	& $\displaystyle \dfrac{n!}{(s-a)^{n+1}}$  \\ 
        \LTNUM &$u_c(t)$	& $\displaystyle \dfrac{e^{-cs}}{s}$  \\ 
        \LTNUM &$u_c(t)f(t-c)$& $\displaystyle e^{-cs}F(s)$  \\ 
        \LTNUM &$e^{ct}f(t)$& $\displaystyle F(s-c)$  \\ 
        \LTNUM &$\int_0^t f(t - \tau) g(\tau) d\tau$& $\displaystyle F(s)G(s)$  \\ 
        \LTNUM &$\delta(t-c)$& $e^{-cs}$  \\ 
        \LTNUM &$f^{(n)}(t)$& $\displaystyle s^nF(s)-s^{(n-1)}f(0) - \ldots -f^{(n-1)}(0)$  \\ 
        \LTNUM & $t^n f(t)$& $\displaystyle (-1)^nF^{(n)}(s)$  \\ 
    \end{tabular}
    \renewcommand{\arraystretch}{.5}

\end{center}

\end{document}
    
    

    

    

