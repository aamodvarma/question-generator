\documentclass[12pt]{exam}

\usepackage{graphicx} % allows for graphics
\usepackage{ifthen}  % for if statements 

\newcommand{\sol}{0} %solution =1 or 0

% LOAD PACKAGES
\usepackage{amsmath} % allows for align env and other things
\usepackage{amssymb} % 
\usepackage{mathtools} % allows for single apostrophe
\usepackage{enumitem} % allows for alpha lettering in enumerated lists
\usepackage{lastpage}
\usepackage{array} % for table alignments
\usepackage{graphicx} % if images are needed

\addpoints

\usepackage{pgfplots} % for surfaces (chapter 7)
\usepackage{tikz-3dplot} 
\pgfplotsset{compat=1.9}
\usetikzlibrary{decorations.pathmorphing,patterns} % for some tikz diagrams
% ~~~~~~~~~~~~~~~~~~~~~~~~~~~~~~~~~~~~
% INITIALS
\newcommand{\Initials}{\textit{\Course, \TestName. Your initials: \underline{\hspace{3cm}}} \vspace{1pt}}

\newcommand{\InitialsLeft}{\noindent \hspace{-18pt}\textit{\Course, \TestName. Your initials: \underline{\hspace{3cm}}} \vspace{1pt}}

\newcommand{\InitialsRight}{\begin{flushright}\textit{\Course, \TestName. Your initials: \underline{\hspace{3cm}}} \vspace{1pt}\end{flushright}}

% ~~~~~~~~~~~~~~~~~~~~~~~~~~~~~~~~~~~~
% INSTRUCTIONS FOR DISTANCE LEARNING WITH NO PROCTOR
\newcommand{\InstructionsFormatAndTiming}{

    \begin{itemize} \setlength\itemsep{.1em}
    
        % \item You should only need 75 min to take the exam, but students will have \Duration to submit the exam, from the time that it is released.
        
        \item {\bf Show your work} and justify your answers for all questions unless stated otherwise.
        
        \item Please write neatly, and use dark and clear writing so that the scan is easy to read. 
        
        \item Please write your name or initials at the top of every page 
        
        \item Please solve the questions in the exam in the order they are given. 
        
        \item You do not need to print the exam. As long as you solve problems in the order they are given (just like the written homework sets), you can write your answers on your own paper. But students can print the exam and write their answers on the printed copy if they prefer. 
        
    \end{itemize}

}

\newcommand{\InstructionsSubmission}{

    \begin{itemize} \setlength\itemsep{.1em}
        \item Students should scan their work and submit it through Gradescope. There should be an \textbf{assignment} in Gradescope for this exam. The process for submitting your work will be similar to what you have used for homework. 
        
        \item Work must be submitted by \DueDate. 
        
        \item Please upload your work as a single PDF file. If this is not possible you can email your work to your instructor. 
        
        \item During the upload process in Gradescope, please indicate which page of your work corresponds to each question in the exam. 
    \end{itemize}
}

\newcommand{\InstructionsQuestions}{

    \begin{itemize} \setlength\itemsep{.1em}
        
        \item If there are questions during the exam, students can email their instructor or message them through Canvas. 
        
        \item Our course Piazza forum will be temporarily inactive during the exam. 
        
        \item If you run into any technical issues or any unanticipated emergencies, please email your instructor as soon as you can. 
    
        
    \end{itemize}

}


\newcommand{\InstructionsHonor}{

    \begin{itemize} \setlength\itemsep{.1em}    
        \item Students can use any resources while taking these tests including online calculators and Mathematica
        \item Students cannot communicate with anyone during these tests.
        \item Students cannot use solutions provided from another student or third party. 
        \item In other words: do your own work but you can use technology to solve problems. 
 
    \end{itemize}

}






\newcommand{\GTHonorCode}{Having read the Georgia Institute of Technology Academic Honor Code, I understand and accept my responsibility as a member of the Georgia Tech community to uphold the Honor Code at all times. }



% FANCY HEADERS - MAKE EMPTY
\pagestyle{headandfoot}
\runningfooter{}{}{}


% ADJUST MARGINS FOR DISTANCE LEARNING REQUIREMENTS
\usepackage[tmargin=1.0in,bmargin=1.0in,left=1in,right=1in]{geometry}


% TIKZ DIAGRAMS
\usepackage{color}
\usepackage{tikz}  \usetikzlibrary{arrows} 
\usetikzlibrary{calc} 


% ADJUST FIRST LINE IN PARAGRAPH INDENTATION 
\setlength\parindent{0pt}


% COURSE SPECIFIC INFORMATION
\newcommand{\Course}{Math 2552}
\newcommand{\Instructors}{}

% WHO TO CONTACT DURING EXAM IF QUESTIONS
\newcommand{\InstructorContact}{}

\usepackage{spalign} % Joe Rabinoff's matrix package

\newcommand{\LastPage}{\begin{center}\textit{This page may be used for scratch work. Please indicate clearly if you would like your work on this page to be graded. }\end{center}   }


% DERIVATIVES
\newcommand{\dydt}{{\frac{dy}{dt}}} % 
\newcommand{\dydx}{{\frac{dy}{dx}}} % 
\newcommand{\dydtt}{{\frac{d ^2y}{dt^2}}} % 
\newcommand{\dydxx}{{\frac{d^2y}{dx^2}}} % 
\newcommand{\dydttt}{{\frac{d^3y}{dt^3}}} % 

\newcommand{\ddt}{{\frac{d}{dt}}} % 
\newcommand{\ddx}{{\frac{d}{dx}}} % 
\newcommand{\dudt}{{\frac{du}{dt}}} % 
\newcommand{\dvdx}{{\frac{dv}{dx}}} % 
\newcommand{\dxdt}{{\frac{dx}{dt}}} % 
\newcommand{\dxdtt}{{\frac{d^2x}{dt^2}}} % 
\newcommand{\dzdt}{{\frac{dz}{dt}}} % 



% COLORS FOR DIAGRAMS
\definecolor{DarkBlue}{rgb}{0.0,0.0,0.6} % 
% \definecolor{DarkGreen}{rgb}{0.0,0.3,0.0} % 
% \definecolor{DarkRed}{rgb}{0.6,0.0,0.0} % 

% TEST SPECIFIC INFORMATION
\newcommand{\TestName}{Midterm 2B}
\newcommand{\TestTime}{}
\newcommand{\Duration}{3 hours }
\newcommand{\Points}{}
\newcommand{\DueDate}{9:00 PM ET}


% \usepackage{tikz}
% \usetikzlibrary{shapes,snakes}   
% \usetikzlibrary{arrows,automata}

\begin{document}
    

\vspace*{-1cm}

\begin{center}
{\Large \TestName, \Course }
\end{center}

% \begin{center}    
% {\small
% Instructor: \Instructors \\ Administered on \TestDate. Students should have 3 hours to take this exam. 
% }
% \end{center}


% INSTRUCTIONS FOR STUDENTS
\vspace{2pt}
\begin{center}\textbf{{\large Instructions (PLEASE READ)}}\end{center}
\textbf{Formatting and Timing}
{\small \InstructionsFormatAndTiming}
\textbf{Submission}
{\small \InstructionsSubmission}
\textbf{Questions}
{\small \InstructionsQuestions}
\textbf{Integrity}
{\small \InstructionsHonor}

 % cover page for unproctored exam

\newpage


\InitialsRight

\begin{questions}
    

    \question[5] Determine all values of $\alpha$, if any, for which all solutions tend to zero as $t\to\infty$. $$y'' - (2\alpha - 1) y' + (\alpha^2-\alpha+1) y = 0, \quad \alpha \in \mathbb R$$ \vspace{10cm}  % 4.3.47
    
    \question[4]  Determine a suitable form for the particular solution, $Y(t)$, to the differential equations if the method of undetermined coefficients is to be used. Do not determine the values of the coefficients or solve the differential equation.
        % $$y'' + 4y = t^2 + \sin(2t)$$
        $$y'' +2y' - 3y = 3te^t\sin t$$
    
    \newpage    \InitialsLeft
    
        \question[10] Consider the system of equations. 
        $$
        x_1 ' = 7x_1 , \qquad x_2 ' = 6x_2 + x_3 , \qquad x_3 ' = x_2 + 6x_3 
        $$
        \begin{enumerate} 
            \item[a)] Express the system of equations in the form $\vec x \, ' = A \vec x$, where $A$ is a $3\times3$ matrix. \vspace{3cm} 

            \item[b)] Determine the eigenvectors of $A$ and use them to write down the general solution. You may use that the only eigenvalues of $A$ are $7$ and $5$. \textit{Show your work when computing the eigenvectors: it ok to check your work with software, but you should show that you can compute eigenvectors by hand. }
                
        \end{enumerate}
        

    \newpage \InitialsLeft
    
    \question[11] A 4 kg mass stretches a spring 0.1 meters. The mass is then pushed upward, contracting the spring a distance of 0.1 meters from its equilibrium position, and then set in motion with an initial downward velocity of 2 m/s. There is no damping. 
    \begin{parts} \part[4] Construct an IVP for this situation. \vspace{3cm} \part[4] Solve your IVP. You may use that the acceleration due to gravity is $g = 10$ m/s$^2$. \textit{Note: the solution to the DE can be found by inspection, it is ok to just write the answer for solving the DE, but if you want to show work you can.} \vspace{9cm} \part[3] Sketch the trajectory of the object in the phase plane that corresponds to the IVP. \textit{Do not forget to label your axes and indicate the directions of motion.}    \end{parts} 

    \newpage \InitialsLeft

    \question[8] Use variation of parameters to identify the general solution to $$y'' - \frac{t+2}{t}y' + \frac{t+2}{t^2}y = 5t, \quad t > 0$$ Solutions to the corresponding homogeneous equation are $y_1(t) = t$, $y_2(t) = te^t$. 
    
    
    \newpage \InitialsLeft
    
    \question[3] Determine whether $\vec x_1$, $\vec x_2$, and $\vec x_3$ form a fundamental set of solutions for $\vec x \, ' = A\vec x$. The only eigenvalues of $A$ are $-1$ and $1$.
    
    $$A = \spalignmat{0 0 1;0 1 0;1 0 0}, \quad \vec x_1 = e^{-t}\spalignmat{-2;0;1}, \quad \vec x_2 = e^{t}\spalignmat{1;0;1}, \quad \vec x_3 = e^{t}\spalignmat{1;1;2} $$
    
    \vspace{5cm}
    
    \newpage \InitialsLeft
    
    \question[7] Use the method of reduction of order to find a second solution $y_2$ of the given differential equation such that $\{ y_1 , y_2 \}$ is a fundamental set of solutions on the given interval.
    
    $$t^2y'' - 3ty' + 3y = 0, \quad t > 0, \quad y_1(t) = t$$
    
    \newpage \InitialsLeft
    

    \question[2] A small number of points will be allocated for presentation, neatness, and organization. Please ensure that
    \begin{enumerate}
        % \item your scan is under 5 MB in file size
        \item your work is legible in the scan
        \item your name or initials are at the top of every page
        \item questions are answered in the order in which they were given
        \item during the upload process you have indicated which pages correspond to which question, and made sure that none of your pages are upside down or sideways (you can also change the orientation of the pages when you upload in Gradescope)
    \end{enumerate}
    Ensuring that these criteria are met helps ensure that your exam is graded efficiently and accurately. 
    


    
\end{questions}
    
    Please sign and date the following GT Honor Code statement. \\ 
    \vspace{2pt}
    
    \textbf{Georgia Tech Honor Code}:\ \GTHonorCode
    
    \begin{center}
    \begin{center}
        \def\arraystretch{0.35}%  1 is the default, change whatever you need
        \begin{tabular}{ b{8cm} b{8cm} }
        \vspace{.5cm} \underline{\hspace{7cm}} & \vspace{.5cm} \underline{\hspace{4.5cm}}  \tabularnewline
        \vspace{6pt} signature & \vspace{6pt} date    
        \end{tabular}
    \end{center}
    \end{center}    
\end{document}

