\documentclass[12pt]{exam}

\usepackage{graphicx} % allows for graphics
\usepackage{ifthen}  % for if statements 

\newcommand{\sol}{0} %solution =1 or 0

% LOAD PACKAGES
\usepackage{amsmath} % allows for align env and other things
\usepackage{amssymb} % 
\usepackage{mathtools} % allows for single apostrophe
\usepackage{enumitem} % allows for alpha lettering in enumerated lists
\usepackage{lastpage}
\usepackage{array} % for table alignments
\usepackage{graphicx} % if images are needed

\addpoints

\usepackage{pgfplots} % for surfaces (chapter 7)
\usepackage{tikz-3dplot} 
\pgfplotsset{compat=1.9}
\usetikzlibrary{decorations.pathmorphing,patterns} % for some tikz diagrams
% ~~~~~~~~~~~~~~~~~~~~~~~~~~~~~~~~~~~~
% INITIALS
\newcommand{\Initials}{\textit{\Course, \TestName. Your initials: \underline{\hspace{3cm}}} \vspace{1pt}}

\newcommand{\InitialsLeft}{\noindent \hspace{-18pt}\textit{\Course, \TestName. Your initials: \underline{\hspace{3cm}}} \vspace{1pt}}

\newcommand{\InitialsRight}{\begin{flushright}\textit{\Course, \TestName. Your initials: \underline{\hspace{3cm}}} \vspace{1pt}\end{flushright}}

% ~~~~~~~~~~~~~~~~~~~~~~~~~~~~~~~~~~~~
% INSTRUCTIONS FOR DISTANCE LEARNING WITH NO PROCTOR
\newcommand{\InstructionsFormatAndTiming}{

    \begin{itemize} \setlength\itemsep{.1em}
    
        % \item You should only need 75 min to take the exam, but students will have \Duration to submit the exam, from the time that it is released.
        
        \item {\bf Show your work} and justify your answers for all questions unless stated otherwise.
        
        \item Please write neatly, and use dark and clear writing so that the scan is easy to read. 
        
        \item Please solve the questions in the exam in the order they are given. 
        
        \item You do not need to print the exam. As long as you solve problems in the order they are given you can write your answers on your own paper or by using a tablet. But students can print the exam and write their answers on the printed copy if they prefer. 

        \item Do not type your answers to any part of the exam. 
        
    \end{itemize}

}

\newcommand{\InstructionsSubmission}{

    \begin{itemize} \setlength\itemsep{.1em}
        \item Students should scan their work and submit it through Gradescope. There should be an \textbf{assignment} in Gradescope for this exam. 
        
        \item Work must be submitted by \DueDate. 
        
        \item Please upload your work as a single file. 
        
        \item During the upload process in Gradescope, please indicate which page of your work corresponds to each question. A small number of points will be allocated for this.
    \end{itemize}
}

\newcommand{\InstructionsQuestions}{

    \begin{itemize} \setlength\itemsep{.1em}
        
        \item If there are questions during the exam, students can email their instructor or message them through Canvas. 
        
        \item Our course Piazza forum will be temporarily inactive during the exam. 
        
        \item If you run into any technical issues or any unanticipated emergencies, please email your instructor as soon as you can. 
    
        
    \end{itemize}

}


\newcommand{\InstructionsHonor}{

    \begin{itemize} \setlength\itemsep{.1em}    
        \item Students can use any resources while taking these tests including online calculators and Mathematica
        \item Students cannot communicate with anyone during these tests.
        \item Students cannot use solutions provided from another student or third party. 
        \item In other words: do your own work but you can use technology to solve problems. 
 
    \end{itemize}

}






\newcommand{\GTHonorCode}{Having read the Georgia Institute of Technology Academic Honor Code, I understand and accept my responsibility as a member of the Georgia Tech community to uphold the Honor Code at all times. }



% FANCY HEADERS - MAKE EMPTY
\pagestyle{headandfoot}
\runningfooter{}{}{}


% ADJUST MARGINS FOR DISTANCE LEARNING REQUIREMENTS
\usepackage[tmargin=1.0in,bmargin=1.0in,left=1in,right=1in]{geometry}


% TIKZ DIAGRAMS
\usepackage{color}
\usepackage{tikz}  \usetikzlibrary{arrows} 
\usetikzlibrary{calc} 


% ADJUST FIRST LINE IN PARAGRAPH INDENTATION 
\setlength\parindent{0pt}


% COURSE SPECIFIC INFORMATION
\newcommand{\Course}{Math 2552}
\newcommand{\Instructors}{}

% WHO TO CONTACT DURING EXAM IF QUESTIONS
\newcommand{\InstructorContact}{}

\usepackage{spalign} % Joe Rabinoff's matrix package

\newcommand{\LastPage}{\begin{center}\textit{This page may be used for scratch work. Please indicate clearly if you would like your work on this page to be graded. }\end{center}   }


% DERIVATIVES
\newcommand{\dydt}{{\frac{dy}{dt}}} % 
\newcommand{\dydx}{{\frac{dy}{dx}}} % 
\newcommand{\dydtt}{{\frac{d ^2y}{dt^2}}} % 
\newcommand{\dydxx}{{\frac{d^2y}{dx^2}}} % 
\newcommand{\dydttt}{{\frac{d^3y}{dt^3}}} % 

\newcommand{\ddt}{{\frac{d}{dt}}} % 
\newcommand{\ddx}{{\frac{d}{dx}}} % 
\newcommand{\dudt}{{\frac{du}{dt}}} % 
\newcommand{\dvdx}{{\frac{dv}{dx}}} % 
\newcommand{\dxdt}{{\frac{dx}{dt}}} % 
\newcommand{\dxdtt}{{\frac{d^2x}{dt^2}}} % 
\newcommand{\dzdt}{{\frac{dz}{dt}}} % 



% COLORS FOR DIAGRAMS
\definecolor{Darkblack}{rgb}{0.0,0.0,0.6} % 
% \definecolor{DarkGreen}{rgb}{0.0,0.3,0.0} % 
% \definecolor{DarkRed}{rgb}{0.6,0.0,0.0} % 

% TEST SPECIFIC INFORMATION
\newcommand{\TestName}{Midterm 3M}
\newcommand{\TestTime}{}
\newcommand{\Duration}{3 hours }
\newcommand{\Points}{}
\newcommand{\DueDate}{12:30 PM ET}


% \usepackage{tikz}
% \usetikzlibrary{shapes,snakes}   
% \usetikzlibrary{arrows,automata}

\begin{document}
    

\vspace*{-1cm}

\begin{center}
{\Large \TestName, \Course }
\end{center}

% \begin{center}    
% {\small
% Instructor: \Instructors \\ Administered on \TestDate. Students should have 3 hours to take this exam. 
% }
% \end{center}


% INSTRUCTIONS FOR STUDENTS
\vspace{2pt}
\begin{center}\textbf{{\large Instructions (PLEASE READ)}}\end{center}
\textbf{Formatting and Timing}
{\small \InstructionsFormatAndTiming}
\textbf{Submission}
{\small \InstructionsSubmission}
\textbf{Questions}
{\small \InstructionsQuestions}
\textbf{Integrity}
{\small \InstructionsHonor}

 % cover page for unproctored exam

\newpage


\newpage \InitialsLeft 

\begin{questions}
    \question[3] % WW HW14 FOR AQM
        Consider the equations describing the interactions of robins $r$ and worms $w$.
        $$\frac{dw}{dt} = 2w - 3wr, \quad \text{and} \quad \frac{dr}{dt} = - 3r+rw$$
        \begin{parts}
            \part What are the non-zero nullclines for this system? \vspace{3cm} 
            \part 
            % When $r=w=1000$, is the population of worms increasing or decreasing? % A
            % When $r=w=1000$, is the population of robins increasing or decreasing? % Q
            When $r=1000$ and $w=2000$, is the population of worms increasing or decreasing? % M
            \vspace{2cm} 
        \end{parts}
        
    \question[5] Compute the inverse Laplace Transform of  $\displaystyle Y = \frac{2s+1}{s^2-2s+2}$. % M 5.3#15
    
    % $\displaystyle Y = \frac{5s+25}{s^2+10s+74}$. % A 5.3#13
    
% $\displaystyle Y = \frac{2s+1}{s^2-2s+2}$. % Q 5.3#15
    
    % 
        
\newpage \InitialsLeft 
    
    \question[6] % WW HW13 FOR AQM 
    Consider the following integral equation, where the unknown dependent variable, \(y(t)\), appears within an integral:
    \[\int_0^t \sin\Large(4 (t-w)\Large)\, y(w) \ dw = 9 t^2.\] This equation is defined for \(t \geq 0\).  
    \begin{parts}
    \part Use convolution and the Laplace transforms to determine an explicit expression for the Laplace transform of $y(t)$.
    \vspace{5cm} 
    \part Obtain an expression for \(y(t)\).
    \end{parts}
    
    

\newpage \InitialsRight
    \question[11] 
    Consider the system $\displaystyle dx/dt = x-y, \quad dy/dt = x-3y+xy - 3$. % M HW7.2#3

    
    % $\displaystyle dx/dt = x+y^2, \quad dy/dt = x + 2y$. % A HW7.2#3
    
% $\displaystyle dx/dt = x(1-x-y), \quad dy/dt = y(1.5-y-x)$. % Q HW7.3#5    
    
    \begin{parts} 
        \part Identify all critical points of the system.
        \vspace{7cm}
        \part For each critical point, use eigenvalues to classify the critical points according to stability (stable, unstable, asymptotically stable) and type (saddle, proper node, etc). 
    \end{parts}

\newpage \InitialsLeft \\ \Scratch 

\newpage \InitialsRight
    
    \question[5] Calculate the Laplace transform of $g(t)$. 
    
    % $$g(t) = (t-4)u_3(t)  - (t - 2)u_4(t) $$ %A HW 5.5#11
    % $$g(t) = (t-5)u_3(t)  - (t - 2)u_4(t) $$ %Q HW 5.5#11
    $$g(t) = (t-6)u_3(t)  - (t - 2)u_4(t) $$ %M HW 5.5#11

\newpage \InitialsLeft

    \question[10] Use the Laplace transform to solve the following IVP.

    % $$\displaystyle y''-4y'+4y=0,\qquad y(0)=1,\quad y'(0)=1$$ %Q,A HW 5.4#4
    $$\displaystyle y''-2y'+4y=0,\qquad y(0)=2,\quad y'(0)=0$$ %M HW 5.4#5
    

\newpage \InitialsRight

    \question[10] Use the Laplace transform to solve the following IVP.

    % $$\displaystyle y''-y=10\delta(t-4)\qquad y(0)=3,\quad y'(0)=3 $$ %A WORKSHEET 5.7#1
    % $$\displaystyle y''+4y=10\delta(t-4\pi)\qquad y(0)=\frac12,\quad y'(0)=0 $$ %Q 5.7#6
    $$\displaystyle y''+2y'+2y=10\delta(t-\pi)\qquad y(0)=0,\quad y'(0)=1 $$ %M 5.7#1



    
\end{questions}

% \newpage \InitialsLeft \\ \LastPage


% \newpage \InitialsRight

% \newcounter{NumberInTable}
% \newcommand{\LTNUM}{\stepcounter{NumberInTable}{\theNumberInTable.}}

% \vspace{-0ex}
%     \renewcommand{\arraystretch}{1.8}
%     \begin{center}
%     \subsection*{Elementary Laplace Transforms}
%     \vspace{.5cm}
    
%     \begin{tabular}{ p{1cm} p{6cm} p{8cm} }
%         & $f(t)$ & $\mathcal  L\{f(t)\}  =F(s)$ \\ \hline
%         \LTNUM & $1$ & $\displaystyle \frac{1}{s}$  \\ 
%         \LTNUM & $e^{at}$	& $\displaystyle \frac{1}{s-a}$ \\ 
%         \LTNUM & $t^n$, $n$ positive integer	& $\displaystyle \frac{n!}{s^{n+1}}$  \\ 
%         \LTNUM & $t^p$, $p > -1$	& $\displaystyle \frac{\Gamma(p+1)}{s^{p+1}}$  \\ 
%         \LTNUM &$\sin at$ 	& $\displaystyle \dfrac{a}{s^2+a^2}$ \\ 
%         \LTNUM &$\cos at$ 	& $\displaystyle \dfrac{s}{s^2+a^2}$ \\ 
%         \LTNUM &$\sinh at$	& $\displaystyle \dfrac{a}{s^2-a^2}$ \\ 
%         \LTNUM &$\cosh at$	& $\displaystyle \dfrac{s}{s^2-a^2}$ \\ 
%         \LTNUM &$e^{at}\sin bt$	& $\displaystyle \dfrac{b}{(s-a)^2+b^2}$   \\ 
%         \LTNUM &$e^{at}\cos bt$	& $\displaystyle \dfrac{s-a}{(s-a)^2+b^2}$  \\ 
%         \LTNUM &$t^ne^{at}$, $n$ positive integer	& $\displaystyle \dfrac{n!}{(s-a)^{n+1}}$  \\ 
%         \LTNUM &$u_c(t)$	& $\displaystyle \dfrac{e^{-cs}}{s}$  \\ 
%         \LTNUM &$u_c(t)f(t-c)$& $\displaystyle e^{-cs}F(s)$  \\ 
%         \LTNUM &$e^{ct}f(t)$& $\displaystyle F(s-c)$  \\ 
%         \LTNUM &$\int_0^t f(t - \tau) g(\tau) d\tau$& $\displaystyle F(s)G(s)$  \\ 
%         \LTNUM &$\delta(t-c)$& $e^{-cs}$  \\ 
%         \LTNUM &$f^{(n)}(t)$& $\displaystyle s^nF(s)-s^{(n-1)}f(0) - \ldots -f^{(n-1)}(0)$  \\ 
%         \LTNUM & $t^n f(t)$& $\displaystyle (-1)^nF^{(n)}(s)$  \\ 
%     \end{tabular}
%     \renewcommand{\arraystretch}{.5}

% \end{center}


\end{document}

