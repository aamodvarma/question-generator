\documentclass[12pt]{exam}

% LOAD PACKAGES
\usepackage{amsmath} % allows for align env and other things
\usepackage{amssymb} % 
\usepackage{mathtools} % allows for single apostrophe
\usepackage{enumitem} % allows for alpha lettering in enumerated lists
\usepackage{lastpage}
\usepackage{array} % for table alignments
\usepackage{graphicx} % if images are needed

\addpoints

\usepackage{pgfplots} % for surfaces (chapter 7)
\usepackage{tikz-3dplot} 
\pgfplotsset{compat=1.9}
\usetikzlibrary{decorations.pathmorphing,patterns} % for some tikz diagrams
% ~~~~~~~~~~~~~~~~~~~~~~~~~~~~~~~~~~~~
% INITIALS
\newcommand{\Initials}{\textit{\Course, \TestName. Your initials: \underline{\hspace{3cm}}} \vspace{1pt}}

\newcommand{\InitialsLeft}{\noindent \hspace{-18pt}\textit{\Course, \TestName. Your initials: \underline{\hspace{3cm}}} \vspace{1pt}}

\newcommand{\InitialsRight}{\begin{flushright}\textit{\Course, \TestName. Your initials: \underline{\hspace{3cm}}} \vspace{1pt}\end{flushright}}

% ~~~~~~~~~~~~~~~~~~~~~~~~~~~~~~~~~~~~
% INSTRUCTIONS FOR DISTANCE LEARNING WITH NO PROCTOR
\newcommand{\InstructionsFormatAndTiming}{

    \begin{itemize} \setlength\itemsep{.1em}
    
        % \item You should only need 75 min to take the exam, but students will have \Duration to submit the exam, from the time that it is released.
        
        \item {\bf Show your work} and justify your answers for all questions unless stated otherwise.
        
        \item Please write neatly, and use dark and clear writing so that the scan is easy to read. 
        
        \item Please solve the questions in the exam in the order they are given. 
        
        \item You do not need to print the exam. As long as you solve problems in the order they are given you can write your answers on your own paper or by using a tablet. But students can print the exam and write their answers on the printed copy if they prefer. 

        \item Do not type your answers to any part of the exam. 
        
    \end{itemize}

}

\newcommand{\InstructionsSubmission}{

    \begin{itemize} \setlength\itemsep{.1em}
        \item Students should scan their work and submit it through Gradescope. There should be an \textbf{assignment} in Gradescope for this exam. 
        
        \item Work must be submitted by \DueDate. 
        
        \item Please upload your work as a single file. 
        
        \item During the upload process in Gradescope, please indicate which page of your work corresponds to each question. A small number of points will be allocated for this.
    \end{itemize}
}

\newcommand{\InstructionsQuestions}{

    \begin{itemize} \setlength\itemsep{.1em}
        
        \item If there are questions during the exam, students can email their instructor or message them through Canvas. 
        
        \item Our course Piazza forum will be temporarily inactive during the exam. 
        
        \item If you run into any technical issues or any unanticipated emergencies, please email your instructor as soon as you can. 
    
        
    \end{itemize}

}


\newcommand{\InstructionsHonor}{

    \begin{itemize} \setlength\itemsep{.1em}    
        \item Students can use any resources while taking these tests including online calculators and Mathematica
        \item Students cannot communicate with anyone during these tests.
        \item Students cannot use solutions provided from another student or third party. 
        \item In other words: do your own work but you can use technology to solve problems. 
 
    \end{itemize}

}






\newcommand{\GTHonorCode}{Having read the Georgia Institute of Technology Academic Honor Code, I understand and accept my responsibility as a member of the Georgia Tech community to uphold the Honor Code at all times. }



% FANCY HEADERS - MAKE EMPTY
\pagestyle{headandfoot}
\runningfooter{}{}{}


% ADJUST MARGINS FOR DISTANCE LEARNING REQUIREMENTS
\usepackage[tmargin=1.0in,bmargin=1.0in,left=1in,right=1in]{geometry}


% TIKZ DIAGRAMS
\usepackage{color}
\usepackage{tikz}  \usetikzlibrary{arrows} 
\usetikzlibrary{calc} 


% ADJUST FIRST LINE IN PARAGRAPH INDENTATION 
\setlength\parindent{0pt}


% COURSE SPECIFIC INFORMATION
\newcommand{\Course}{Math 2552}
\newcommand{\Instructors}{}

% WHO TO CONTACT DURING EXAM IF QUESTIONS
\newcommand{\InstructorContact}{}

\usepackage{spalign} % Joe Rabinoff's matrix package

\newcommand{\LastPage}{\begin{center}\textit{This page may be used for scratch work. Please indicate clearly if you would like your work on this page to be graded. }\end{center}   }


% DERIVATIVES
\newcommand{\dydt}{{\frac{dy}{dt}}} % 
\newcommand{\dydx}{{\frac{dy}{dx}}} % 
\newcommand{\dydtt}{{\frac{d ^2y}{dt^2}}} % 
\newcommand{\dydxx}{{\frac{d^2y}{dx^2}}} % 
\newcommand{\dydttt}{{\frac{d^3y}{dt^3}}} % 

\newcommand{\ddt}{{\frac{d}{dt}}} % 
\newcommand{\ddx}{{\frac{d}{dx}}} % 
\newcommand{\dudt}{{\frac{du}{dt}}} % 
\newcommand{\dvdx}{{\frac{dv}{dx}}} % 
\newcommand{\dxdt}{{\frac{dx}{dt}}} % 
\newcommand{\dxdtt}{{\frac{d^2x}{dt^2}}} % 
\newcommand{\dzdt}{{\frac{dz}{dt}}} % 



% TEST SPECIFIC INFORMATION
\newcommand{\TestName}{Quiz 3}
\newcommand{\TestDate}{}
\newcommand{\DueDate}{the due date indicated in the syllabus}
\newcommand{\TestTime}{8:15 pm to 9:15 pm}

\begin{document}
    
\input{2021Sum/coverpage}

\newpage

\begin{questions}

\question[5] Use the method of reduction of order to find a second solution, $y_2$, to the DE so that $\{y_1,y_2\}$ forms a fundamental set. $$ t^2y'' +8t y' - 18y = 0, \quad t>0, \quad y_1(t) = t^2$$. 

\newpage 
\question[5] Consider the DE $$y''-4y'+5y = 0$$
\begin{parts}
    \part Solve the DE. \vspace{10cm}
    \part Classify the critical point of the corresponding dynamical system according to type (node, spiral, saddle, etc) and stability. \vspace{2cm}
    \part Sketch the phase portrait of the system. Include in your diagram straight-line orbits (if any). Please label your axes. Indicate the direction of motion on your solution curves. 
    
\end{parts}
    
\newpage

\question[4] If the Wronskian, $W$, of $f(t)$ and $g(t)$ is $e^{\omega t}$, and if $f(t) = e^{kt}$, determine $g(t)$. You can assume $k>0$ and $\omega>0$. And you can leave your answer in terms of $\omega$ and $k$. Show your work. 

\newpage 

\question[3] A spring is stretched 40 cm by a force of 20 newtons (N). A mass of 2 kg is then hung from the same spring and is also attached to a viscous damper that exerts a force of 3 N when the velocity of the mass is 12 m/s. The mass is pulled down 2 cm below its equilibrium position and given an initial downward velocity of 40 cm/s. Write down the IVP based on this physical description. You do not need to solve your IVP. 

\vspace{8cm}

\question[2] Give an example of a constant coefficient differential equation whose general solution is $y = c_1 e^{-5t} + c_2 e^{-4t}$. You do not need to show your work for this question, you can just write down your differential equation. 





\newpage



\question[1] A small number of points will be allocated for presentation, neatness, and organization. Please ensure that
    \begin{parts}
        % \item your scan is under 5 MB in file size
        \part your work is legible in the scan
        \part questions are answered in the order in which they were given
        \part during the upload process you have indicated which pages correspond to which question, and made sure that none of your pages are upside down or sideways (you can also change the orientation of the pages when you upload in Gradescope)
    \end{parts}
    Ensuring that these criteria are met helps ensure that your work is graded efficiently and accurately. 

\end{questions}
\vspace{2cm}
    
    \vspace{6pt}
    \textbf{Georgia Tech Honor Code}\\
    \GTHonorCode
    
    \begin{center}
    \begin{center}
        \def\arraystretch{0.35}%  1 is the default, change whatever you need
        \begin{tabular}{ b{8cm} b{8cm} }
        \vspace{.5cm} \underline{\hspace{7cm}} & \vspace{.5cm} \underline{\hspace{4.5cm}}  \tabularnewline
        \vspace{6pt} signature & \vspace{6pt} date    
        \end{tabular}
    \end{center}
    \end{center}    

\end{document}