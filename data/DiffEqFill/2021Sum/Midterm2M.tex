\documentclass[12pt]{exam}

\usepackage{graphicx} % allows for graphics
\usepackage{ifthen}  % for if statements 

\newcommand{\sol}{0} %solution =1 or 0

% LOAD PACKAGES
\usepackage{amsmath} % allows for align env and other things
\usepackage{amssymb} % 
\usepackage{mathtools} % allows for single apostrophe
\usepackage{enumitem} % allows for alpha lettering in enumerated lists
\usepackage{lastpage}
\usepackage{array} % for table alignments
\usepackage{graphicx} % if images are needed

\addpoints

\usepackage{pgfplots} % for surfaces (chapter 7)
\usepackage{tikz-3dplot} 
\pgfplotsset{compat=1.9}
\usetikzlibrary{decorations.pathmorphing,patterns} % for some tikz diagrams
% ~~~~~~~~~~~~~~~~~~~~~~~~~~~~~~~~~~~~
% INITIALS
\newcommand{\Initials}{\textit{\Course, \TestName. Your initials: \underline{\hspace{3cm}}} \vspace{1pt}}

\newcommand{\InitialsLeft}{\noindent \hspace{-18pt}\textit{\Course, \TestName. Your initials: \underline{\hspace{3cm}}} \vspace{1pt}}

\newcommand{\InitialsRight}{\begin{flushright}\textit{\Course, \TestName. Your initials: \underline{\hspace{3cm}}} \vspace{1pt}\end{flushright}}

% ~~~~~~~~~~~~~~~~~~~~~~~~~~~~~~~~~~~~
% INSTRUCTIONS FOR DISTANCE LEARNING WITH NO PROCTOR
\newcommand{\InstructionsFormatAndTiming}{

    \begin{itemize} \setlength\itemsep{.1em}
    
        % \item You should only need 75 min to take the exam, but students will have \Duration to submit the exam, from the time that it is released.
        
        \item {\bf Show your work} and justify your answers for all questions unless stated otherwise.
        
        \item Please write neatly, and use dark and clear writing so that the scan is easy to read. 
        
        \item Please solve the questions in the exam in the order they are given. 
        
        \item You do not need to print the exam. As long as you solve problems in the order they are given you can write your answers on your own paper or by using a tablet. But students can print the exam and write their answers on the printed copy if they prefer. 

        \item Do not type your answers to any part of the exam. 
        
    \end{itemize}

}

\newcommand{\InstructionsSubmission}{

    \begin{itemize} \setlength\itemsep{.1em}
        \item Students should scan their work and submit it through Gradescope. There should be an \textbf{assignment} in Gradescope for this exam. 
        
        \item Work must be submitted by \DueDate. 
        
        \item Please upload your work as a single file. 
        
        \item During the upload process in Gradescope, please indicate which page of your work corresponds to each question. A small number of points will be allocated for this.
    \end{itemize}
}

\newcommand{\InstructionsQuestions}{

    \begin{itemize} \setlength\itemsep{.1em}
        
        \item If there are questions during the exam, students can email their instructor or message them through Canvas. 
        
        \item Our course Piazza forum will be temporarily inactive during the exam. 
        
        \item If you run into any technical issues or any unanticipated emergencies, please email your instructor as soon as you can. 
    
        
    \end{itemize}

}


\newcommand{\InstructionsHonor}{

    \begin{itemize} \setlength\itemsep{.1em}    
        \item Students can use any resources while taking these tests including online calculators and Mathematica
        \item Students cannot communicate with anyone during these tests.
        \item Students cannot use solutions provided from another student or third party. 
        \item In other words: do your own work but you can use technology to solve problems. 
 
    \end{itemize}

}






\newcommand{\GTHonorCode}{Having read the Georgia Institute of Technology Academic Honor Code, I understand and accept my responsibility as a member of the Georgia Tech community to uphold the Honor Code at all times. }



% FANCY HEADERS - MAKE EMPTY
\pagestyle{headandfoot}
\runningfooter{}{}{}


% ADJUST MARGINS FOR DISTANCE LEARNING REQUIREMENTS
\usepackage[tmargin=1.0in,bmargin=1.0in,left=1in,right=1in]{geometry}


% TIKZ DIAGRAMS
\usepackage{color}
\usepackage{tikz}  \usetikzlibrary{arrows} 
\usetikzlibrary{calc} 


% ADJUST FIRST LINE IN PARAGRAPH INDENTATION 
\setlength\parindent{0pt}


% COURSE SPECIFIC INFORMATION
\newcommand{\Course}{Math 2552}
\newcommand{\Instructors}{}

% WHO TO CONTACT DURING EXAM IF QUESTIONS
\newcommand{\InstructorContact}{}

\usepackage{spalign} % Joe Rabinoff's matrix package

\newcommand{\LastPage}{\begin{center}\textit{This page may be used for scratch work. Please indicate clearly if you would like your work on this page to be graded. }\end{center}   }


% DERIVATIVES
\newcommand{\dydt}{{\frac{dy}{dt}}} % 
\newcommand{\dydx}{{\frac{dy}{dx}}} % 
\newcommand{\dydtt}{{\frac{d ^2y}{dt^2}}} % 
\newcommand{\dydxx}{{\frac{d^2y}{dx^2}}} % 
\newcommand{\dydttt}{{\frac{d^3y}{dt^3}}} % 

\newcommand{\ddt}{{\frac{d}{dt}}} % 
\newcommand{\ddx}{{\frac{d}{dx}}} % 
\newcommand{\dudt}{{\frac{du}{dt}}} % 
\newcommand{\dvdx}{{\frac{dv}{dx}}} % 
\newcommand{\dxdt}{{\frac{dx}{dt}}} % 
\newcommand{\dxdtt}{{\frac{d^2x}{dt^2}}} % 
\newcommand{\dzdt}{{\frac{dz}{dt}}} % 



% COLORS FOR DIAGRAMS
\definecolor{DarkBlue}{rgb}{0.0,0.0,0.6} % 
% \definecolor{DarkGreen}{rgb}{0.0,0.3,0.0} % 
% \definecolor{DarkRed}{rgb}{0.6,0.0,0.0} % 

% TEST SPECIFIC INFORMATION
\newcommand{\TestName}{Midterm 2 Make-Up}
\newcommand{\TestTime}{}
\newcommand{\Duration}{12 hours }
\newcommand{\Points}{}
% \newcommand{\DueDate}{12:30 PM ET}
\newcommand{\DueDate}{8:30 PM ET}


% \usepackage{tikz}
% \usetikzlibrary{shapes,snakes}   
% \usetikzlibrary{arrows,automata}

\begin{document}
    

\input{2021Spr/coverpage} % cover page for unproctored exam

\newpage


\InitialsRight

\begin{questions}




    \question[2] Determine all values of $\alpha$, if any, for which all solutions tend to zero as $t\to\infty$. $$y'' +(\alpha+2)y'+4y = 0 , \quad \alpha \in \mathbb R$$ Show your work. \vspace{6cm}  
    
    \question[2] Determine whether the functions $y_1 = e^{2t}$ and $y_2 = e^{3t} $ are a fundamental set of solutions to the differential equation $y''+5y'+6y = 0$ with $y=y(t)$. Show your work. 
    
    \newpage
    
    \question[1] State whether the following differential equation is linear or non-linear, and whether it is homogeneous or non-homogeneous. 
    
    $$y'' + (1-y)y' + 2y = 1$$
    
    \vspace{4cm}
    
    \question[3] Construct an initial value problem for the following situation. Show your work.  

    \begin{itemize}
        \item[] A 0.05 Newtons (N) force stretches a spring 0.01 m. A mass weighing 2 kg is attached to the spring, and the spring is also attached to a viscous damper that applies a force of 0.4 N when the velocity of the mass is 0.1 m/s. The mass is pulled down 0.1 m below its equilibrium position and given an initial upward velocity of 0.6 m/s. 
    \end{itemize}
    
    
    \newpage \Initials
    
    \question[3] If $Y = W[f, g]$ is the Wronskian of $f$ and $g$, and if $u=3f+g, \ v=f-3g$, express the Wronskian $W[u,v]$ of $u$ and $v$ in terms of $Y$. Show your work. 
    
    \vspace{8cm}
    
    \question[3] Determine a suitable form for the particular solution \(Y(t)\) if the method of undetermined coefficients is to be used. Please show your work. 
    $$y''+ 11y'+24y = 4\sin(4t)+2e^{-8t}$$    
    
    \newpage \Initials
    
    \question[5] The position of a moving object, $y(t)$, for time $t \ge 0$ satisfies the IVP
    $$y''+ 2y' + 4y=0,\quad y(0)=1,\ y'(0)=4,  \quad y=y(t)$$
    
    \begin{parts}
        \part Express the differential equation in the IVP as a first-order system in the form $\vec x \, ' = A\vec x$.
        
        \vspace{2cm}
        
        \part Solve the DE using any method you like. You do not need to solve the IVP, but show your work. 
        
        \vspace{4cm}
        
        \part Sketch the trajectory of the object for $t\ge 0$ in the phase plane. Indicate the location of the object at time $t=0$, the direction of motion, and label your axes. 
        
    \end{parts}

    
    % \newpage \Initials

    % \question[10] Use the variation of parameters method to identify the general solution to \[\vec{x} \, ' = \left( \begin{array}{rr} 2 & 4 \\ 4 & 2 \end{array} \right) \vec{x}  + \left( \begin{array}{r}  12 \\ 0 \end{array} \right)  \]
    

    \newpage \Initials

    \question[10] Solve the DE. Show your work. $$y'' + 25y = 60\cos(5t), \quad y=y(t)$$
    
    \newpage \Initials

    \question[10] Solve the DE using variation of parameters. Solutions to the homogeneous problem are $y_1 = t^{2}$ and $y_2 = t^{-2}$. Please show your work. $$t^2y'' + ty' -4y = t, \quad y=y(t), \quad t > 0$$    

    \newpage \Initials
    
    \question[1] A small number of points will be allocated for presentation, neatness, and organization. Please ensure that
    \begin{enumerate}
        % \item your scan is under 5 MB in file size
        \item your work is legible in the scan
        \item your name or initials are at the top of every page
        \item questions are answered in the order in which they were given
        \item during the upload process you have indicated which pages correspond to which question, and made sure that none of your pages are upside down or sideways (you can also change the orientation of the pages when you upload in Gradescope)
    \end{enumerate}
    Ensuring that these criteria are met helps ensure that your exam is graded efficiently and accurately. 
    


    
\end{questions}
    
    Please sign and date the following GT Honor Code statement. \\ 
    
    \vspace{6pt}
    \textbf{Georgia Tech Honor Code}\\
    \GTHonorCode
    
    \begin{center}
    \begin{center}
        \def\arraystretch{0.35}%  1 is the default, change whatever you need
        \begin{tabular}{ b{8cm} b{8cm} }
        \vspace{.5cm} \underline{\hspace{7cm}} & \vspace{.5cm} \underline{\hspace{4.5cm}}  \tabularnewline
        \vspace{6pt} signature & \vspace{6pt} date    
        \end{tabular}
    \end{center}
    \end{center}    
\end{document}

