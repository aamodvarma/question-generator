\documentclass[11pt]{exam}

% \usetikzlibrary{calc,patterns}

\newcommand{\Version}{1} 
\newcommand{\Solutions}{0} 

% TEST SPECIFIC INFORMATION
\ifnum \Version=1 \newcommand{\TestName}{Lecture Participation Activity 9: Test 3 Review} \fi


% LOAD PACKAGES
\usepackage{amsmath} % allows for align env and other things
\usepackage{amssymb} % 
\usepackage{mathtools} % allows for single apostrophe
\usepackage{enumitem} % allows for alpha lettering in enumerated lists
\usepackage{lastpage}
\usepackage{array} % for table alignments

\usepackage{graphicx} % if images are needed
\usepackage{wrapfig} % to allow text wrapping

\addpoints

\usepackage{pgfplots} % for surfaces (chapter 7)
\usepackage{tikz-3dplot} 
\pgfplotsset{compat=1.9}
\usetikzlibrary{decorations.pathmorphing,patterns} % for some tikz diagrams
% ~~~~~~~~~~~~~~~~~~~~~~~~~~~~~~~~~~~~
% INITIALS
\newcommand{\Initials}{\textit{\Course, \TestName. Your initials: \underline{\hspace{3cm}}} \vspace{1pt}}

\newcommand{\InitialsLeft}{\noindent \hspace{-18pt}\textit{\Course, \TestName. Your initials: \underline{\hspace{3cm}}} \vspace{1pt}}

\newcommand{\InitialsRight}{\begin{flushright}\textit{\Course, \TestName. Your initials: \underline{\hspace{3cm}}} \vspace{1pt}\end{flushright}}

% ADJUST FIRST LINE IN PARAGRAPH INDENTATION 
\setlength\parindent{0pt}

% FONT FORMAT
\renewcommand*\rmdefault{lmss} % change font to lat mod ss

% ADJUST MARGINS 
\usepackage[a4paper, tmargin=0.8in,bmargin=0.8in,left=1in,right=1in]{geometry}

% TIKZ DIAGRAMS
\usepackage{color}
\usepackage{tikz}  \usetikzlibrary{arrows} 
\usetikzlibrary{calc} 

% COURSE SPECIFIC INFORMATION
\newcommand{\Course}{Math 2552, Differential Equations}
\newcommand{\Instructors}{}

\newcommand{\LastPage}{\begin{center}\textit{This page may be used for scratch work. Please indicate clearly if you would like your work on this page to be graded. }\end{center}   }

% DERIVATIVES
\newcommand{\dfdy}{{\frac{df}{dy}}} % 
\newcommand{\dydt}{{\frac{dy}{dt}}} % 
\newcommand{\dxdt}{{\frac{dx}{dt}}} % 
\newcommand{\dydx}{{\frac{dy}{dx}}} % 
\newcommand{\dydtt}{{\frac{d ^2y}{dt^2}}} % 
\newcommand{\dydxx}{{\frac{d^2y}{dx^2}}} % 
\newcommand{\dydttt}{{\frac{d^3y}{dt^3}}} % 

\newcommand{\ddt}{{\frac{d}{dt}}} % 
\newcommand{\ddx}{{\frac{d}{dx}}} % 
\newcommand{\ddy}{{\frac{d}{dy}}} % 
\newcommand{\dudt}{{\frac{du}{dt}}} % 
\newcommand{\dvdx}{{\frac{dv}{dx}}} % 
\newcommand{\dxdtt}{{\frac{d^2x}{dt^2}}} % 
\newcommand{\dzdt}{{\frac{dz}{dt}}} % 

% COLORS FOR SOLUTIONS
\definecolor{DarkBlue}{rgb}{0.0,0.2,0.4} % 
\definecolor{DarkRed}{rgb}{0.4,0.1,0.1} % 
\definecolor{DarkGreen}{rgb}{0.0,0.25,0.15} % 

% % ADJUST MARGINS
% \usepackage[tmargin=2.0in,bmargin=1.5in]{geometry}
% \geometry{margin=0.76in}

% ADJUST FIRST LINE IN PARAGRAPH INDENTATION 
\setlength\parindent{0pt}

% FONT FORMAT
\renewcommand*\rmdefault{lmss} % change font to lat mod ss


% HEADERS AND FOOTERS
\pagestyle{headandfoot}
\runningfooter{}{}{}
\runningheader{}{}{\textit{\TestName, Page \thepage \ of \pageref{LastPage}} }
% \headheight 42pt % distance from top of page to top of header
% \headsep 24pt % space between header and top of body


\begin{document}
    
\vspace*{-1cm}

\begin{center}
{\Large \TestName}
\end{center}
\newcommand{\ID}{Please print your first name: \framebox{\strut\hspace{4.2cm}}, last name: \framebox{\strut\hspace{4.2cm}}, \\[2pt] and the remaining digits of your GTID:  \framebox{\strut $9$}\framebox{\strut $0$}\framebox{\strut\hspace{0.19cm}}\framebox{\strut\hspace{0.19cm}}\framebox{\strut\hspace{0.19cm}}\framebox{\strut\hspace{0.19cm}}\framebox{\strut\hspace{0.19cm}}\framebox{\strut\hspace{0.19cm}}\framebox{\strut\hspace{0.19cm}}.}

\ID

\vspace{6pt}
\textbf{Instructions}: for full credit please show your work and answer all questions. Your work will be graded for completion, not accuracy. Please submit your work before the end of lecture. 
\def\dm{\displaystyle}

% \def\dm{\displaystyle}

\begin{center}
    {\bf Elementary Laplace Transforms}\\[2pt]
\end{center}
$
\hspace*{-.5em}
\begin{array}{lllllll}
 1.  & 1 \quad  & 1/s,\quad s>0 & \qquad  \ \quad & 10.  & t^n e^{at}\  \quad  & \frac{n!}{(s-a)^{n+1}},\quad s>a\\[1ex] 
 2.  & e^{at} \quad  & 1/(s-a),\quad s>a  & \qquad  \ \quad & 11. & u_c(t)\ \ (c\ge 0) \quad  & e^{-cs}/s,\quad s>0 \\[1ex]
 3.  & t^n  \quad  & n!/s^{n+1},\quad s>0 & \qquad  \ \quad & 12.  & u_c(t)f(t-c)\  \quad  & \dm e^{-cs} F(s), \ c\ge 0 \\[1ex] 
 4.  & \sin(at) \quad  & a/(s^2+a^2),\quad s>0  & \qquad  \ \quad & 13.  & e^{ct}f(t) \quad  & \dm F(s-c) \\[1ex] 
 5.  & \cos(at) \quad  & s/(s^2+a^2),\quad s>0  & \qquad  \ \quad & 14.  & t^n f(t) \quad  & \dm (-1)^n F^{(n)}(s) \\[1ex]
 6.  & e^{at}\sin(bt) \quad  & \dm\frac{b}{(s-a)^2+b^2},\quad s>a  & \qquad  \ \quad & 15. & f(t-T) = f(t) \quad  & \dm \frac{1}{1-e^{-sT}}\int_0^Te^{-st}f(t)\,dt \\[1ex]
 7.  & e^{at}\cos(bt) \quad  & \dm\frac{s-a}{(s-a)^2+b^2},\quad s>a & \qquad  \ \quad &  16. & \delta(t-c) & e^{-cs}\\[1ex]
 8.  & f'(t) \  \quad  &  sF(s) - f(0) & \qquad  \ \quad & 17. & f\ast g & F(s)G(s) \\[1ex]
 9.  & f''(t) \  \quad   & s^2F(s) - sf(0) - f'(0) & \qquad  \ \quad &  & & \\[1ex]
\end{array} 
$

\subsection*{Exercises}
\begin{questions}


    \question[2] Calculate the Laplace transforms of the following. 
    \begin{parts}
        \part The function $f(t)$ has period $2$ and 
        $$f(t) = \begin{cases} e^{-2t}, \quad 0 \le t < 1 \\ 0, \quad 1 \le t < 2 \end{cases}$$
        \vfill
        % \part $f(t) = u_4\cos(2(t-4))$ \vfill 
        \part $f(t) = \begin{cases}0, \quad 0 \le t < 1 \\ t-4, \quad 1 \le t < 2 \\ 0 , \quad t \ge 2\end{cases}$
        \vfill

    \end{parts}

        \newpage 
   \question[2] 
    % FROM 2019 M3A 
    Consider the system: $\displaystyle dx/dt = x(1-x-y), \quad dy/dt = y(1.5-y-x)$. % Q HW7.3#5 
    
    % $\displaystyle dx/dt = x+y^2, \quad dy/dt = x + 2y$. % A HW7.2#3
       
    \begin{parts} 
        \part Sketch the set of points where solution curves are horizontal, and the set of points where solution curves are vertical. 
        \vfill
        \part Identify all critical points of the system. You can use your sketch from the previous part. 
        \vspace{3cm}
        \part Use eigenvalues to classify the critical point at $(1,0)$ according to stability (stable, unstable, asymptotically stable) and type (saddle, proper node, etc). 
        \vfill
    \end{parts}
    
    \newpage 


\vspace{-6pt}     
\end{questions}
\end{document}