\ifnum \Solutions=0
\newpage 
\fi
\ifnum \Version=1    
    \question[6] Use the Laplace transform to solve the following IVP. Do not leave your answer in terms of an integral. Please show your work.
    % $$\displaystyle y''-4y'+4y=0,\qquad y(0)=1,\quad y'(0)=1$$ %Q,A HW 5.4#4
    % $$\displaystyle y''-2y'+4y=0,\qquad y(0)=2,\quad y'(0)=0$$ %M HW 5.4#5
    $$ y'' -6y'+13y = 0, \quad y(0) = 0, \quad y'(0) = -3$$. % A TRIM P301 #27
    % $$ y''+ 2y'+y = 0, \quad y(0) = 1, \quad y'(0) = 1$$. % B TRIM P301 #23
\ifnum \Solutions=1 {\color{DarkBlue} 
Take the transform of the DE and solve for $Y(s) = \mathcal{L}[y(t)]$.
\begin{align}
    0
    &= s^2Y-sy(0) - y'(0) - 6(sY-y(0)) + 13Y \\
    &= s^2Y-0 +3 - 6(sY-0) + 13Y \\
    -3 &= s^2Y - 6sY + 13Y \\
    -3 &= (s^2 - 6s + 13)Y \\
    Y &= \frac{-3}{s^2- 6s+13} 
\end{align}    
Complete the square:
\begin{align}
    Y &= \frac{-3}{s^2- 6s+9-9+13} = \frac{-3}{ (s-3)^2+4}=\frac{-3}{2}\cdot \frac{2}{ (s-3)^2+4}
\end{align}    
Inverse transform:
\begin{align}
    y(t) &= \frac{-3}{2}\sin(2t)e^{3t}
\end{align}
} 
\else 
\vspace{3cm}
\fi
\fi 



\ifnum \Version=2
    \question[6] Consider the IVP: $y''+4y'+3y = 4e^{-2t}$, $y(0)=0, y'(0) = 0$. 
        \begin{parts}
        \part Use the Laplace Transform to obtain an explicit expression for $Y(s) = \mathcal{L}[y]$. 
        \ifnum \Solutions=1 {\color{DarkBlue} \\[12pt] 
        Taking the transform of the IVP yields
        \begin{align}
            s^2Y-sy(0) - y'(0) +4(sY-y(0)) +3 Y &= \frac{4}{s+2} \\
            (s^2 +4s+3)Y &= \frac{4}{s+2} \\
            Y&= \frac{1}{s+2} \frac{4}{s^2 +4s+3} \\
            &= \frac{4}{(s+1)(s+2)(s+3)}
        \end{align}
        } 
        \else 
        \vfill
        \fi
        \part Use the inverse Laplace Transform to solve the IVP. 
        \ifnum \Solutions=1 {\color{DarkBlue} \\[12pt] 
        Partial fractions: 
        \begin{align}
            \frac{4}{(s+2)(s+1)(s+3)} & = \frac{A}{s+1} + \frac{B}{s+2} + \frac{C}{s+3} \\
            4 & = A(s+2)(s+3) + B(s+1)(s+3) + C(s+1)(s+2) 
        \end{align}
        Selecting values of $s$ yields
        \begin{align}
            s=-1: \quad 4 & = 2A + 0 + 0 \quad \Rightarrow \quad A = 2 \\
            s=-2: \quad 4 & = 0 - B + 0 \quad \Rightarrow \quad B = -4 \\
            s=-3: \quad 4 & = 0 + 0 + 2C \quad \Rightarrow \quad C = 2
        \end{align}
        Thus
       \begin{align}
            Y (s) = \frac{4}{(s+1)(s+2)(s+3)} 
            &= \frac{A}{s+1} + \frac{B}{s+2} + \frac{C}{s+3} \\
            &= \frac{2}{s+1} + \frac{-4}{s+2} + \frac{2}{s+3} 
        \end{align}        
        With the inverse transform this becomes 
        \begin{align}
            y(t) &= 2e^{-t} -4 e^{-2t} +2 e^{-3t}
        \end{align}
        } 
        \else 
        \vfill
        \fi
    \end{parts}
\fi 




\ifnum \Version>5
    \question[4] Consider the IVP: $y''+4y = 8e^{-2t}$, $y(0)=0, y'(0) = 0$. 
        \begin{parts}
        \part Use the Laplace Transform to obtain an explicit expression for $Y(s) = \mathcal{L}[y]$. 
        \ifnum \Solutions=1 {\color{DarkBlue} \\[12pt] 
        Taking the transform of the IVP yields
        \begin{align}
            s^2Y-sy(0) - y'(0) + 4 Y &= \frac{8}{s+2} \\
            (s^2 + 4)Y &= \frac{8}{s+2} \\
            Y&= \frac{8}{s+2} \frac{1}{s^2 + 4} \\
            &= \frac{8}{(s^2+4)(s+2)}
        \end{align}
        } 
        \else 
        \vfill
        \fi
        \part Use the inverse Laplace Transform to solve the IVP. 
        \ifnum \Solutions=1 {\color{DarkBlue} \\[12pt] 
        Partial fractions: 
        \begin{align}
            \frac{8}{(s^2+4)(s+2)} & = \frac{As+B}{s^2+4} + \frac{C}{s+2} \\
            8 & = (As+B)(s+2) + C(s^2+4) \\
            8 &= (A+C)s^2 +(2A+B)s+(2B+4C)
        \end{align}
        Selecting $s = -2$ yields
        \begin{align}
            s=-2: \quad 8 & = 0 + C((-2)^2+4) \quad \Rightarrow \quad C = 1 
        \end{align}
        The quadratic terms give us 
        \begin{align}
            A+C = 0 \Rightarrow \quad A = - 1 
        \end{align}
        The linear terms give us 
        \begin{align}
            2A+B = 0 \Rightarrow \quad B = 2
        \end{align}        
        Thus
       \begin{align}
            Y (s) = \frac{8}{(s^2+4)(s+2)} 
            &= \frac{As+B}{s^2+4} + \frac{C}{s+2} \\
            &= \frac{-s+2}{s^2+4} + \frac{1}{s+2} \\
            &= -\frac{s}{s^2+4} + \frac{2}{s^2+4} + \frac{1}{s+2} \\
        \end{align}        
        With the inverse transform this becomes 
        \begin{align}
            y(t) &= \sin(2t) - \cos(2t) + e^{-2t}
        \end{align}
        \textbf{Additional Notes}\\
        Some students made an error in taking the transform that resulted in the expression
        $$Y = \frac{8}{s(s+4)(s+2)} = \frac1s+\frac{1}{s+4}- \frac{2}{s+2}$$
        } 
        \else 
        \vfill
        \fi
    \end{parts}
\fi 

