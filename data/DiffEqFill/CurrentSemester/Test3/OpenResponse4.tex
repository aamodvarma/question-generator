\ifnum \Solutions=0
\newpage 
\fi

\ifnum \Version=1    
\question[2] 
Compute the Laplace transform of the function below. Do not leave your answer in terms of an integral. Please show your work. 
$$y(t) = \begin{cases} 0, \quad 0 \le t < 1 \\ t-2, \quad 1 \le t < 3 \\ 0, \quad 3 \le t < \infty\end{cases}$$
\ifnum \Solutions=1 {\color{DarkBlue} \\[12pt] 
There are a few different ways to solve this. Any of the approaches below are sufficient. 
\subsection*{Method 1: Direct Integration }
Direct integration requires integration by parts, but using the definition of the Laplace Transform:
    \begin{align}
        \int_0^{\infty} e^{-st} y(t) \, dt 
        &= \int_1^{3} e^{-st} \, (t-2) \, dt \\
        &= \int_1^{3} te^{-st}  \, dt - 2 \int_1^{3} e^{-st}  \, dt \\
        &=  \left. t\cdot \frac{-1}{s}e^{-st} \right|_1^3
        - \int_1^{3} \frac{-1}{s}e^{-st} \, dt 
        -2 \left( \left. \frac{-1}{s}e^{-st}\right|_{t=1}^{t=3}\right) \\
        &=  \frac{-1}{s}\left( 3e^{-3s} - e^{-s} \right)  
        + \frac{1}{s} \int_1^{3} e^{-st} \, dt 
        + \frac{2}{s} \left(  e^{-3s} - e^{-s} \right) \\
        &=  \frac{-1}{s}\left( 3e^{-3s} - e^{-s} \right)  
        + \frac{1}{s} \left( \left. \frac{-1}{s}e^{-st}\right|_{t=1}^{t=3}\right) \, dt 
        + \frac{2}{s} \left(  e^{-3s} - e^{-s} \right) \\     
        &=  \frac{-1}{s}\left( 3e^{-3s} - e^{-s} \right)  
        - \frac{1}{s^2} \left(  e^{-3s} - e^{-s} \right) 
        + \frac{2}{s} \left(  e^{-3s} - e^{-s} \right) \\      
        &= - \frac{1}{s^2} \left(  e^{-3s} - e^{-s} \right) 
        - \frac{ e^{-3s}}{s}   - \frac{e^{-s}}{s}  \\   
        &= \frac{e^{-s}}{s^2} - \frac{e^{-3s}}{s^2} 
        - \frac{ e^{-3s}}{s}   - \frac{e^{-s}}{s}           
    \end{align}
    \subsection*{Method 2: Step Functions}
    We could also use the table of transforms, specifically the transform of the step function $u_c(t)$ and its product with another function. 
    \begin{align}
        y(t) &= 0u_{0,1} + (t-2)u_{1,3} + 0u_{3} 
        =(t-2)(u_1-u_3) 
        = tu_1-tu_3 - 2u_1 + 2 u_3 
    \end{align}   
    At this point we can either take the transform and use the derivative theorem (in the table) to work out the transform of $tu_1$ and $tu_3$, or we can use the following trick:
    \begin{align}
        y(t) &= tu_1-tu_3 - 2u_1 + 2 u_3 \\
        &= (t+1-1)u_1-(t+3-3)u_3 - 2u_1 + 2 u_3 \\
        &= (t-1)u_1 + u_1 -(t-3)u_3 - 3u_3 - 2u_1 + 2 u_3 \\
        &= (t-1)u_1 -(t-3)u_3 - u_1 - u_3 
    \end{align}
    Using the table of transforms the Laplace transform is
    \begin{align}
        Y(s) = \frac{e^{-s}}{s^2} - \frac{e^{-3s}}{s^2} - \frac{e^{-s}}{s} - \frac{e^{-3s}}{s}
    \end{align}
} 
\else 
    \vfill
    \begin{center}
        \textit{This remainder of this page can be used for scratch work. }
    \end{center}
    \vfill
\fi
\fi


\ifnum \Version=5
\question[2] 
Compute the Laplace transform of the function below. Do not leave your answer in terms of an integral. Please show your work. 
$$y(t) = \begin{cases} 0, \quad 0 \le t < 1 \\ 4, \quad 1 \le t < 3 \\ 0, \quad 3 \le t < \infty\end{cases}$$
\ifnum \Solutions=1 {\color{DarkBlue} \\[12pt] 
There are a few different ways to solve this. Any of the approaches below are sufficient. 
\subsection*{Method 1: Direct Integration }
Direct integration uses the definition of the Laplace Transform:
    \begin{align}
        \int_0^{\infty} e^{-st} y(t) \, dt 
        = \int_1^{3} e^{-st} \, 4 \, dt 
        &= 4\int_1^{3} e^{-st}  \, dt  \\         
        &= 4\left.\frac{-1}{s} e^{-st} \right|_1^3  \\         
        &= \frac{-4}{s} \left(e^{-3s}  - e^{-s} \right)     
    \end{align}
    \subsection*{Method 2: Step Functions}
    We could also use the table of transforms, specifically the transform of the step function $u_c(t)$ and its product with another function. 
    \begin{align}
        y(t) &= 0u_{0,1} + 4u_{1,3} + 0u_{3} 
        =4(u_1-u_3) 
        = 4u_1 - 4 u_3 
    \end{align}   
    Using the table of transforms the Laplace transform is
    \begin{align}
        Y(s) = 4 \frac{e^{-s}}{s} - 4\frac{e^{-3s}}{s}
    \end{align}
    \newpage
} 
\else 
    \vfill
    \begin{center}
        \textit{This remainder of this page can be used for scratch work. }
    \end{center}
    \vfill
\fi
\fi







\ifnum \Version=6
\question[3] 
The function $f(t)$ is periodic, has period $4$, and 
        $$f(t) = \begin{cases} e^{-3t}, \quad 0 \le t < 2 \\ 0, \quad 2 \le t < 4 \end{cases}$$
        Compute the Laplace transform of $f(t)$. Do not leave your answer in terms of an integral. Please show your work. 

\ifnum \Solutions=1 {\color{DarkBlue} 
    We use the formula for a periodic function with $T=4$. 
    
        With $T=4$,
        
        \begin{align}
            \mathcal{L}\{f(t)\} &=  \frac{\int_{0}^{T} e^{-st} f(t) dt}{1 - e^{-Ts}} \\
            &= \frac{1}{1 - e^{-4s}} \left( \int_{0}^{2} e^{-st} e^{-3t} dt + \int_2^4 0 \, dt \right)\\
            &= \frac{1}{1 - e^{-4s}} \cdot \frac{1}{-s-3} \left. e^{(-s-3)t}\right|_0^2 \\
            &= -\frac{1}{1 - e^{-4s}} \cdot \frac{1}{s+3} \left( e^{-2(s+3)} - 1 \right)
        \end{align}
        \newpage
} 
\else 
    \newpage
    \begin{center}
        \textit{This remainder of this page can be used for scratch work. }
    \end{center}
    \vfill
\fi
\fi



\ifnum \Version=7
\question[3] 
The function $f(t)$ is periodic, has period $4$, and 
        $$f(t) = \begin{cases} e^{-2t}, \quad 0 \le t < 3 \\ 0, \quad 3 \le t < 4 \end{cases}$$
        Compute the Laplace transform of $f(t)$. Do not leave your answer in terms of an integral. Please show your work. 

\ifnum \Solutions=1 {\color{DarkBlue} 
    We use the formula for a periodic function with $T=4$. 
    
        With $T=4$,
        
        \begin{align}
            \mathcal{L}\{f(t)\} &=  \frac{\int_{0}^{T} e^{-st} f(t) dt}{1 - e^{-Ts}} \\
            &= \frac{1}{1 - e^{-4s}} \left( \int_{0}^{3} e^{-st} e^{-2t} dt + \int_3^4 0 \, dt \right)\\
            &= \frac{1}{1 - e^{-4s}} \cdot \frac{1}{-s-2} \left. e^{(-s-2)t}\right|_0^3 \\
            &= -\frac{1}{1 - e^{-4s}} \cdot \frac{1}{s+2} \left( e^{-3(s+2)} - 1 \right)
        \end{align}
} 
\else 
    \newpage
    \begin{center}
        \textit{This remainder of this page can be used for scratch work. }
    \end{center}
    \vfill
\fi
\fi



