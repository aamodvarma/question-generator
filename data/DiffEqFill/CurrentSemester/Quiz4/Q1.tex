\ifnum \Version=1

\question[2] A spring is hung from a horizontal surface as in the figure below. A mass of $m=0.2$ kg stretches the spring 0.04 m. Frictional forces also exert a force on the mass while it is moving that is proportional to the velocity of the mass. The proportionality constant for these frictional forces is 2 N s/m. The mass is then pulled down an additional 0.03 m and then released from rest. Assume that position of the mass, $y$, increases as the mass moves down. Write down an IVP based on this physical description that describes the position of the mass $y(t)$, as a function of time, $t$. You do not need to solve your IVP. \\[4pt]
\input{2023Summer/Quiz4/DiagramSpringMass}
\vspace{2cm}
\fi 


\ifnum \Version=2
\question[2] A small car with mass $m = 0.2$ kg is moving along a straight line on a horizontal surface. The car is attached to a spring. The spring exerts a force of $4$ N when it is extended a distance of $0.8$ meters from its equilibrium position. Frictional forces also exert a force on the car while it is moving that is proportional to the velocity of the car. The proportionality constant for these frictional forces is $0.6$ N s/m. The car also has an engine that pushes the car forward with a force of $F = 0.5t$ N. The car is released from rest after it is moved 0.01 m to the right of its equilibrium position. Assume that position of the car, $y$, increases as the car move to the right. Write down the IVP based on this physical description that describes the position of the car as a function of time, $t$. You do not need to solve your IVP. \\
\hspace{1cm}\begin{tikzpicture}
\tikzstyle{spring}=[thick,decorate,decoration={zigzag,pre length=0.3cm,post length=0.3cm,segment length=6}]
\tikzstyle{damper}=[thick,decoration={markings,  
  mark connection node=dmp,
  mark=at position 0.5 with 
  {
    \node (dmp) [thick,inner sep=0pt,transform shape,rotate=-90,minimum width=15pt,minimum height=3pt,draw=none] {};
    \draw [thick] ($(dmp.north east)+(2pt,0)$) -- (dmp.south east) -- (dmp.south west) -- ($(dmp.north west)+(2pt,0)$);
    \draw [thick] ($(dmp.north)+(0,-5pt)$) -- ($(dmp.north)+(0,5pt)$);
  }
}, decorate]

% WALL PATTERN
\tikzstyle{ground}=[fill,pattern=north east lines,draw=none,minimum width=0.5cm,minimum height=0.3cm,inner sep=0pt,outer sep=0pt]

% CART
\node [style={draw,outer sep=0pt,very thick}] (M) [minimum width=1cm, minimum height=1.0cm] {$m$};
% WHEELS
\draw [very thick] (M.south west) ++ (0.2cm,-0.125cm) circle (0.125cm)  (M.south east) ++ (-0.2cm,-0.125cm) circle (0.125cm);

% BOTTOM GROUND
\node (ground) [ground,anchor=north,yshift=-0.25cm,minimum width=5.6cm,xshift=-0.03cm] at (M.south) {};
\draw (ground.north east) -- (ground.north west);
\draw (ground.south east) -- (ground.south west);
\draw (ground.north east) -- (ground.south east);

% WEST WALL
\node (wall) [ground, rotate=-90, minimum width=3cm,xshift=-0.42cm,yshift=-3cm] {};
\draw (wall.north east) -- (wall.north west);
\draw (wall.north west) -- (wall.south west);
\draw (wall.south west) -- (wall.south east);
\draw (wall.south east) -- (wall.north east);

% DAMPER AND SPRING
\draw [spring] (wall.20) -- ($(M.north west)!(wall.20)!(M.south west)$);
% \node (y) at (wall.130) [yshift = 0.02cm,xshift=1.2cm] {$k$};

\end{tikzpicture}

\fi 

\ifnum \Version=3
\question[2] A small car with mass $m = 0.1$ kg is moving along a straight line on a horizontal surface. The car is attached to a spring. The spring exerts a force of $2$ N when it is extended a distance of $0.1$ meters from its equilibrium position. Frictional forces also exert a force on the car while it is moving that is proportional to the velocity of the car. The proportionality constant for these frictional forces is $0.3$ N s/m. The car also has an engine that pushes the car forward with a force of $F = 0.1$ N. The car is released from rest after it is moved 0.01 m to the left of its equilibrium position. Assume that position of the car, $y$, increases as the car move to the right. Write down the IVP based on this physical description that describes the position of the car as a function of time, $t$. You do not need to solve your IVP. \\
\hspace{1cm}\begin{tikzpicture}
\tikzstyle{spring}=[thick,decorate,decoration={zigzag,pre length=0.3cm,post length=0.3cm,segment length=6}]
\tikzstyle{damper}=[thick,decoration={markings,  
  mark connection node=dmp,
  mark=at position 0.5 with 
  {
    \node (dmp) [thick,inner sep=0pt,transform shape,rotate=-90,minimum width=15pt,minimum height=3pt,draw=none] {};
    \draw [thick] ($(dmp.north east)+(2pt,0)$) -- (dmp.south east) -- (dmp.south west) -- ($(dmp.north west)+(2pt,0)$);
    \draw [thick] ($(dmp.north)+(0,-5pt)$) -- ($(dmp.north)+(0,5pt)$);
  }
}, decorate]

% WALL PATTERN
\tikzstyle{ground}=[fill,pattern=north east lines,draw=none,minimum width=0.5cm,minimum height=0.3cm,inner sep=0pt,outer sep=0pt]

% CART
\node [style={draw,outer sep=0pt,very thick}] (M) [minimum width=1cm, minimum height=1.0cm] {$m$};
% WHEELS
\draw [very thick] (M.south west) ++ (0.2cm,-0.125cm) circle (0.125cm)  (M.south east) ++ (-0.2cm,-0.125cm) circle (0.125cm);

% BOTTOM GROUND
\node (ground) [ground,anchor=north,yshift=-0.25cm,minimum width=5.6cm,xshift=-0.03cm] at (M.south) {};
\draw (ground.north east) -- (ground.north west);
\draw (ground.south east) -- (ground.south west);
\draw (ground.north east) -- (ground.south east);

% WEST WALL
\node (wall) [ground, rotate=-90, minimum width=3cm,xshift=-0.42cm,yshift=-3cm] {};
\draw (wall.north east) -- (wall.north west);
\draw (wall.north west) -- (wall.south west);
\draw (wall.south west) -- (wall.south east);
\draw (wall.south east) -- (wall.north east);

% DAMPER AND SPRING
\draw [spring] (wall.20) -- ($(M.north west)!(wall.20)!(M.south west)$);
% \node (y) at (wall.130) [yshift = 0.02cm,xshift=1.2cm] {$k$};

\end{tikzpicture}

\fi 

\ifnum \Version=4
\question[2] A small car with mass $m = 0.2$ kg is moving along a straight line on a horizontal surface. The car is attached to a spring. The spring exerts a force of $4$ N when it is extended a distance of $0.8$ meters from its equilibrium position. Frictional forces also exert a force on the car while it is moving that is proportional to the velocity of the car. The proportionality constant for these frictional forces is $0.6$ N s/m. The car also has an engine that pushes the car forward with a force of $F = 0.5t$ N. The car is released from rest after it is moved 0.05 m to the left of its equilibrium position. Assume that position of the car, $y$, increases as the car move to the right. Write down the IVP based on this physical description that describes the position of the car as a function of time, $t$. You do not need to solve your IVP. \\
\hspace{1cm}\begin{tikzpicture}
\tikzstyle{spring}=[thick,decorate,decoration={zigzag,pre length=0.3cm,post length=0.3cm,segment length=6}]
\tikzstyle{damper}=[thick,decoration={markings,  
  mark connection node=dmp,
  mark=at position 0.5 with 
  {
    \node (dmp) [thick,inner sep=0pt,transform shape,rotate=-90,minimum width=15pt,minimum height=3pt,draw=none] {};
    \draw [thick] ($(dmp.north east)+(2pt,0)$) -- (dmp.south east) -- (dmp.south west) -- ($(dmp.north west)+(2pt,0)$);
    \draw [thick] ($(dmp.north)+(0,-5pt)$) -- ($(dmp.north)+(0,5pt)$);
  }
}, decorate]

% WALL PATTERN
\tikzstyle{ground}=[fill,pattern=north east lines,draw=none,minimum width=0.5cm,minimum height=0.3cm,inner sep=0pt,outer sep=0pt]

% CART
\node [style={draw,outer sep=0pt,very thick}] (M) [minimum width=1cm, minimum height=1.0cm] {$m$};
% WHEELS
\draw [very thick] (M.south west) ++ (0.2cm,-0.125cm) circle (0.125cm)  (M.south east) ++ (-0.2cm,-0.125cm) circle (0.125cm);

% BOTTOM GROUND
\node (ground) [ground,anchor=north,yshift=-0.25cm,minimum width=5.6cm,xshift=-0.03cm] at (M.south) {};
\draw (ground.north east) -- (ground.north west);
\draw (ground.south east) -- (ground.south west);
\draw (ground.north east) -- (ground.south east);

% WEST WALL
\node (wall) [ground, rotate=-90, minimum width=3cm,xshift=-0.42cm,yshift=-3cm] {};
\draw (wall.north east) -- (wall.north west);
\draw (wall.north west) -- (wall.south west);
\draw (wall.south west) -- (wall.south east);
\draw (wall.south east) -- (wall.north east);

% DAMPER AND SPRING
\draw [spring] (wall.20) -- ($(M.north west)!(wall.20)!(M.south west)$);
% \node (y) at (wall.130) [yshift = 0.02cm,xshift=1.2cm] {$k$};

\end{tikzpicture}

\fi 


\ifnum \Version=5
\question[2] A small car with mass $m = 0.2$ kg is moving along a straight line on a horizontal surface. The car is attached to a spring. The spring exerts a force of $8$ N when it is extended a distance of $0.2$ meters from its equilibrium position. Frictional forces also exert a force on the car while it is moving that is proportional to the velocity of the car. The proportionality constant for these frictional forces is $0.3$ N s/m. The car also has an engine that pushes the car forward with a force of $F = 2t$ N. The car is released from rest after it is moved 0.01 m to the left of its equilibrium position. Assume that position of the car, $y$, increases as the car move to the right. Write down the IVP based on this physical description that describes the position of the car as a function of time, $t$. You do not need to solve your IVP. \\
\hspace{1cm}\begin{tikzpicture}
\tikzstyle{spring}=[thick,decorate,decoration={zigzag,pre length=0.3cm,post length=0.3cm,segment length=6}]
\tikzstyle{damper}=[thick,decoration={markings,  
  mark connection node=dmp,
  mark=at position 0.5 with 
  {
    \node (dmp) [thick,inner sep=0pt,transform shape,rotate=-90,minimum width=15pt,minimum height=3pt,draw=none] {};
    \draw [thick] ($(dmp.north east)+(2pt,0)$) -- (dmp.south east) -- (dmp.south west) -- ($(dmp.north west)+(2pt,0)$);
    \draw [thick] ($(dmp.north)+(0,-5pt)$) -- ($(dmp.north)+(0,5pt)$);
  }
}, decorate]

% WALL PATTERN
\tikzstyle{ground}=[fill,pattern=north east lines,draw=none,minimum width=0.5cm,minimum height=0.3cm,inner sep=0pt,outer sep=0pt]

% CART
\node [style={draw,outer sep=0pt,very thick}] (M) [minimum width=1cm, minimum height=1.0cm] {$m$};
% WHEELS
\draw [very thick] (M.south west) ++ (0.2cm,-0.125cm) circle (0.125cm)  (M.south east) ++ (-0.2cm,-0.125cm) circle (0.125cm);

% BOTTOM GROUND
\node (ground) [ground,anchor=north,yshift=-0.25cm,minimum width=5.6cm,xshift=-0.03cm] at (M.south) {};
\draw (ground.north east) -- (ground.north west);
\draw (ground.south east) -- (ground.south west);
\draw (ground.north east) -- (ground.south east);

% WEST WALL
\node (wall) [ground, rotate=-90, minimum width=3cm,xshift=-0.42cm,yshift=-3cm] {};
\draw (wall.north east) -- (wall.north west);
\draw (wall.north west) -- (wall.south west);
\draw (wall.south west) -- (wall.south east);
\draw (wall.south east) -- (wall.north east);

% DAMPER AND SPRING
\draw [spring] (wall.20) -- ($(M.north west)!(wall.20)!(M.south west)$);
% \node (y) at (wall.130) [yshift = 0.02cm,xshift=1.2cm] {$k$};

\end{tikzpicture}

\fi 



\ifnum \Version=6

\question[6] Solve the following Cauchy-Euler equation. 
$$t^2y'' - 6ty' + 6y = 0, \quad t > 0$$

You may find the following table of formulas to be helpful. 

\begin{center}
    \begin{tabular}{ p{6.2cm} p{6cm} }
        roots &  general solution 
        \\[2pt] \hline 
        real distinct, $m_1 \ne m_2$ &  $c_1 t^{m_1} + c_2 t^{m_2}$\\       
        real repeated, $m = m_1 = m_2$ & $c_1 t^{m} + c_2 t^m \ln(t)$\\
        complex, $m_1 = \alpha + i \beta$ & $c_1t^{\alpha}\cos(\beta \ln(t)) + c_2t^{\alpha}\sin(\beta \ln (t))$\\[2pt] \hline
    \end{tabular}    
\end{center}

\ifnum \Solutions=1 {\color{DarkBlue} 
\textbf{Solutions:}

To solve the differential equation \( t^2 y'' - 6t y' + 6y = 0 \), we recognize it as a Cauchy-Euler equation. The solution to a Cauchy-Euler equation can be found by assuming a solution of the form \( y = t^r \). 

\begin{enumerate}
    \item First, we substitute \( y = t^r \) into the differential equation. We need the first and second derivatives of \( y \):
   \[ y = t^r \]
   \[ y' = r t^{r-1} \]
   \[ y'' = r (r-1) t^{r-2} \]

    \item Substitute these into the differential equation:
   \[ t^2 (r (r-1) t^{r-2}) - 6t (r t^{r-1}) + 6 t^r = 0 \]
   \[ r (r-1) t^r - 6r t^r + 6 t^r = 0 \]
   \[ (r (r-1) - 6r + 6) t^r = 0 \]
   \[ (r^2 - r - 6r + 6) t^r = 0 \]
   \[ (r^2 - 7r + 6) t^r = 0 \]

    \item Since \( t^r \neq 0 \) for \( t \neq 0 \), we set the characteristic polynomial to zero:
   \[ r^2 - 7r + 6 = 0 \]
   \[ (r - 1)(r - 6) = 0 \]
   So \( r = 1 \) and \( r = 6 \).

    \item Therefore, the general solution to the differential equation is:
   \[ y(t) = C_1 t^1 + C_2 t^6 \]
   \[ y(t) = C_1 t + C_2 t^6 \]

where \( C_1 \) and \( C_2 \) are arbitrary constants.
\end{enumerate}


} 
\else 
\newpage
\fi
\fi 


\ifnum \Version=7
\question[6] Solve the following Cauchy-Euler equation. 
$$t^2y'' - 5ty' + 5y = 0, \quad t > 0$$

You may find the following table of formulas to be helpful. 

\begin{center}
    \begin{tabular}{ p{6.2cm} p{6cm} }
        roots &  general solution 
        \\[2pt] \hline 
        real distinct, $m_1 \ne m_2$ &  $c_1 t^{m_1} + c_2 t^{m_2}$\\       
        real repeated, $m = m_1 = m_2$ & $c_1 t^{m} + c_2 t^m \ln(t)$\\
        complex, $m_1 = \alpha + i \beta$ & $c_1t^{\alpha}\cos(\beta \ln(t)) + c_2t^{\alpha}\sin(\beta \ln (t))$\\[2pt] \hline
    \end{tabular}    
\end{center}

\ifnum \Solutions=1 {\color{DarkBlue} 
\textbf{Solutions:}
To solve the differential equation \( t^2 y'' - 5t y' + 5y = 0 \), we recognize it as a Cauchy-Euler equation. The solution to a Cauchy-Euler equation can be found by assuming a solution of the form \( y = t^r \). 

\begin{enumerate}
    \item First, we substitute \( y = t^r \) into the differential equation. We need the first and second derivatives of \( y \):
   \[ y = t^r \]
   \[ y' = r t^{r-1} \]
   \[ y'' = r (r-1) t^{r-2} \]

    \item Substitute these into the differential equation and combine like terms:
   \[ t^2 (r (r-1) t^{r-2}) - 5t (r t^{r-1}) + 5 t^r = 0 \]
   \[ r (r-1) t^r - 5r t^r + 5 t^r = 0 \]
   \[ (r (r-1) - 5r + 5) t^r = 0 \]
   \[ (r^2 - r - 5r + 5) t^r = 0 \]
   \[ (r^2 - 6r + 5) t^r = 0 \]

    \item Since \( t^r \neq 0 \) for \( t \neq 0 \), we set the characteristic polynomial to zero:
   \[ r^2 - 6r + 5 = 0 \]

    \item Solve the characteristic polynomial:
   \[ r^2 - 6r + 5 = 0 \quad \Rightarrow \quad (r - 1)(r - 5) = 0 \]
   So \( r = 1 \) and \( r = 5 \).

    \item Therefore, the general solution to the differential equation is:
   \[ y(t) = C_1 t^1 + C_2 t^5 \]
   \[ y(t) = C_1 t + C_2 t^5 \]

where \( C_1 \) and \( C_2 \) are arbitrary constants.
\end{enumerate}


} 
\else 
\newpage
\fi
\fi 
