\ifnum \Version=1
\question[2] A small car with mass $m = 0.2$ kg is moving along a straight line on a horizontal surface. The car is attached to a spring that exerts a force of $4$ N when it is extended $0.8$ meters from its equilibrium position. Frictional forces also exert a force on the car while it is moving that is proportional to the velocity of the car. The proportionality constant for these frictional forces is $0.6$ N s/m. The car also has an engine that pushes the car forward with a force of $F = 0.5t$ N. The car is released from rest after it is moved 0.01 m to the right of its equilibrium position. Assume that position of the car, $y$, increases as the car moves to theright. Write down the IVP based on this physical description that describes the position of the car as a function of time, $t$. You do not need to solve your IVP. \\[4pt]
\hspace{1cm}\begin{tikzpicture}
\tikzstyle{spring}=[thick,decorate,decoration={zigzag,pre length=0.3cm,post length=0.3cm,segment length=6}]
\tikzstyle{damper}=[thick,decoration={markings,  
  mark connection node=dmp,
  mark=at position 0.5 with 
  {
    \node (dmp) [thick,inner sep=0pt,transform shape,rotate=-90,minimum width=15pt,minimum height=3pt,draw=none] {};
    \draw [thick] ($(dmp.north east)+(2pt,0)$) -- (dmp.south east) -- (dmp.south west) -- ($(dmp.north west)+(2pt,0)$);
    \draw [thick] ($(dmp.north)+(0,-5pt)$) -- ($(dmp.north)+(0,5pt)$);
  }
}, decorate]

% WALL PATTERN
\tikzstyle{ground}=[fill,pattern=north east lines,draw=none,minimum width=0.5cm,minimum height=0.3cm,inner sep=0pt,outer sep=0pt]

% CART
\node [style={draw,outer sep=0pt,very thick}] (M) [minimum width=1cm, minimum height=1.0cm] {$m$};
% WHEELS
\draw [very thick] (M.south west) ++ (0.2cm,-0.125cm) circle (0.125cm)  (M.south east) ++ (-0.2cm,-0.125cm) circle (0.125cm);

% BOTTOM GROUND
\node (ground) [ground,anchor=north,yshift=-0.25cm,minimum width=5.6cm,xshift=-0.03cm] at (M.south) {};
\draw (ground.north east) -- (ground.north west);
\draw (ground.south east) -- (ground.south west);
\draw (ground.north east) -- (ground.south east);

% WEST WALL
\node (wall) [ground, rotate=-90, minimum width=3cm,xshift=-0.42cm,yshift=-3cm] {};
\draw (wall.north east) -- (wall.north west);
\draw (wall.north west) -- (wall.south west);
\draw (wall.south west) -- (wall.south east);
\draw (wall.south east) -- (wall.north east);

% DAMPER AND SPRING
\draw [spring] (wall.20) -- ($(M.north west)!(wall.20)!(M.south west)$);
% \node (y) at (wall.130) [yshift = 0.02cm,xshift=1.2cm] {$k$};

\end{tikzpicture}

\ifnum \Solutions=1 {\color{DarkBlue} \\[12pt] 
To determine the spring constant we use Hooks law, 
$$F = kx \quad \Rightarrow \quad 4 = 0.8k \quad \Rightarrow \quad k = 5$$
The equation of motion is $my''+\gamma y' + ky = f(t)$.
But $m=0.2, \gamma = 0.6, k = 5, f(t) = 0.5t$, so the IVP is
$$0.2y'' + 0.6 y' + 5y = 0.5t, \quad y(0) = 0.01, \quad y'(0) = 0$$
Don't forget that we need to include the initial conditions to have a complete initial value problem.
} 
\else 
\vspace{0.5cm}
\fi
\fi 



\ifnum \Version=2
\question[2] A small car with mass $m = 0.2$ kg is moving along a straight line on a horizontal surface. The car is attached to a spring that exerts a force of $4$ N when it is extended a distance of $0.2$ meters from its equilibrium position. Frictional forces also exert a force on the car while it is moving that is proportional to the velocity of the car. The proportionality constant for these frictional forces is $2$ N s/m. The car also has an engine that pushes the car forward with a force of $F = 3t^2$ N. The car is released after it is moved 0.01 m to the right of its equilibrium position, with an initial velocity of $0.1$ m/s. Assume that position of the car, $y$, increases as the car moves to theright. Write down the IVP based on this physical description that describes the position of the car as a function of time, $t$. You do not need to solve your IVP. \\
\input{2023Summer/Quiz4/DiagramCar}
\ifnum \Solutions=1 {\color{DarkBlue} \\[12pt] 
To determine the spring constant we use Hooks law, 
$$F = kx \quad \Rightarrow \quad 4 = 0.2k \quad \Rightarrow \quad k = 20$$
The equation of motion is $my''+\gamma y' + ky = f(t)$.
But $m=0.2, \gamma = 2, f(t) = 0.5t$, so the IVP is
$$0.2y'' + 2 y' + 20y = 3t^2, \quad y(0) = 0.01, \quad y'(0) = 0.1$$
Don't forget that we need to include the initial conditions to have a complete initial value problem.
} 
\else 
\vfill
\fi
\fi 



\ifnum \Version=3
\question[2] A small car with mass $m = 0.4$ kg is moving along a straight line on a horizontal surface. The car is attached to a spring that exerts a force of $4$ N when it is extended a distance of $0.2$ meters from its equilibrium position. Frictional forces also exert a force on the car while it is moving that is proportional to the velocity of the car. The proportionality constant for these frictional forces is $5$ N s/m. The car also has an engine that pushes the car forward with a force of $F = 3t$ N. The car is released after it is moved 0.01 m to the right of its equilibrium position, with an initial velocity of $0.1$ m/s. Assume that position of the car, $y$, increases as the car moves to the right. Write down the IVP based on this physical description that describes the position of the car as a function of time, $t$. You do not need to solve your IVP. \\
\hspace{1cm}\begin{tikzpicture}
\tikzstyle{spring}=[thick,decorate,decoration={zigzag,pre length=0.3cm,post length=0.3cm,segment length=6}]
\tikzstyle{damper}=[thick,decoration={markings,  
  mark connection node=dmp,
  mark=at position 0.5 with 
  {
    \node (dmp) [thick,inner sep=0pt,transform shape,rotate=-90,minimum width=15pt,minimum height=3pt,draw=none] {};
    \draw [thick] ($(dmp.north east)+(2pt,0)$) -- (dmp.south east) -- (dmp.south west) -- ($(dmp.north west)+(2pt,0)$);
    \draw [thick] ($(dmp.north)+(0,-5pt)$) -- ($(dmp.north)+(0,5pt)$);
  }
}, decorate]

% WALL PATTERN
\tikzstyle{ground}=[fill,pattern=north east lines,draw=none,minimum width=0.5cm,minimum height=0.3cm,inner sep=0pt,outer sep=0pt]

% CART
\node [style={draw,outer sep=0pt,very thick}] (M) [minimum width=1cm, minimum height=1.0cm] {$m$};
% WHEELS
\draw [very thick] (M.south west) ++ (0.2cm,-0.125cm) circle (0.125cm)  (M.south east) ++ (-0.2cm,-0.125cm) circle (0.125cm);

% BOTTOM GROUND
\node (ground) [ground,anchor=north,yshift=-0.25cm,minimum width=5.6cm,xshift=-0.03cm] at (M.south) {};
\draw (ground.north east) -- (ground.north west);
\draw (ground.south east) -- (ground.south west);
\draw (ground.north east) -- (ground.south east);

% WEST WALL
\node (wall) [ground, rotate=-90, minimum width=3cm,xshift=-0.42cm,yshift=-3cm] {};
\draw (wall.north east) -- (wall.north west);
\draw (wall.north west) -- (wall.south west);
\draw (wall.south west) -- (wall.south east);
\draw (wall.south east) -- (wall.north east);

% DAMPER AND SPRING
\draw [spring] (wall.20) -- ($(M.north west)!(wall.20)!(M.south west)$);
% \node (y) at (wall.130) [yshift = 0.02cm,xshift=1.2cm] {$k$};

\end{tikzpicture}

\ifnum \Solutions=1 {\color{DarkBlue} \\[12pt] 
To determine the spring constant we use Hooks law, 
$$F = kx \quad \Rightarrow \quad 4 = 0.2k \quad \Rightarrow \quad k = 20$$
The equation of motion is $my''+\gamma y' + ky = f(t)$.
But $m=0.4, \gamma = 5, f(t) = 3t$, so the IVP is
$$0.4y'' + 5 y' + 20y = 3t, \quad y(0) = 0.01, \quad y'(0) = 0.1$$
Don't forget that we need to include the initial conditions to have a complete initial value problem.
} 
\else 
\vfill
\fi
\fi 

\ifnum \Version=4
\question[2] A spring is hung from a horizontal surface as in the figure below. A mass of $m=0.4$ kg stretches the spring 0.08 m. Frictional forces also exert a force on the mass while it is moving that is proportional to the velocity of the mass. The proportionality constant for these frictional forces is 3 N s/m. The mass is then pulled down an additional 0.07 m and then released with an initial velocity of $1$ m/s. Assume that position of the mass, $y$, increases as the mass moves down. Write down an IVP based on this physical description that describes the position of the mass $y(t)$, as a function of time, $t$. You do not need to solve your IVP. You may use $g=10 m/s^2$.  \\[4pt]
\input{2023Summer/Quiz4/DiagramSpringMass}
\ifnum \Solutions=1 {\color{DarkBlue} \\[12pt] 
To determine the spring constant we use Hooks law, 
$$F = mg  = kx \quad \Rightarrow \quad 0.4g = 0.08k \quad \Rightarrow \quad k = 5g$$
We can use $g=10$ or $g=9.8$. The equation of motion is $my''+\gamma y' + ky = f(t)$.
But $m=0.4, \gamma = 3, f(t) = 0$, so the IVP is
$$0.4y'' + 3 y' + 5gy = 0, \quad y(0) = 0.07, \quad y'(0) = 1$$
Don't forget that we need to include the initial conditions to have a complete initial value problem.
} 
\else 
\vfill
\fi
\fi 

\ifnum \Version=5
\question[2] A spring is hung from a horizontal surface as in the figure below. A mass of $m=0.6$ kg stretches the spring 0.12 m. Frictional forces also exert a force on the mass while it is moving that is proportional to the velocity of the mass. The proportionality constant for these frictional forces is 2 N s/m. The mass is then pulled up an additional 0.1 m and then released with an initial velocity of $1$ m/s. Assume that position of the mass, $y$, increases as the mass moves down. Write down an IVP based on this physical description that describes the position of the mass $y(t)$, as a function of time, $t$. You do not need to solve your IVP. \\[4pt]
% \input{2023Summer/Quiz4/DiagramSpringMass}
\ifnum \Solutions=1 {\color{DarkBlue} \\[12pt] 
To determine the spring constant we use Hooks law, 
$$F = mg  = kx \quad \Rightarrow \quad 0.6g = 0.12k \quad \Rightarrow \quad k = 0.5g$$
We can use $g=10$ or $g=9.8$. The equation of motion is $my''+\gamma y' + ky = f(t)$.
But $m=0.6, \gamma = 2, f(t) = 0$, so the IVP is
$$0.6y'' + 2 y' + 0.5gy = 0, \quad y(0) = -0.1, \quad y'(0) = 1$$
Don't forget that we need to include the initial conditions to have a complete IVP.
} 
\else 
\vfill
\fi
\fi 





\ifnum \Version=6
\question[3] A toy car with mass $m = 0.2$ kg is moving along a straight line on a horizontal surface. See diagram below. The car is attached to a spring. The spring exerts a force of $2$ N when it is extended a distance of $0.1$ meters from its equilibrium position. Frictional forces also exert a force on the car while it is moving that is proportional to the velocity of the car. The proportionality constant for the frictional forces is $0.3$ N s/m. The car has an engine that pushes the car forward with a force of $F = 0.25$ N. The car is released from rest after it is moved 0.06 m to the left of its equilibrium position. Write down the IVP based on this physical description that describes the position of the car as a function of time, $t$. Do not solve your IVP but determine the values of any parameters in your DE. You can use $g = 10$ m/s$^2$. Assume that position of the car, $y$, increases as the car moves to the right. \\
\hspace{1cm}\begin{tikzpicture}
\tikzstyle{spring}=[thick,decorate,decoration={zigzag,pre length=0.3cm,post length=0.3cm,segment length=6}]
\tikzstyle{damper}=[thick,decoration={markings,  
  mark connection node=dmp,
  mark=at position 0.5 with 
  {
    \node (dmp) [thick,inner sep=0pt,transform shape,rotate=-90,minimum width=15pt,minimum height=3pt,draw=none] {};
    \draw [thick] ($(dmp.north east)+(2pt,0)$) -- (dmp.south east) -- (dmp.south west) -- ($(dmp.north west)+(2pt,0)$);
    \draw [thick] ($(dmp.north)+(0,-5pt)$) -- ($(dmp.north)+(0,5pt)$);
  }
}, decorate]

% WALL PATTERN
\tikzstyle{ground}=[fill,pattern=north east lines,draw=none,minimum width=0.5cm,minimum height=0.3cm,inner sep=0pt,outer sep=0pt]

% CART
\node [style={draw,outer sep=0pt,very thick}] (M) [minimum width=1cm, minimum height=1.0cm] {$m$};
% WHEELS
\draw [very thick] (M.south west) ++ (0.2cm,-0.125cm) circle (0.125cm)  (M.south east) ++ (-0.2cm,-0.125cm) circle (0.125cm);

% BOTTOM GROUND
\node (ground) [ground,anchor=north,yshift=-0.25cm,minimum width=5.6cm,xshift=-0.03cm] at (M.south) {};
\draw (ground.north east) -- (ground.north west);
\draw (ground.south east) -- (ground.south west);
\draw (ground.north east) -- (ground.south east);

% WEST WALL
\node (wall) [ground, rotate=-90, minimum width=3cm,xshift=-0.42cm,yshift=-3cm] {};
\draw (wall.north east) -- (wall.north west);
\draw (wall.north west) -- (wall.south west);
\draw (wall.south west) -- (wall.south east);
\draw (wall.south east) -- (wall.north east);

% DAMPER AND SPRING
\draw [spring] (wall.20) -- ($(M.north west)!(wall.20)!(M.south west)$);
% \node (y) at (wall.130) [yshift = 0.02cm,xshift=1.2cm] {$k$};

\end{tikzpicture}


\ifnum \Solutions=1 {\color{DarkBlue} 
\textbf{Solutions:} spring constant:
$$F = kx \quad \Rightarrow \quad k = \frac{2}{0.1} = 20$$
General form of damped forced oscillator:
$$my'' + \gamma y' + ky = F$$
Using given information and initial conditions, the IVP is
\begin{align}
    0.2 y'' + 0.3 y' + 20y = 0.25, \quad y(0) = -0.06, \quad y'(0) = 0
\end{align}
} 
\else 
\vfill
\fi
\fi 


\ifnum \Version=7

\question[3] A spring is hung from a horizontal surface as in the figure below. A mass of $m=0.2$ kg stretches the spring 0.04 m. Frictional forces also exert a force on the mass while it is moving that is proportional to the velocity of the mass. The proportionality constant for these frictional forces is 2 N s/m. The mass is then pulled down an additional 0.03 m and then released from rest. Assume that position of the mass, $y$, increases as the mass moves down. Write down an IVP based on this physical description that describes the position of the mass $y(t)$, as a function of time, $t$. Do not solve your IVP. You can use $g = 10$ m/s$^2$. \\[4pt]
\hspace{1cm}\begin{tikzpicture}

\tikzstyle{spring}=[thick,decorate,decoration={zigzag,pre length=0.3cm,post length=0.3cm,segment length=6}]

% WALL PATTERN
\tikzstyle{ground}=[fill,pattern=north east lines,draw=none,minimum width=0.75cm,minimum height=0.3cm,inner sep=0pt,outer sep=0pt]

% MASS
\node [style={draw,outer sep=0pt,very thick},yshift=-2cm] (M) [minimum width=1cm, minimum height=1.0cm] {$m$};

% WALL
\node (ground) [ground,anchor=north,yshift=3cm,minimum width=3.0cm,xshift=-0.03cm] at (M.south) {};
\draw (ground.north east) -- (ground.north west);
\draw (ground.south east) -- (ground.south west);
\draw (ground.north east) -- (ground.south east);
\draw (ground.north west) -- (ground.south west);

% SPRING
\draw [spring] (ground.south) -- (M.north);
% \node (y) at (ground.130) [yshift = 0.02cm,xshift=1.2cm] {$k$};
% \draw [spring] (wall.100) -- ($(M.north west)!(wall.100)!(M.south west)$);

\end{tikzpicture}


\ifnum \Solutions=1 {\color{DarkBlue} 
\textbf{Solutions:} spring constant:
$$F  = m g = kx \quad \Rightarrow \quad k = \frac{0.2\cdot g }{0.04} = 5g$$
It is ok to leave in terms of $g$, or to write $k=50$. Also ok to leave in an un-simplified form such as $\displaystyle k = \frac{0.2\cdot g }{0.04}$.
General form of damped forced oscillator:
$$my'' + \gamma y' + ky = F$$
Using given information and initial conditions, the IVP is
\begin{align}
    0.2 y'' + 2 y' + 5gy = 0, \quad y(0) = 0.03, \quad y'(0) = 0
\end{align}
The IVP must have a DE and initial conditions. 
} 
\else 
\vfill
\fi
\fi 
