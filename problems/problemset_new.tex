\documentclass{article}
\usepackage{geometry}
\usepackage{commath}

\geometry{
 top=1in,
 bottom=1in, 
 left=1.5in,  
 right=1.5in  
}

\begin{document}

% Heading
\begin{center}
\textbf{\Large True/False Questions MATH-2551} \\ 
\end{center}

\vspace{1cm}






\vspace{1cm}
\textbf {Question 1:} The dot product of two vectors is a scalar quantity that represents the projection of one vector onto the other.

\textbf{Answer:} True

\textbf{Explanation:} According to the given definition, the dot product of two vectors, $\mathbf{u} = \langle u_1, u_2, u_3 \rangle$ and $\mathbf{v} = \langle v_1, v_2, v_3 \rangle$, is given by the formula:

$$u \cdot v = u_1v_1 + u_2v_2 + u_3v_3$$

This mathematical expression represents the sum of the products of the corresponding components of the vectors. The result of this operation is a single numerical value, which is a scalar quantity.

The dot product has a geometric interpretation that relates it to the projection of u onto v. The projection of u onto v is a vector that lies in the direction of v and has a magnitude equal to the length of u times the cosine of the angle between u and v. The dot product of u and v is equal to the magnitude of the projection of u onto v.

Therefore, the statement that the dot product of two vectors is a scalar quantity that represents the projection of one vector onto the other is true.

\pagebreak 

    
\textbf {Question 2:} The dot product of two vectors is a measure of the area of the parallelogram defined by the vectors.

\textbf{Answer:} False.

\textbf{Explanation:} The dot product of two vectors $$ \vec{u} = \langle u_1, u_2 \rangle $$ and $$ \vec{v} = \langle v_1, v_2 \rangle  $$ is defined by:

$$ \vec{u} \cdot \vec{v} = u_1 v_1 + u_2 v_2 $$

This quantity is a scalar value that represents the cosine of the angle between the vectors and the product of their magnitudes.

The area of a parallelogram defined by two vectors is given by:

$$ \text{Area} = \|\vec{u} \times \vec{v}\| $$

where the cross product $$\vec{u}\times\vec{v}$$ gives a vector perpendicular to both $$\vec{u}$$ and $$\vec{v}$$, and its magnitude is equal to the area of the parallelogram.

Therefore, the dot product of two vectors is not a measure of the area of the parallelogram defined by the vectors.

\pagebreak 

    
\textbf {Question 3:} Question:

True or False: Orthogonal vectors are always parallel to each other.

\textbf{Answer:} False.

\textbf{Explanation:} The definition of orthogonal vectors states that they have a dot product of zero. The dot product of two vectors is a scalar quantity that represents the projection of one vector onto the other. A dot product of zero indicates that the vectors are perpendicular to each other, not parallel.

\pagebreak 

    
\textbf {Question 4:} Question:
True or False: The direction cosines of a nonzero vector are unique for any given three-dimensional coordinate system.

\textbf{Answer:} True.

\textbf{Explanation:} The direction angles of a nonzero vector are determined by the angles formed between the vector and each of the coordinate axes. These angles are unique for any given vector and coordinate system. The direction cosines, which are the cosines of these angles, are also unique since they are determined by the direction angles.

In other words, if two vectors have different direction angles in a particular coordinate system, then they will also have different direction cosines. Conversely, if two vectors have the same direction angles in a particular coordinate system, then they will also have the same direction cosines.

\pagebreak 

    
\textbf {Question 5:} Question:

True or False: The magnitude of the cross product of two non-zero vectors is proportional to the area of the parallelogram formed by those vectors.

\textbf{Answer:} True.

\textbf{Explanation:} According to the given definition, the magnitude of the cross product of two vectors $u$ and $v$, denoted by $\Vert u \times v \Vert$, is given by:

$$\Vert u \times v \Vert = \Vert u \Vert \cdot \Vert v \Vert \cdot \sin \theta$$

where $\theta$ is the angle between $u$ and $v$.

The area of the parallelogram formed by $u$ and $v$ can be expressed as the product of the magnitudes of $u$ and $v$ multiplied by the sine of $\theta$:

$$Area = \Vert u \Vert \cdot \Vert v \Vert \cdot \sin \theta$$

Therefore, we have:

$$\Vert u \times v \Vert = Area$$

which implies that the magnitude of the cross product is indeed proportional to the area of the parallelogram formed by the two vectors.

\pagebreak 

    
\textbf {Question 6:} Question:

If two vectors, u and v, form the sides of a parallelogram, then the area of the parallelogram is always equal to the magnitude of the vector $u \times v$.

\textbf{Answer:} True.

\textbf{Explanation:} The given definition explicitly states that the area of a parallelogram with sides u and v is given by the magnitude of their cross product:

$$Area = \Vert \mathbf{u} \times \mathbf{v} \Vert$$

This formula captures the geometric relationship between the lengths and orientations of the vectors and the area they enclose. The cross product of two vectors in three-dimensional space represents a vector perpendicular to both u and v, with its magnitude proportional to the area of the parallelogram formed by those vectors.

\pagebreak 

    
\textbf {Question 7:}
Determine whether the following statement is true or false:
The triple scalar product of vectors \(u\), \(v\), and \(w\) can be calculated as 
\(u_1(v_2w_3 - v_3w_2) - u_2(v_1w_3 - v_3w_1) + u_3(v_1w_2 - v_2 w_1)\) 
where \(u_j\) and \(v_j\) represent the \(j\)th components of the vectors \(u\) and \(v\), respectively.


\textbf{Answer:} 
True

\textbf{Explanation:} 

The triple scalar product can be expressed using the determinant of the matrix with rows $u, v$, and $w$:

$$u \cdot (v \times w) = \begin{vmatrix} u_1 & u_2 & u_3 \\\ v_1 & v_2 & v_3 \\\ w_1 & w_2 & w_3 \end{vmatrix}$$

Expanding the determinant along the first row gives the formula provided in the question.

\pagebreak 

    
\textbf {Question 8:} Question:

Determine if the following statement is true or false:

The volume of a parallelepiped is always positive, regardless of the orientation of its edges.

\textbf{Answer:} False

\textbf{Explanation:} The volume of a parallelepiped is given by the absolute value of the triple scalar product:

$$V = |u\cdot (v \times w)|$$

where u, v, and w are the adjacent edges of the parallelepiped.

The triple scalar product calculates the signed volume of the parallelepiped. The sign of the volume depends on the orientation of the edges. If the edges form a right-handed coordinate system, the volume is positive. If they form a left-handed coordinate system, the volume is negative.

Therefore, the volume of a parallelepiped is not always positive but depends on the orientation of its edges.

\pagebreak 

    
\textbf {Question 9:} Question

True or False: The distance between a point M and a line L in three-dimensional space is independent of the choice of point P on the line L that is used in the formula $d = \frac{\Vert PM \times v \Vert}{\Vert v \Vert}$.

\textbf{Answer:} 
True.

\textbf{Explanation:} 

The formula $d = \frac{\Vert PM \times v \Vert}{\Vert v \Vert}$ gives the distance between a point M and a line L passing through a point P with direction vector $\mathbf{v}$. The key observation is that the cross product $PM \times v$ is perpendicular to both $\overrightarrow{PM}$ and $\mathbf{v}$. This means that the length of $PM \times v$ is the same for any point P on the line L. Therefore, the distance $d$ is independent of the choice of P on L.

In other words, the distance from M to L is the length of the vector $\overrightarrow{PM} \times \mathbf{v}$ projected onto the direction of $\mathbf{v}$. Since the direction of $\mathbf{v}$ is the same at any point on L, the length of this projection is the same regardless of which point P is chosen on L.



\pagebreak 

    
\textbf {Question 10:} Question
Given a vector n and a point P, the vector equation $n \cdot PQ = \theta$ represents any plane that passes through P and is perpendicular to n.

\textbf{Answer:} 
True.

\textbf{Explanation:} 
$n \cdot PQ = \theta$ satisfies the definition of a plane because it denotes a set of points Q that are collinear with the vector n. This means that every vector PQ→ is orthogonal to the normal vector n, which implies that all vectors emanating from P to points on the plane are perpendicular to n. Hence, the given vector equation does indeed represent any plane that passes through P and is perpendicular to n.



\pagebreak 

    
\end{document}
