\documentclass{article}
\usepackage{geometry}
\usepackage{commath}

% \geometry{
%  top=1in,
%  bottom=1in, 
%  left=1.5in,  
%  right=1.5in  
% }
%
\begin{document}
% ... (Rest of your content) ...

% Heading
\begin{center}
\textbf{\Large True/False Questions MATH-2551} \\ % Adjust title/style if desired
\end{center}

\vspace{1cm}






\vspace{1cm}
                \textbf {Question:} \\( \hspace{0.5 cm} \\) Consider the parametric equations $x = t^2$ and $y = t + 1$ where $t$ is a real number. Is the statement that the parametric curve has a cusp at $t = 0$ true or false? Explain your answer.
                
                \textbf{Answer:} False
                
                \textbf{Explanation:} A cusp occurs when the derivative of a parametric curve is undefined or zero and the second derivative is also zero at the same point. For the given parametric equations, the derivatives are $dx/dt = 2t$ and $dy/dt = 1$. At $t = 0$, both derivatives are defined and nonzero, so there is no cusp at that point. Therefore, the statement is false.
                
                \textbf{Hint:} Hint: Consider the derivatives of the parametric equations at $t = 0$.
                \vspace{0.5cm} 
        
            
                \textbf {Question:} **True or False:** The eccentricity of a conic section is defined as the ratio of the perpendicular distance from the point to the directrix to the distance from the point to the focus.
                
                \textbf{Answer:} True
                
                \textbf{Explanation:} The definition of eccentricity is the ratio of the distance from the point to the focus to the perpendicular distance from the point to the nearest directrix. This value is constant for any conic section and can define the type of conic section: a parabola if $e=1$, an ellipse if $e<1$, and a hyperbola if $e>1$. For example, consider an ellipse with foci at $(-c,0)$ and $(c,0)$ and vertices at $(-a,0)$ and $(a,0)$. The eccentricity of this ellipse is $e=\\frac{c}{a}$, which is constant for all points on the ellipse.
                
                \textbf{Hint:} Hint: The focus and directrix are two important aspects of the definition of eccentricity.
                \vspace{0.5cm} 
        
            
                \textbf {Question:} **True or False:** A quantity that has only magnitude but no direction can be considered a vector.
                
                \textbf{Answer:} False
                
                \textbf{Explanation:} By definition, a vector is a quantity that possesses both magnitude and direction. A quantity with only magnitude and no specified direction is not a vector. For example, the distance from point A to B is a scalar quantity with only magnitude, not a vector.
                
                \textbf{Hint:} Hint: By definition, a vector includes both magnitude and direction.
                \vspace{0.5cm} 
        
            
                \textbf {Question:} If a vector \\(\\mathbf{v}\\) has a magnitude of $3$ and a scalar $k$ is negative, then the magnitude of the vector $k\\mathbf{v}$ is:
                
                \textbf{Answer:} False
                
                \textbf{Explanation:} According to the definition of scalar multiplication, the magnitude of the vector $k\\mathbf{v}$ is $\\left|k\\right|\\left|\\mathbf{v}\\right|$. Since $k$ is negative, $\\left|k\\right| = -k$. Therefore, the magnitude of the vector $k\\mathbf{v}$ is $-k\\left|\\mathbf{v}\\right| = -(-3)\\left|\\mathbf{v}\\right| = 3\\left|\\mathbf{v}\\right| = 9$. So, the magnitude of the vector $k\\mathbf{v}$ is $9$, not $3$.
                
                \textbf{Hint:} Hint: The magnitude of the vector is equal to the scalar's magnitude multiplied by the vector's magnitude.
                \vspace{0.5cm} 
        
            
                \textbf {Question:} **True or False:** In vector addition, the vector sum of two vectors \\(\\mathbf{v}\\) and \\(\\mathbf{w}\\) can be found by translating \\(\\mathbf{w}\\) so that its initial point coincides with the terminal point of \\(\\mathbf{v}\\).
                
                \textbf{Answer:} **True**
                
                \textbf{Explanation:} According to the definition of vector addition given, the vector sum \\(\\mathbf{v} + \\mathbf{w}\\) is constructed by placing the initial point of \\(\\mathbf{w}\\) at the terminal point of \\(\\mathbf{v}\\). This means that the vector \\(\\mathbf{w}\\) is shifted or translated so that its initial point coincides with the terminal point of \\(\\mathbf{v}\\).
                
                \textbf{Hint:} Hint: Consider the definition of vector addition.
                \vspace{0.5cm} 
        
            
                \textbf {Question:} True or False: The vector with initial point (1,2) and terminal point (3,5) can be written in component form as $v$=(2,3).
                
                \textbf{Answer:} False
                
                \textbf{Explanation:} The vector with initial point (1,2) and terminal point (3,5) can be written in component form as v=(3-1,5-2)=(2,3).
                
                \textbf{Hint:} Hint: Calculate the difference between the coordinates of the terminal and initial points to determine the component form.
                \vspace{0.5cm} 
        
            
                \textbf {Question:} True or False: The radius of a sphere is the distance between any two points on the surface of the sphere.
                
                \textbf{Answer:} False
                
                \textbf{Explanation:} The radius of a sphere is the distance from the center of the sphere to any point on the surface of the sphere.  For example, if a sphere has a radius of 5 cm, then the distance from the center of the sphere to any point on the surface of the sphere is 5 cm.
                
                \textbf{Hint:} Hint: The radius of a sphere is related to the distance between the center of the sphere and a point on its surface.
                \vspace{0.5cm} 
        
            
                \textbf {Question:} Determine if the following statement is true or false:
                The dot product of two vectors only depends on their direction.
                
                \textbf{Answer:} True
                
                \textbf{Explanation:} The dot product of two vectors u=(u1,u2,u3) and v=(v1,v2,v3) is defined as u$\cdot$v=u1v1+u2v2+u3v3. This formula shows that the dot product is the sum of the products of the corresponding components of the two vectors. Therefore, the dot product only depends on the direction of the vectors, not their magnitudes.
                
                \textbf{Hint:} Hint: The dot product solely relies on the vectors' orientations, not their magnitudes.
                \vspace{0.5cm} 
        
            
                \textbf {Question:} **True or False:** The direction cosines of a nonzero vector are always positive.
                
                \textbf{Answer:} False
                
                \textbf{Explanation:} The direction cosines of a vector are the cosines of the angles formed by the vector and the coordinate axes. These angles can be acute, obtuse, or right, so the direction cosines can be positive or negative. For example, the direction cosines of the vector $\mathbf{v} = (1, 1, 1)$ are $(\frac{1}{\sqrt{3}}, \frac{1}{\sqrt{3}}, \frac{1}{\sqrt{3}})$. The first two direction cosines are positive because the angles between $\mathbf{v}$ and the $x$- and $y$-axes are acute. However, the third direction cosine is negative because the angle between $\mathbf{v}$ and the $z$-axis is obtuse.
                
                \textbf{Hint:} Hint: The direction cosines of a vector are the cosines of the angles formed by the vector and the coordinate axes.
                \vspace{0.5cm} 
        
            
                \textbf {Question:} $proju_v$ is a vector that has the same initial point as $\vec{u}$and $\vec{v}$, and the same direction as $\vec{u}$.
                
                \textbf{Answer:} True
                
                \textbf{Explanation:} The definition of the vector projection of \(\\vec{v}\) onto \(\\vec{u}\), denoted by \(\\text{proju}_v\), states that it is the vector with the same initial point as \(\\vec{u}\) and \(\\vec{v}\), and the same direction as \(\\vec{u}\).
                
                \textbf{Hint:} Hint: Think about the initial point and direction of \(\\text{proju}_v\).
                \vspace{0.5cm} 
        
            
\end{document}
